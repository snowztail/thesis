\tikzstyle{glass}=[color=blue!10]
\tikzset{
	light beam/.style={decoration={markings,
					mark=at position 0.5 with {\arrow[xshift=3pt]{latex}}},
			postaction={decorate}},
	light beam/.default=0.5}


% REFLECTION & REFRACTION
\begin{tikzpicture}
	\def\w{3.2}     % width interface
	\def\d{1}     % distance between source and glass
	\def\l{3.5}     % distance from glass
	\def\t{2}     % thickness glass
	% \def\na{1.0}  % air
	% \def\ng{2}    % glass
	\def\n{0.5}

\begin{scope}[rotate=90]
	% MEDIUM
	\coordinate (L) at (-\w,-\d);              % lower-left point interface
	\fill[glass] (L) rectangle++ (2*\w,-\t);   % glass

% 	\foreach \anga in {-30,-20,...,30} {
	\foreach \anga in {-30,0,...,30} {
		\pgfmathsetmacro{\angg}{asin(\n*sin(\anga))}

		\pgfmathsetmacro{\a}{\d/cos(\anga)}
		\pgfmathsetmacro{\b}{\t/cos(\angg)}
		\pgfmathsetmacro{\c}{\l/cos(\anga)}

		\coordinate (S) at (0,0);                         % ray source
		\coordinate (I) at (\anga-90:\a);                 % impinging point
		\coordinate (O) at ($(I) + ({\angg-90}:\b)$);     % departing point
		\coordinate (E) at ($(O) + ({\anga-90}:\c)$);     % refraction tail

		% LINES
		\draw[light beam,thick] (S) -- (I);    % incoming ray
		\draw[light beam,thick] (I) -- (O);     % refracted ray in glass
		\draw[light beam,thick] (O) -- (E);     % refracted ray in air
	}
	\node at (-2.6,-2) {$n=2$};
\end{scope}
\end{tikzpicture}
