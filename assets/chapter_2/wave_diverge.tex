\tikzstyle{glass}=[color=blue!10]
\tikzset{
	light beam/.style={decoration={markings,
					mark=at position 0.5 with {\arrow[xshift=3pt]{latex}}},
			postaction={decorate}},
	light beam/.default=0.5}


% REFLECTION & REFRACTION
\begin{tikzpicture}
	\def\L{4}     % width interface
	\def\l{1.5}   % length ray
	\def\t{1}     % depth glass gradient
	\def\h{2}     % bisector height
	\def\f{0.5}   % fraction of interface to the left
	\def\na{1.0}  % air
	\def\ng{2}    % glass
	\def\anga{35} % angle of incident ray
	\def\angg{asin(\na/\ng*sin(\anga))}
	\def\angm{2*asin(\ng/\na*sin(\angg))}

	\coordinate (I) at (0,0);                                         % 1st boundary
	\coordinate (O) at ({-90+\angg}:{\t/cos(\angg)});                 % 2nd boundary
	\coordinate (S) at (90+\anga:\l);                                  % incident
	\coordinate (E) at ($ (O) + ({-90+\anga}:\l)$);                  % ordinary refraction tail
	\coordinate (L) at (-\f*\L,0);                                     % left point interface
	\coordinate (R) at ({(1-\f)*\L},0);                                % right point interface

	% MEDIUM
	\fill[glass] (L) rectangle++ (\L,-\t); % glass gradient

	% LINES
	\draw[light beam,black] (S) -- (I);    % incoming ray
	\draw[light beam,blue] (I) -- (O);    % refracted ray
	\draw[light beam,blue] (O) -- (E);    % ordinary refraction

\end{tikzpicture}
