% ************************** Thesis Abstract *****************************
% Use `abstract' as an option in the document class to print only the titlepage and the abstract.
\begin{abstract}
	\gls{ris} is the most promising physical-layer technology for 6G.
	It adopts many low-power scattering elements to customize the propagation environment for improved network performance, unlocking a new potential of \emph{channel design} as never before.
	In this dissertation, we first provide an overview of its structure, characteristics, applications, principles, and models, then discuss how it addresses the key issues in \gls{swipt} and compare it with \gls{bc} in terms of functionalities, principles, models, and preferences.
	The work chapters investigate three typical use cases of \gls{ris}: Joint design with transceiver for a particular objective (beamforming); ride its own information over legacy networks (modulation); and manipulate the wireless environment as a stand-alone device (channel shaping).
	In particular, we address the following topics:
	\begin{itemize}
		\item \emph{\gls{ris}-aided \gls{swipt}:} We introduce \gls{ris} into multi-antenna, multi-carrier \gls{swipt} systems, investigating joint waveform and beamforming design for maximizing harvested energy under various communication rate constraints. Algorithms, performance trade-offs, and asymptotic behaviors are analyzed.
		\item \emph{RIScatter:} This novel scatter protocol integrates \gls{ris} and \gls{bc} from an input distribution perspective. The reflection pattern is exploited simultaneously for primary-link beamforming and backscatter-link modulation. We propose a practical cooperative receiver, characterize the achievable rate region, and analyze the impact of system parameters.
		\item \emph{\gls{mimo} channel shaping:} We exploit an advanced \gls{bd}-\gls{ris} architecture for singular value redistribution and power maximization in \gls{pc}, and leakage interference minimization in \gls{ic}.
		Their implications on rate-optimal joint designs are also discussed.
		We highlight the unique features of \gls{bd}-\gls{ris} and propose an efficient design framework.
	\end{itemize}
\end{abstract}
