%!TEX root = ../thesis.tex

\chapter{Conclusions and Future Works}\label{ch:conclusions}

\begin{section}{Conclusions}
	This thesis explored the versatile potential of \gls{ris} for augmenting future wireless networks from a physical-layer perspective.
	After introducing the necessary background, we focused on three most promising directions (passive beamforming, backscatter modulation, channel shaping) and investigated those in different application scenarios:
	\begin{itemize}
		\item In the domain of \gls{swipt}, we unveiled new trade-offs and design insights by introducing \gls{ris} into multi-antenna, multi-carrier systems. Considering harvester nonlinearity, we characterize the achievable \gls{r-e} region through a joint optimization of waveform, active and passive beamforming based on the \gls{csit}. This problem is solved by \gls{bcd}, where we obtain the active precoder in closed form, the passive beamforming by \gls{sca}, and the waveform amplitude by \gls{gp}. To facilitate practical implementation, we also propose two low-complexity designs based on closed-form adaptive waveform schemes. Simulation results highlight a second-order array gain and a fourth-order power scaling order with respect to \gls{ris} elements, while being robust to \gls{csi} inaccuracy and \gls{ris} resolution.
		\item The proposed RIScatter concept demonstrated the synergy between active and passive transmissions. RIScatter is a batteryless cognitive radio that recycles ambient signal in an adaptive and customizable manner, where dispersed or co-located scatter nodes partially modulate their information and partially engineer the wireless channel.
		The key is to render the probability distribution of reflection states as a joint function of the information source, \gls{csi}, and relative priority of coexisting links.
		This enables RIScatter to softly bridge \gls{bc} and \gls{ris}; reduce to either in special cases; or evolve in a mixed form for heterogeneous traffic control and universal hardware design.
		We also propose a low-complexity \gls{sic}-free receiver that exploits the properties of RIScatter.
		For a single-user multi-node network, we characterize the achievable primary-(total-)backscatter rate region by optimizing the input distribution at scatter nodes, the active beamforming at the \gls{ap}, and the energy decision regions at the user.
		Simulation results demonstrate the received signal can be exploited for dual purposes and the nodes can shift smoothly between backscatter modulation and passive beamforming via input distribution control.
		\item Finally, we investigated the fundamental limits of \gls{ris} for reshaping \gls{mimo} \gls{pc} and \gls{ic}. We depart from the widely-adopted diagonal phase shift model to a general \gls{bd} architecture with superior subspace alignment and subchannel rearrangement capabilities.
		An efficient geodesic \gls{rcg} algorithm is tailored for smooth \gls{bd}-\gls{ris} optimization problems, which features lower complexity and faster convergence.
		For \gls{mimo}-\gls{pc}, we characterize the achievable channel singular value region through numerical and analytical approaches, then tackle the rate maximization problem by a local-optimal \gls{ao} approach and a low-complexity shaping-inspired approach.
		For \gls{mimo}-\gls{ic}, we extend the proposed algorithms to solve leakage interference minimization and \gls{wsr} maximization problems.
		Simulation results suggest channel shaping offers a promising path to decouple the joint \gls{ris}-transceiver design with minor performance degradation and much lower computational complexity.
	\end{itemize}
\end{section}

\begin{section}{Future Works}
	Many interesting problems are countered during the research but not fully addressed due to time constraints, to name a few:
	\begin{itemize}
		\item \emph{\gls{ris}-aided multi-user \gls{swipt}:} Extension to multi-user scenario is non-trivial because (i) the waveform and passive beamforming are shared among users while the active beamforming and power splitting ratios are user-specific; (ii) the interference can be harvested for power purpose such that dedicated information and power precoders are desired; (iii) due to harvester nonlinearity, the optimal design is highly non-convex and depends heavily on the multiple access scheme; (iv) a proper criterion is needed to cover the \gls{r-e} trade-off and fairness among users.
		\item \emph{Capacity region of RIScatter:} The current work considers a decoding order from backscatter to primary, corresponding to a primary-optimal point (but not a corner due to energy detector) in the capacity region.
		If the primary message can be decoded first, it can be modelled as a spreading sequence that enhances the backscatter \gls{snr} by $N$ times (where $N$ is the symbol period ratio) \cite{Long2020a}, corresponding to a backscatter-optimal point.
		Besides, the backscatter detection can be performed coherently in the signal domain to further improve the achievable rate.
		\item \emph{RIScatter, index modulation, and rate splitting:} In RIScatter, the primary and backscatter symbols are superimposed by multiplicative coding, and the nodes usually have a finite number of reflection states. Instead of acting as a pure channel controller or individual information source, a \gls{ris} can be integrated into the \gls{rf} transmitter as an additional index modulator to enhance the spectral efficiency \cite{Basar2020}.
		The transmitted message is thus split into an amplitude-phase modulated component and an index-modulated component, which can be decoded separately at the receiver \cite{Zhong2018}. This could further improve the capacity of \gls{ris}-aided \gls{mimo} systems \cite{Ye2022}, or make a low-complexity realization of \gls{rsma} \cite{Mao2018} in a multi-user system where the reflection pattern conveys the common message.
		\item \emph{RIScatter and sensing:} The proposed RIScatter concept can be extended to sensing applications, where the scatter nodes are equipped with sensors to monitor the environment and report the information to the \gls{ap}. The challenge is to design the input distribution and reflection states to stabilize the environment for improved sensing performance while guaranteeing the communication quality.
		\item \emph{\gls{bd}-\gls{ris} and frequency shifting:} Similar to \gls{ambc}, the different groups of \gls{bd}-\gls{ris} can be used to shift the incident wave to different frequencies.
		This phenomenon can be exploited to mitigate the self and inter-operator interference \cite{Miridakis2024}, generate a multi-frequency signal out of a \gls{cw}, or manipulate the response of frequency-selective channels. However, it could be challenging to design and implement the time-varying scattering matrices at a low power consumption.
		\item \emph{Multi-sector \gls{bd}-\gls{ris} and interference alignment:} The current work solves leakage interference minimization and \gls{wsr} maximization problems in \gls{ris}-aided \gls{ic}.
		It remains unknown how many scattering elements are required to completely eliminate the interference, and what is the \gls{dof} in a $(N_\mathrm{T},N_\mathrm{S},N_\mathrm{R})$ \gls{ic}.
		Besides, there exists a multi-sector model \cite{Li2023c} that provides full-space coverage and highly directional beams.
		It would be interesting to see how those sector-specific (instead of being shared by all users) scattering matrices can help to align the interference and improve the fairness issue.
	\end{itemize}
\end{section}
