%!TEX root = ../thesis.tex
% ******************************* Thesis Appendix A ****************************
\chapter{Appendix}


% \begin{section}{Proofs for Chapter \ref{ch:ris_aided_swipt}}
% 	\begin{subsection}{Proof of Proposition~\ref{pr:relaxation}}\label{ap:relaxation}
% 		For any feasible $\bar{\mathbf{\Theta}}$ to problem~\eqref{op:irs}, $\mathrm{tr}(\bar{\mathbf{\Theta}})=L+1$ always holds because of the modulus constraint~\eqref{co:irs_modulus}.
% 		Therefore, problem~\eqref{op:irs} can be recast as
% 		\begin{maxi!}
% 			{\scriptstyle{\bar{\mathbf{\Theta}}}}{\tilde{z}(\bar{\mathbf{\Theta}})-\mathrm{tr}(\bar{\mathbf{\Theta}})}{\label{op:irs_equivalent}}{\label{ob:irs_equivalent}}
% 			\addConstraint{R(\bar{\mathbf{\Theta}}) \ge \bar{R}}\label{co:irs_equivalent_rate}
% 			\addConstraint{\mathrm{diag}^{-1}(\bar{\mathbf{\Theta}})=\mathbf{1}}\label{co:irs_equivalent_modulus}
% 			\addConstraint{\bar{\mathbf{\Theta}}\succeq{\mathbf{0}}}\label{co:irs_equivalent_sd}
% 			\addConstraint{\mathrm{rank}(\bar{\mathbf{\Theta}})=1.\label{co:irs_equivalent_rank}}
% 		\end{maxi!}
% 		It is straightforward to verify the convex problem~\eqref{ob:irs_equivalent}--\eqref{co:irs_equivalent_sd} satisfies the Slater's condition and strong duality holds \cite{Boyd2004}.
% 		The corresponding Lagrangian function at iteration $r$ is
% 		\begin{equation}\label{eq:lagrangian}
% 			\begin{split}
% 				\mathcal{L}
% 				& = \mathrm{tr}\left(\bar{\mathbf{\Theta}}^{(r)}\right) - \frac{1}{2} \beta_2 \rho \mathrm{tr}\Bigl(
% 						(\mathbf{C}_{\mathrm{I},0} + \mathbf{C}_{\mathrm{P},0}) \bar{\mathbf{\Theta}}^{(r)}
% 					\Bigr) \\
% 				& \quad - \frac{3}{4} \beta_4 \rho^2 \Biggl(
% 						2 t_{\mathrm{I},0}^{(r-1)} \mathrm{tr}\Bigl(
% 							\mathbf{C}_{\mathrm{I},0} \bar{\mathbf{\Theta}}^{(r)}
% 						\Bigr) + \sum_{k=-N+1}^{N-1} (t_{\mathrm{P},k}^{(r-1)})^* \mathrm{tr}\Bigl(
% 							\mathbf{C}_{\mathrm{P},k} \bar{\mathbf{\Theta}}^{(r)}
% 						\Bigr)\\
% 				& \qquad + 2 t_{\mathrm{P},0}^{(r-1)} \mathrm{tr}\Bigl(
% 						\mathbf{C}_{\mathrm{I},0} \bar{\mathbf{\Theta}}^{(r)}
% 					\Bigr) + 2 t_{\mathrm{I},0}^{(r-1)} \mathrm{tr}\Bigl(
% 						\mathbf{C}_{\mathrm{P},0} \bar{\mathbf{\Theta}}^{(r)}
% 					\Bigr)
% 					\Biggr)\\
% 				& \quad + \mu \Biggl(
% 					2^{\bar{R}} - \prod_{n=1}^N \biggl(
% 						1 + \frac{(1-\rho) \mathrm{tr}\Bigl(
% 							\mathbf{C}_n \bar{\mathbf{\Theta}}^{(r)}
% 						\Bigr)}{\sigma_n^2}
% 					\biggr)
% 				\Biggr)\\
% 				& \quad + \mathrm{tr}\biggl(
% 					\mathrm{diag}(\mathbf{\nu}) \odot \Bigl(
% 						\bar{\mathbf{\Theta}}^{(r)} \odot \mathbf{I} - \mathbf{I}
% 					\Bigr)
% 				\biggr) - \mathrm{tr} \Bigl(
% 					\mathbf{\Upsilon} \bar{\mathbf{\Theta}}^{(r)}
% 				\Bigr) + \zeta,
% 			\end{split}
% 		\end{equation}
% 		where $\mu$, $\mathbf{\nu}$, $\mathbf{\Upsilon}$ denote respectively the scalar, vector and matrix Lagrange multiplier associated with constraint~\eqref{co:irs_equivalent_rate}, \eqref{co:irs_equivalent_modulus} and \eqref{co:irs_equivalent_sd}, and $\zeta$ collects all terms irrelevant to $\bar{\mathbf{\Theta}}^{(r)}$.
% 		The \gls{kkt} conditions on the primal and dual solutions are
% 		\begin{subequations}
% 			\begin{equation}\label{eq:lagrange_multiplier}
% 				\mu^\star \ge 0, \mathbf{\Upsilon}^\star \succeq \mathbf{0},
% 			\end{equation}
% 			\begin{equation}\label{eq:complementary_slackness}
% 				\mathbf{\nu}^\star \odot \mathrm{diag}^{-1}(\bar{\mathbf{\Theta}}^\star) = \mathbf{0}, \mathbf{\Upsilon}^\star \bar{\mathbf{\Theta}}^\star = \mathbf{0},
% 			\end{equation}
% 			\begin{equation}\label{eq:gradient}
% 				\nabla_{\bar{\mathbf{\Theta}}^\star} \mathcal{L} = 0.
% 			\end{equation}
% 		\end{subequations}
% 		We then express \eqref{eq:gradient} explicitly as $\mathbf{\Upsilon}^\star = \mathbf{I} - \mathbf{\Delta}^\star$, where
% 		\begin{equation}
% 			\begin{split}\label{eq:delta}
% 				\mathbf{\Delta}^\star
% 				& = \frac{1}{2} \beta_2 \rho (\mathbf{C}_{\mathrm{I},0}+\mathbf{C}_{\mathrm{P},0})\\
% 				& \quad + \frac{3}{4} \beta_4 \rho^2
% 					\Biggl(
% 						2 t_{\mathrm{I},0}^{(r-1)} \mathbf{C}_{\mathrm{I},0} + \sum_{k=-N+1}^{N-1} (t_{\mathrm{P},k}^{(r-1)})^* \mathbf{C}_{\mathrm{P},k} + 2 t_{\mathrm{P},0}^{(r-1)} \mathbf{C}_{\mathrm{I},0} + 2 t_{\mathrm{I},0}^{(r-1)} \mathbf{C}_{\mathrm{P},0}
% 					\Biggr)\\
% 				& \quad + \mu^\star \sum_{n=1}^N \frac{(1-\rho) \mathbf{C}_n}{\sigma_n^2} \prod_{n'=1,n' \ne n}^N \Biggl(
% 					1 + \frac{(1-\rho)\mathrm{tr}\Bigl(
% 						\mathbf{C}_{n'} \bar{\mathbf{\Theta}}^\star
% 					\Bigr)}{\sigma_{n'}^2}
% 				\Biggr) - \mathrm{diag}(\mathbf{\nu^\star}).
% 			\end{split}
% 		\end{equation}
% 		Note that \eqref{eq:complementary_slackness} suggests $\mathrm{rank}(\mathbf{\Upsilon}^\star)+\mathrm{rank}(\bar{\mathbf{\Theta}}^\star) \le L+1$. By reusing the proof in \cite[Appendix~A]{Xu2020}, we conclude $\mathrm{rank}(\mathbf{\Upsilon}^\star) \ge L$. On the other hand, $\bar{\mathbf{\Theta}}^\star$ cannot be $\mathbf{0}$ and $\mathrm{rank}(\bar{\mathbf{\Theta}}^\star) \ge 1$. Therefore, any optimal solution $\bar{\mathbf{\Theta}}^\star$ to the relaxed problem~\eqref{op:irs_equivalent} must be rank-\num{1}. Due to the equivalence between \eqref{ob:irs} and \eqref{ob:irs_equivalent}, $\bar{\mathbf{\Theta}}^\star$ is also optimal to the relaxed problem~\eqref{op:irs}. The proof is completed.
% 	\end{subsection}

% 	\begin{subsection}{Proof of Proposition~\ref{pr:sca}}\label{ap:sca}
% 		The objective function \eqref{ob:irs} is non-decreasing over iterations because the solution to \eqref{ob:irs}--\eqref{co:irs_sd} at iteration $r{-}1$ is still feasible at iteration $r$. As $r \to \infty$, $\tilde{z}(\bar{\mathbf{\Theta}}^{(r)})$ is bounded above because of the unit-modulus constraint \eqref{co:irs_modulus}. Thus, Algorithm~\ref{al:sca} is guaranteed to converge. Besides, we notice that Algorithm~\ref{al:sca} is an inner approximation algorithm \cite{Marks1978a}, because $\tilde{z}(\bar{\mathbf{\Theta}}) \le z(\bar{\mathbf{\Theta}})$, $\partial\tilde{z}(\bar{\mathbf{\Theta}}^{(r)})/\partial\bar{\mathbf{\Theta}}=\partial z(\bar{\mathbf{\Theta}}^{(r)})/\partial\bar{\mathbf{\Theta}}$ and the approximation \eqref{eq:taylor_1}--\eqref{eq:taylor_3} are asymptotically tight as $r \to \infty$ \cite{Li2013}. Therefore, it is guaranteed to provide a local optimal $\bar{\mathbf{\Theta}}^{\star}$ to the relaxed passive beamforming problem. According to Proposition~\ref{pr:relaxation}, $\bar{\mathbf{\Theta}}^{\star}$ is rank-\num{1} such that $\boldsymbol{\theta}^{\star}$ can be extracted without performance loss and the local optimality inherits to the original problem~\eqref{op:original}.
% 	\end{subsection}

% 	\begin{subsection}{Proof of Proposition~\ref{pr:mrt}}\label{ap:mrt}
% 		From the perspective of \gls{wit}, the \gls{mrt} precoder \eqref{eq:precoder_mrt} maximizes $\lvert{\mathbf{h}_{n}^\mathsf{H} \mathbf{w}_{\mathrm{I}, n}}\rvert = \lVert{\mathbf{h}_{n}}\rVert s_{\mathrm{I}, n}$ and thus the rate \eqref{eq:R}. From the perspective of \gls{wpt}, the \gls{mrt} precoder \eqref{eq:precoder_mrt} maximizes $(\mathbf{h}_{n}^\mathsf{H} \mathbf{w}_{\mathrm{I/P}, n})(\mathbf{h}_{n}^\mathsf{H} \mathbf{w}_{\mathrm{I/P}, n})^* = \lVert{\mathbf{h}_{n}}\rVert^2 s_{\mathrm{I/P}, n}^2$ and thus the second and fourth order \gls{dc} terms \eqref{eq:y_I2}--\eqref{eq:y_P4}. Therefore, the global optimal information and power precoders coincide at \gls{mrt}.
% 	\end{subsection}
% \end{section}

% \begin{section}{Proofs for Chapter \ref{ch:riscatter}}
% 	\begin{subsection}{Proof of Proposition \ref{pr:input_kkt_condition}}
% 		Denote the Lagrange multipliers associated with \eqref{co:sum_probability} and \eqref{co:nonnegative_probability} as $\{\nu_k\}_{k \in \mathcal{K}}$ and $\{\lambda_{m_k}\}_{k \in \mathcal{K},m_k \in \mathcal{M}}$, respectively.
% 		The Lagrangian function of problem \eqref{op:input_distribution} is
% 		\begin{equation}
% 			L(p, \nu, \lambda) = I(x_{\mathcal{K}}) + \sum_k \nu_k \Bigl( \sum_{m_k} p(x_{m_k}) - 1 \Bigr) + \sum_{k,m_k} \lambda_{m_k} p(x_{m_k}),
% 		\end{equation}
% 		and the \gls{kkt} conditions are, $\forall k,m_k$,
% 		\begin{subequations}
% 			\label{eq:input_kkt_condition_original}
% 			\begin{gather}
% 				I_k^\star(x_{m_k}) - (1 - \rho) + \nu_k^\star + \lambda_{m_k}^\star = 0, \label{eq:input_kkt_condition_1} \\
% 				\lambda_{m_k}^\star = 0, \quad    \text{if} \ p^\star(x_{m_k}) > 0, \label{eq:input_kkt_condition_2} \\
% 				\lambda_{m_k}^\star \ge 0, \quad  \text{if} \ p^\star(x_{m_k}) = 0 \label{eq:input_kkt_condition_3}.
% 			\end{gather}
% 		\end{subequations}
% 		Plugging \eqref{eq:input_kkt_condition_2} and \eqref{eq:input_kkt_condition_3} into \eqref{eq:input_kkt_condition_1} yields
% 		\begin{subequations}
% 			\label{eq:input_kkt_condition_transformed}
% 			\begin{alignat}{2}
% 				I_k^\star(x_{m_k}) & = 1 - \rho - \nu_k^\star, \quad   &  & \text{if} \ p^\star(x_{m_k}) > 0,\label{eq:probable_states_marginal} \\
% 				I_k^\star(x_{m_k}) & \le 1 - \rho - \nu_k^\star, \quad &  & \text{if} \ p^\star(x_{m_k}) = 0,\label{eq:dropped_states_marginal}
% 			\end{alignat}
% 		\end{subequations}
% 		such that
% 		\begin{equation}
% 			\sum_{m_k} p^\star(x_{m_k}) I_k^\star(x_{m_k}) = 1 - \rho - \nu_k^\star.
% 			\label{eq:input_kkt_condition_implied}
% 		\end{equation}
% 		On the other hand, by definition \eqref{eq:weighted_sum_marginal_information} we have
% 		\begin{equation}
% 			\sum_{m_k} p^\star(x_{m_k}) I_k^\star(x_{m_k}) = I^\star(x_{\mathcal{K}}),
% 			\label{eq:weighted_sum_marginal_information_implied}
% 		\end{equation}
% 		where the right-hand side is irrelevant to $k$.
% 		\eqref{eq:input_kkt_condition_transformed}, \eqref{eq:input_kkt_condition_implied}, and \eqref{eq:weighted_sum_marginal_information_implied} together complete the proof.
% 		\label{ap:input_kkt_condition}
% 	\end{subsection}

% 	\begin{subsection}{Proof of Proposition \ref{pr:input_kkt_solution}}
% 		We first prove sequence \eqref{eq:input_kkt_solution} is non-decreasing in weighted sum mutual information.
% 		Let $p(x_{m_{\mathcal{K}}}) = \prod_{q \in \mathcal{K}} p(x_{m_q})$ and $p'(x_{m_{\mathcal{K}}}) = p'(x_{m_k}) \prod_{q \in \mathcal{K} \setminus \{k\}} p(x_{m_q})$ be two distributions potentially different at $x_{m_k}$, and $J \bigl(p(x_{m_{\mathcal{K}}}),p'(x_{m_{\mathcal{K}}}) \bigr)$ be a joint function defined as
% 		\begin{equation}
% 			\begin{split}
% 				J \bigl( p(x_{m_{\mathcal{K}}}),p'(x_{m_{\mathcal{K}}}) \bigr)
% 				& \triangleq \sum_{m_{\mathcal{K}}} p(x_{m_{\mathcal{K}}})
% 				\Biggl( \rho \log \Bigl(1 + \frac{\lvert \mathbf{h}^\mathsf{H}(x_{m_{\mathcal{K}}}) \mathbf{w} \rvert^2}{\sigma_v^2}\Bigr)\\
% 				& \quad + (1 - \rho) \sum_{m_{\mathcal{K}}'} q(\hat{x}_{m_{\mathcal{K}}'}|x_{m_{\mathcal{K}}}) \log \frac{q(\hat{x}_{m_{\mathcal{K}}'}|x_{m_{\mathcal{K}}}) p'(x_{m_{\mathcal{K}}})}{p'(\hat{x}_{m_{\mathcal{K}}'}) p(x_{m_{\mathcal{K}}})} \Biggr).
% 			\end{split}
% 			\label{eq:intermediate_information_function}
% 		\end{equation}
% 		It is straightforward to verify $J \bigl( p(x_{m_{\mathcal{K}}}),p(x_{m_{\mathcal{K}}}) \bigr) = I(x_{\mathcal{K}})$ and $J \bigl( p(x_{m_{\mathcal{K}}}),p'(x_{m_{\mathcal{K}}}) \bigr)$ is a concave function for a given $p'(x_{m_{\mathcal{K}}})$.
% 		Setting $\nabla_{p(x_{m_k})} J \bigl( p(x_{m_{\mathcal{K}}}),p'(x_{m_{\mathcal{K}}}) \bigr) = 0$ yields
% 		\begin{equation}
% 			S_k'(x_{m_k}) - S_k'(x_{i_k}) + (1 - \rho) \log \frac{p(x_{i_k})}{p^\star(x_{m_k})} = 0,
% 			\label{eq:optimal_intermediate_information_condition}
% 		\end{equation}
% 		where $i_k \ne m_k$ is the reference state and
% 		\begin{equation}
% 			S_k'(x_{m_k}) \triangleq I_k'(x_{m_k}) + (1 - \rho) \sum_{m_{\mathcal{K} \setminus \{k\}}} p(x_{m_{\mathcal{K} \setminus \{k\}}}) \sum_{m_{\mathcal{K}}'} q(\hat{x}_{m_{\mathcal{K}}'}|x_{m_{\mathcal{K}}}) \log p'(x_{m_{\mathcal{K}}}).
% 		\end{equation}
% 		Evidently, $\forall m_k \ne i_k$, \eqref{eq:optimal_intermediate_information_condition} boils down to
% 		\begin{equation}
% 			p^\star(x_{m_k}) = \frac{p'(x_{m_k}) \exp \Bigl( \frac{\rho}{1 - \rho} I_k'(x_{m_k}) \Bigr)}{\sum_{m_k'} p'(x_{m_k'}) \exp \Bigl( \frac{\rho}{1 - \rho} I_k'(x_{m_k'}) \Bigr)}.
% 			\label{eq:optimal_relative_distribution}
% 		\end{equation}
% 		Since $p(x_{i_k}) = 1 - \sum_{m_k \ne i_k} p^\star(x_{m_k})$ has exactly the same form as \eqref{eq:optimal_relative_distribution}, the choice of reference does not matter and \eqref{eq:optimal_relative_distribution} is optimal $\forall m_k \in \mathcal{M}$.
% 		That is, for a fixed $p'(x_{m_{\mathcal{K}}})$, \eqref{eq:optimal_relative_distribution} ensures
% 		\begin{equation}
% 			J \bigl( p(x_{m_{\mathcal{K}}}),p'(x_{m_{\mathcal{K}}}) \bigr) \ge I'(x_{\mathcal{K}}).
% 			\label{eq:information_difference_lower}
% 		\end{equation}
% 		On the other hand, we notice
% 		\begin{subequations}
% 			\allowdisplaybreaks
% 			\label{eq:information_difference_upper}
% 			\begin{align}
% 				 & I(x_{\mathcal{K}}) - J \bigl( p(x_{m_{\mathcal{K}}}),p'(x_{m_{\mathcal{K}}}) \bigr)                                                                                                                    \nonumber           \\
% 				 & = (1 - \rho) \sum_{m_k} \frac{p'(x_{m_k}) f_k'(x_{m_k})}{\sum_{m_k'} p'(x_{m_k'}) f_k'(x_{m_k'})} \sum_{m_{\mathcal{K}}''} q(\hat{x}_{m_{\mathcal{K}}''}|x_{m_k}) \log \frac{\sum_{m_k'} p'(x_{m_k'}) q(\hat{x}_{m_{\mathcal{K}}''}|x_{m_k'}) f_k'(x_{m_k})}{\sum_{m_k'} p'(x_{m_k'}) q(\hat{x}_{m_{\mathcal{K}}''}|x_{m_k'}) f_k'(x_{m_k'})}                                   \\
% 				 & \ge (1 - \rho) \sum_{m_k} \frac{p'(x_{m_k}) f_k'(x_{m_k})}{\sum_{m_k'} p'(x_{m_k'}) f_k'(x_{m_k'})} \sum_{m_{\mathcal{K}}''} q(\hat{x}_{m_{\mathcal{K}}''}|x_{m_k}) \Biggl( 1 - \frac{\sum_{m_k'} p'(x_{m_k'}) q(\hat{x}_{m_{\mathcal{K}}''}|x_{m_k'}) f_k'(x_{m_k'})}{\sum_{m_k'} p'(x_{m_k'}) q(\hat{x}_{m_{\mathcal{K}}''}|x_{m_k'}) f_k'(x_{m_k})} \Biggr)                    \\
% 				 & = (1 - \rho) \Biggl( 1 - \sum_{m_k} \frac{p'(x_{m_k}) \cancel{f_k'(x_{m_k})}}{\sum_{m_k'} p'(x_{m_k'}) f_k'(x_{m_k'})} \sum_{m_{\mathcal{K}}''} q(\hat{x}_{m_{\mathcal{K}}''}|x_{m_k}) \frac{\sum_{m_k'} p'(x_{m_k'}) q(\hat{x}_{m_{\mathcal{K}}''}|x_{m_k'}) f_k'(x_{m_k'})}{\sum_{m_k'} p'(x_{m_k'}) q(\hat{x}_{m_{\mathcal{K}}''}|x_{m_k'}) \cancel{f_k'(x_{m_k})}} \Biggr)                       \\
% 				 & = (1 - \rho) \Biggl( 1 - \sum_{m_{\mathcal{K}}''} \frac{\cancel{\sum_{m_k} p'(x_{m_k}) q(\hat{x}_{m_{\mathcal{K}}''}|x_{m_k})}}{\sum_{m_k'} p'(x_{m_k'}) f_k'(x_{m_k'})} \frac{\sum_{m_k'} p'(x_{m_k'}) q(\hat{x}_{m_{\mathcal{K}}''}|x_{m_k'}) f_k'(x_{m_k'})}{\cancel{\sum_{m_k'} p'(x_{m_k'}) q(\hat{x}_{m_{\mathcal{K}}''}|x_{m_k'})}} \Biggr)                                     \\
% 				 & = (1 - \rho) \Biggl( 1 - \frac{\sum_{m_k'} p'(x_{m_k'}) f_k'(x_{m_k'}) \sum_{m_{\mathcal{K}}''} q(\hat{x}_{m_{\mathcal{K}}''}|x_{m_k'})}{\sum_{m_k'} p'(x_{m_k'}) f_k'(x_{m_k'})} \Biggr)                        \nonumber \\
% 				 & = 0,
% 			\end{align}
% 		\end{subequations}
% 		where $f_k'(x_{m_k}) \triangleq \exp \bigl( \frac{\rho}{1 - \rho} I_k'(x_{m_k}) \bigr)$ and the equality holds if and only if $p(x_{m_{\mathcal{K}}})$ and $p'(x_{m_{\mathcal{K}}})$ equals (i.e., \eqref{eq:optimal_relative_distribution} converges).
% 		\eqref{eq:information_difference_lower} and \eqref{eq:information_difference_upper} together imply $I(x_{\mathcal{K}}) \ge I'(x_{\mathcal{K}})$.
% 		Since mutual information is bounded above, we conclude the sequence \eqref{eq:input_kkt_solution} is non-decreasing and convergent.

% 		Next, we prove any converging point of sequence \eqref{eq:input_kkt_solution}, denoted as $p^\star(x_{m_k})$, fulfills the \gls{kkt} conditions \eqref{eq:input_kkt_condition}.
% 		Let
% 		\begin{equation}
% 			D^{(r)}(x_{m_k}) \triangleq \frac{p^{(r+1)}(x_{m_k})}{p^{(r)}(x_{m_k})} = \frac{f_k^{(r)}(x_{m_k})}{\sum_{m_k'} p^{(r)}(x_{m_k'}) f_k^{(r)}(x_{m_k'})}.
% 		\end{equation}
% 		As sequence \eqref{eq:input_kkt_solution} is convergent, any state with $p^\star(x_{m_k}) > 0$ need to satisfy $D^\star(x_{m_k}) \triangleq \lim_{r \to \infty} D^{(r)}(x_{m_k}) = 1$, namely
% 		\begin{equation}
% 			I_k^\star(x_{m_k}) = \frac{1 - \rho}{\rho} \log \sum_{m_k'} p^\star(x_{m_k'}) \exp \Bigl( \frac{\rho}{1 - \rho} I_k^\star(x_{m_k'}) \Bigr).
% 		\end{equation}
% 		The right-hand side is a constant for node $k$ and implies \eqref{eq:probable_states_marginal}.
% 		That is, any converging point with nonzero probability must satisfy \eqref{eq:probable_states}.
% 		On the other hand, we assume $p^\star(x_{m_k}) = 0$ does not satisfy \eqref{eq:dropped_states}, namely
% 		\begin{equation}
% 			I_k^\star(x_{m_k}) > I^\star(x_{\mathcal{K}}) = \sum_{m_k'} p^\star(x_{m_k'}) I_k^\star(x_{m_k'}),
% 			\label{eq:dropped_states_assumption}
% 		\end{equation}
% 		Since the exponential function is monotonically increasing, \eqref{eq:dropped_states_assumption} implies $f_k^\star(x_{m_k}) > \sum_{m_k'} p^\star(x_{m_k'}) f_k^\star(x_{m_k'})$ and $D^\star(x_{m_k}) > 1$.
% 		It contradicts with
% 		\begin{equation}
% 			p^{(r)}(x_{m_k}) = p^{(0)}(x_{m_k}) \prod_{n=1}^r D^{(n)}(x_{m_k}),
% 		\end{equation}
% 		since the left-hand side is zero while all terms on the right-hand side are strictly positive.
% 		The proof is completed.
% 		\label{ap:input_kkt_solution}
% 	\end{subsection}
% \end{section}

\begin{section}{Proofs for Chapter \ref{ch:channel_shaping}}
	\begin{subsection}{Proof of Lemma \ref{lm:pareto_gradient}}\label{ap:pareto_gradient}
		Let $\mathbf{H} = \sum_n \mathbf{u}_n \sigma_n \mathbf{v}_n^\mathsf{H}$ be the compact \gls{svd} of the equivalent channel.
		Since the singular vectors are orthonormal, the $n$-th singular value can be expressed as
		\begin{equation}
			\sigma_n = \mathbf{u}_n^\mathsf{H} \mathbf{H} \mathbf{v}_n = \mathbf{u}_n^\mathsf{T} \mathbf{H}^* \mathbf{v}_n^*,
		\end{equation}
		whose differential w.r.t. $\mathbf{\Theta}_g^*$ is
		\begin{align*}
			\partial \sigma_n
			 & = \partial \mathbf{u}_n^\mathsf{T} \underbrace{\mathbf{H}^* \mathbf{v}_n^*}_{\sum_m \mathbf{u}_m^* \sigma_m \mathbf{v}_m^\mathsf{T} \mathbf{v}_n} + \mathbf{u}_n^\mathsf{T} \cdot \partial \mathbf{H}^* \cdot \mathbf{v}_n^* + \underbrace{\mathbf{u}_n^\mathsf{T} \mathbf{H}^*}_{\mathbf{u}_n^\mathsf{T} \sum_m \mathbf{u}_m^* \sigma_m \mathbf{v}_m^\mathsf{T}} \partial \mathbf{v}_n^* \\
			 & = \underbrace{\partial \mathbf{u}_n^\mathsf{T} \mathbf{u}_n^*}_{\partial 1 = 0} \cdot \sigma_n + \mathbf{u}_n^\mathsf{T} \cdot \partial \mathbf{H}^* \cdot \mathbf{v}_n^* + \sigma_n \cdot \underbrace{\mathbf{v}_n^\mathsf{T} \partial \mathbf{v}_n^*}_{\partial 1 = 0}                                                                                                                  \\
			 & = \mathbf{u}_n^\mathsf{T} \mathbf{H}_{\mathrm{B},g}^* \cdot \partial \mathbf{\Theta}_g^* \cdot \mathbf{H}_{\mathrm{F},g}^* \mathbf{v}_n^*                                                                                                                                                                                                                                                 \\
			 & = \mathrm{tr}(\mathbf{H}_{\mathrm{F},g}^* \mathbf{v}_n^*\mathbf{u}_n^\mathsf{T} \mathbf{H}_{\mathrm{B},g}^* \cdot \partial \mathbf{\Theta}_g^*).
		\end{align*}
		According to \cite{Hjorungnes2007}, the corresponding complex derivative is
		\begin{equation}
			\frac{\partial \sigma_n}{\partial \mathbf{\Theta}_g^*} = \mathbf{H}_{\mathrm{B},g}^\mathsf{H} \mathbf{u}_n \mathbf{v}_n^\mathsf{H} \mathbf{H}_{\mathrm{F},g}^\mathsf{H}.
			\label{eq:derivative_sv}
		\end{equation}
		A linear combination of \eqref{eq:derivative_sv} yields \eqref{eq:pareto_gradient}.
	\end{subsection}

	\begin{subsection}{Proof of Proposition \ref{pp:dof}}\label{ap:dof}
		The scattering matrix of \gls{bd}-\gls{ris} can be decomposed as\footnote{This is because (block) unitary matrices are closed under multiplication.}
		\begin{equation}
			\mathbf{\Theta} = \mathbf{L} \mathbf{\Theta}_\mathrm{D} \mathbf{R}^\mathsf{H},
		\end{equation}
		where $\mathbf{\Theta}_\mathrm{D} \in \mathbb{U}^{N_\mathrm{S} \times N_\mathrm{S}}$ corresponds to diagonal \gls{ris} and $\mathbf{L}, \mathbf{R} \in \mathbb{U}^{N_\mathrm{S} \times N_\mathrm{S}}$ are block diagonal matrices of $L \times L$ unitary blocks.
		Manipulating $\mathbf{L}$ and $\mathbf{R}$ rotates the linear spans of $\bar{\mathbf{H}}_\mathrm{B} \triangleq \mathbf{H}_\mathrm{B} \mathbf{L}$ and $\bar{\mathbf{H}}_\mathrm{F} \triangleq \mathbf{R}^\mathsf{H} \mathbf{H}_\mathrm{F}$ and maintains their rank.
		On the other hand, there exists a $\mathbf{\Theta}_\mathrm{D}$ such that
		\begin{equation*}
			\begin{split}
				\mathrm{rank}(\mathbf{H}_\mathrm{B} \mathbf{\Theta}_\mathrm{D} \mathbf{H}_\mathrm{F})
				& = \min \bigl( \mathrm{rank}(\mathbf{H}_\mathrm{B}), \mathrm{rank}(\mathbf{\Theta}_\mathrm{D}), \mathrm{rank}(\mathbf{H}_\mathrm{F}) \bigr) \\
				& = \min \bigl( \mathrm{rank}(\bar{\mathbf{H}}_\mathrm{B}), N_\mathrm{S}, \mathrm{rank}(\bar{\mathbf{H}}_\mathrm{F}) \bigr) \\
				& = \max_\mathbf{\Theta} \ \mathrm{rank}(\mathbf{H}_\mathrm{B} \mathbf{\Theta} \mathbf{H}_\mathrm{F})
			\end{split}
		\end{equation*}
		The same result holds if the direct link is present.
	\end{subsection}

	\begin{subsection}{Proof of Proposition \ref{pp:rank_deficient}}\label{ap:rank_deficient}
		We consider rank-$k$ forward channel and the proof follows similarly for rank-$k$ backward channel.
		Let $\mathbf{H}_\mathrm{F} = \mathbf{U}_\mathrm{F} \mathbf{\Sigma}_\mathrm{F} \mathbf{V}_\mathrm{F}^\mathsf{H}$ be the compact \gls{svd} of the forward channel.
		The channel Gram matrix $\mathbf{G} \triangleq \mathbf{H} \mathbf{H}^\mathsf{H} $ can be written as
		\begin{equation*}
			\begin{split}
				\mathbf{G}
				& = \mathbf{H}_\mathrm{D} \mathbf{H}_\mathrm{D}^\mathsf{H} + \mathbf{H}_\mathrm{B} \mathbf{\Theta} \mathbf{U}_\mathrm{F} \mathbf{\Sigma}_\mathrm{F} \mathbf{\Sigma}_\mathrm{F}^\mathsf{H} \mathbf{U}_\mathrm{F}^\mathsf{H} \mathbf{\Theta}^\mathsf{H} \mathbf{H}_\mathrm{B}^\mathsf{H} \\
				& \quad + \mathbf{H}_\mathrm{B} \mathbf{\Theta} \mathbf{U}_\mathrm{F} \mathbf{\Sigma}_\mathrm{F} \mathbf{V}_\mathrm{F}^\mathsf{H} \mathbf{H}_\mathrm{D}^\mathsf{H} + \mathbf{H}_\mathrm{D} \mathbf{V}_\mathrm{F} \mathbf{\Sigma}_\mathrm{F} \mathbf{U}_\mathrm{F}^\mathsf{H} \mathbf{\Theta}^\mathsf{H} \mathbf{H}_\mathrm{B}^\mathsf{H} \\
				& = \mathbf{H}_\mathrm{D} (\mathbf{I} - \mathbf{V}_\mathrm{F} \mathbf{V}_\mathrm{F}^\mathsf{H}) \mathbf{H}_\mathrm{D}^\mathsf{H} \\
				& \quad + (\mathbf{H}_\mathrm{B} \mathbf{\Theta} \mathbf{U}_\mathrm{F} \mathbf{\Sigma}_\mathrm{F} + \mathbf{H}_\mathrm{D} \mathbf{V}_\mathrm{F}) (\mathbf{\Sigma}_\mathrm{F} \mathbf{U}_\mathrm{F}^\mathsf{H} \mathbf{\Theta}^\mathsf{H} \mathbf{H}_\mathrm{B}^\mathsf{H} + \mathbf{V}_\mathrm{F}^\mathsf{H} \mathbf{H}_\mathrm{D}^\mathsf{H}) \\
				& = \mathbf{Y} + \mathbf{Z} \mathbf{Z}^\mathsf{H},
			\end{split}
		\end{equation*}
		where we define $\mathbf{Y} \triangleq \mathbf{H}_\mathrm{D} (\mathbf{I} - \mathbf{V}_\mathrm{F} \mathbf{V}_\mathrm{F}^\mathsf{H}) \mathbf{H}_\mathrm{D}^\mathsf{H} \in \mathbb{H}^{N_\mathrm{R} \times N_\mathrm{R}}$ and $\mathbf{Z} \triangleq \mathbf{H}_\mathrm{B} \mathbf{\Theta} \mathbf{U}_\mathrm{F} \mathbf{\Sigma}_\mathrm{F} + \mathbf{H}_\mathrm{D} \mathbf{V}_\mathrm{F} \in \mathbb{C}^{N_\mathrm{R} \times k}$.
		That is to say, $\mathbf{G}$ can be expressed as a Hermitian matrix plus $k$ rank-1 perturbations.
		According to the Cauchy interlacing formula \cite{Golub2013}, the $n$-th eigenvalue of $\mathbf{G}$ is bounded by
		\begin{align}
			\lambda_n(\mathbf{G}) & \le \lambda_{n-k}(\mathbf{Y}), &  & \text{if } n > k, \label{iq:ev:bound_enlarge}          \\
			\lambda_n(\mathbf{G}) & \ge \lambda_n(\mathbf{Y}),     &  & \text{if } n < N - k + 1 \label{iq:ev_bound_suppress}.
		\end{align}
		Since $\mathbf{Y} = \mathbf{T} \mathbf{T}^\mathsf{H}$ is positive semi-definite, taking the square roots of \eqref{iq:ev:bound_enlarge} and \eqref{iq:ev_bound_suppress} gives \eqref{iq:sv_bound_enlarge} and \eqref{iq:sv_bound_suppress}.
	\end{subsection}

	\begin{subsection}{Proof of Proposition \ref{pp:fully_connected}}\label{ap:fully_connected}
		Let $\mathbf{H}_\mathrm{B} = \mathbf{U}_\mathrm{B} \mathbf{\Sigma}_\mathrm{B} \mathbf{V}_\mathrm{B}^\mathsf{H}$ and $\mathbf{H}_\mathrm{F} = \mathbf{U}_\mathrm{F} \mathbf{\Sigma}_\mathrm{F} \mathbf{V}_\mathrm{F}^\mathsf{H}$ be the \gls{svd} of the backward and forward channels, respectively.
		The scattering matrix of fully-connected \gls{ris} can be decomposed as
		\begin{equation}
			\mathbf{\Theta} = \mathbf{V}_\mathrm{B} \mathbf{X} \mathbf{U}_\mathrm{F}^\mathsf{H},
			\label{eq:scattering_fc}
		\end{equation}
		where $\mathbf{X} \in \mathbb{U}^{N_\mathrm{S} \times N_\mathrm{S}}$ is a unitary matrix to be designed.
		The equivalent channel is thus a function of $\mathbf{X}$
		\begin{equation}
			\mathbf{H} = \mathbf{H}_\mathrm{B} \mathbf{\Theta} \mathbf{H}_\mathrm{F} = \mathbf{U}_\mathrm{B} \mathbf{\Sigma}_\mathrm{B} \mathbf{X} \mathbf{\Sigma}_\mathrm{F} \mathbf{V}_\mathrm{F}^\mathsf{H}.
			\label{eq:channel_equivalent_fc}
		\end{equation}
		Since $\mathrm{sv}(\mathbf{U} \mathbf{A} \mathbf{V}^\mathsf{H}) = \mathrm{sv}(\mathbf{A})$ for unitary $\mathbf{U}$ and $\mathbf{V}$, we have
		\begin{align*}
			\mathrm{sv}(\mathbf{H}) & = \mathrm{sv}(\mathbf{U}_\mathrm{B} \mathbf{\Sigma}_\mathrm{B} \mathbf{X} \mathbf{\Sigma}_\mathrm{F} \mathbf{V}_\mathrm{F}^\mathsf{H})                                                                     \\
			                        & = \mathrm{sv}(\mathbf{\Sigma}_\mathrm{B} \mathbf{X} \mathbf{\Sigma}_\mathrm{F})                                                                                                                            \\
			                        & = \mathrm{sv}(\bar{\mathbf{U}}_\mathrm{B} \mathbf{\Sigma}_\mathrm{B} \mathbf{\bar{V}}_\mathrm{B}^\mathsf{H} \bar{\mathbf{U}}_\mathrm{F} \mathbf{\Sigma}_\mathrm{F} \mathbf{\bar{V}}_\mathrm{F}^\mathsf{H}) \\
			                        & = \mathrm{sv}(\mathbf{BF}),
		\end{align*}
		where $\bar{\mathbf{U}}_{\mathrm{B}/\mathrm{F}}$ and $\bar{\mathbf{V}}_{\mathrm{B}/\mathrm{F}}$ are arbitrary unitary matrices.
	\end{subsection}

	\begin{subsection}{Proof of Lemma \ref{lm:rate_gradient}}\label{ap:rate_gradient}
		The differential of $R$ w.r.t. $\mathbf{\Theta}_g^*$ is \cite{Hjorungnes2007}
		\begin{align*}
			\partial R
			 & = \frac{1}{\eta} \mathrm{tr} \biggl\{ \partial \mathbf{H}^* \cdot \mathbf{Q}^\mathsf{T} \mathbf{H}^\mathsf{T} \Bigl(\mathbf{I} + \frac{\mathbf{H}^* \mathbf{Q}^\mathsf{T} \mathbf{H}^\mathsf{T}}{\eta}\Bigr)^{-1} \biggr\}                                                                      \\
			 & = \frac{1}{\eta} \mathrm{tr} \biggl\{ \mathbf{H}_{\mathrm{B},g}^* \cdot \partial \mathbf{\Theta}_g^* \cdot \mathbf{H}_{\mathrm{F},g}^* \mathbf{Q}^\mathsf{T} \mathbf{H}^\mathsf{T} \Bigl(\mathbf{I} + \frac{\mathbf{H}^* \mathbf{Q}^\mathsf{T} \mathbf{H}^\mathsf{T}}{\eta}\Bigr)^{-1} \biggr\} \\
			 & = \frac{1}{\eta} \mathrm{tr} \biggl\{ \mathbf{H}_{\mathrm{F},g}^* \mathbf{Q}^\mathsf{T} \mathbf{H}^\mathsf{T} \Bigl(\mathbf{I} + \frac{\mathbf{H}^* \mathbf{Q}^\mathsf{T} \mathbf{H}^\mathsf{T}}{\eta}\Bigr)^{-1} \mathbf{H}_{\mathrm{B},g}^* \cdot \partial \mathbf{\Theta}_g^* \biggr\},
		\end{align*}
		and the corresponding complex derivative is \eqref{eq:rate_gradient}.
	\end{subsection}

	\begin{subsection}{Proof of Proposition \ref{pp:power}}\label{ap:power}
		The differential of \eqref{ob:power_passive} w.r.t. $\mathbf{\Theta}_g^*$ is
		\begin{align*}
			\partial \lVert \mathbf{H} \rVert _\mathrm{F}^2
			 & = \mathrm{tr}\bigl(\mathbf{H}_{\mathrm{B},g}^* \cdot \partial \mathbf{\Theta}_g^* \cdot \mathbf{H}_{\mathrm{F},g}^* (\mathbf{H}_\mathrm{D}^\mathsf{T} + \mathbf{H}_\mathrm{F}^\mathsf{T} \mathbf{\Theta}^\mathsf{T} \mathbf{H}_\mathrm{B}^\mathsf{T})\bigr) \\
			 & = \mathrm{tr}\bigl(\mathbf{H}_{\mathrm{F},g}^* (\mathbf{H}_\mathrm{D}^\mathsf{T} + \mathbf{H}_\mathrm{F}^\mathsf{T} \mathbf{\Theta}^\mathsf{T} \mathbf{H}_\mathrm{B}^\mathsf{T}) \mathbf{H}_{\mathrm{B},g}^* \cdot \partial \mathbf{\Theta}_g^*\bigr)
		\end{align*}
		and the corresponding complex derivative is
		\begin{equation}
			\frac{\partial \lVert \mathbf{H} \rVert _\mathrm{F}^2}{\partial \mathbf{\Theta}_g^*} = \mathbf{H}_{\mathrm{B},g}^\mathsf{H} (\mathbf{H}_\mathrm{D} + \mathbf{H}_\mathrm{B} \mathbf{\Theta} \mathbf{H}_\mathrm{F}) \mathbf{H}_{\mathrm{F},g}^\mathsf{H} = \mathbf{M}_g.
		\end{equation}
		% For a given $\mathbf{\Theta}$, we approximate the quadratic objective \eqref{ob:power_passive} by local Taylor expansion and recast problem \eqref{op:power_passive} as

		First, we approximate the quadratic objective \eqref{ob:power_passive} by its local Taylor expansion
		\begin{maxi!}
			{\scriptstyle{\mathbf{\Theta}}}{\sum_g 2 \Re\bigl\{ \mathrm{tr}(\mathbf{\Theta}_g^\mathsf{H} \mathbf{M}_g) \bigr\}}{\label{op:power_passive_approx}}{\label{ob:power_passive_approx}}
			\addConstraint{\mathbf{\Theta}_g^\mathsf{H} \mathbf{\Theta}_g=\mathbf{I}, \quad \forall g.}{}{}
		\end{maxi!}
		Let $\mathbf{M}_g = \mathbf{U}_g \mathbf{\Sigma}_g \mathbf{V}_g^\mathsf{H}$ be the compact \gls{svd} of $\mathbf{M}_g$.
		We have
		\begin{equation}
			\Re \bigl\{\mathrm{tr}(\mathbf{\Theta}_g^\mathsf{H} \mathbf{M}_g)\bigr\} = \Re \bigl\{ \mathrm{tr}(\mathbf{\Sigma}_g \mathbf{V}_g^\mathsf{H} \mathbf{\Theta}_g^\mathsf{H} \mathbf{U}_g) \bigr\} \le \mathrm{tr}(\mathbf{\Sigma}_g).
		\end{equation}
		The upper bound is tight when $\mathbf{V}_g^\mathsf{H} \mathbf{\Theta}_g^\mathsf{H} \mathbf{U}_g = \mathbf{I}$, which implies the optimal solution of \eqref{op:power_passive_approx} is $\tilde{\mathbf{\Theta}}_g = \mathbf{U}_g \mathbf{V}_g^\mathsf{H}$, $\forall g$.

		Next, we prove that solving \eqref{op:power_passive_approx} successively does not decrease \eqref{ob:power_passive}.
		Since $\tilde{\mathbf{\Theta}}$ optimal for problem \eqref{op:power_passive_approx}, we have $\sum_g 2 \Re\bigl\{ \mathrm{tr}(\tilde{\mathbf{\Theta}}_g^\mathsf{H} \mathbf{M}_g) \bigr\} \ge \sum_g 2 \Re\bigl\{ \mathrm{tr}(\mathbf{\Theta}_g^\mathsf{H} \mathbf{M}_g) \bigr\}$ which is explicitly expressed by \eqref{iq:power_passive_approx}.
		% Expanding both side gives \eqref{iq:power_passive_approx}.
		On the other hand, expanding $\lVert \sum_g \mathbf{H}_{\mathrm{B},g} \tilde{\mathbf{\Theta}}_g \mathbf{H}_{\mathrm{F},g} - \sum_g \mathbf{H}_{\mathrm{B},g} \mathbf{\Theta}_g \mathbf{H}_{\mathrm{F},g} \rVert _\mathrm{F}^2 \ge 0$ gives \eqref{iq:nonnegative_norm_square}.
		Adding \eqref{iq:power_passive_approx} and \eqref{iq:nonnegative_norm_square}, we have
		\begin{multline}
			2 \Re \Bigl\{\mathrm{tr}(\tilde{\mathbf{\Theta}}^\mathsf{H} \mathbf{H}_\mathrm{B}^\mathsf{H} \mathbf{H}_\mathrm{D} \mathbf{H}_\mathrm{F}^\mathsf{H}) \Bigr\} + \mathrm{tr}(\mathbf{H}_\mathrm{F}^\mathsf{H} \tilde{\mathbf{\Theta}}^\mathsf{H} \mathbf{H}_\mathrm{B}^\mathsf{H} \mathbf{H}_\mathrm{B} \tilde{\mathbf{\Theta}} \mathbf{H}_\mathrm{F}) \\
			\ge 2 \Re \Bigl\{\mathrm{tr}({\mathbf{\Theta}}^\mathsf{H} \mathbf{H}_\mathrm{B}^\mathsf{H} \mathbf{H}_\mathrm{D} \mathbf{H}_\mathrm{F}^\mathsf{H}) \Bigr\} + \mathrm{tr}(\mathbf{H}_\mathrm{F}^\mathsf{H} {\mathbf{\Theta}}^\mathsf{H} \mathbf{H}_\mathrm{B}^\mathsf{H} \mathbf{H}_\mathrm{B} {\mathbf{\Theta}} \mathbf{H}_\mathrm{F}),
		\end{multline}
		which suggests that updating $\tilde{\mathbf{\Theta}}$ does not decrease \eqref{ob:power_passive}.
		\begin{figure*}
			\begin{equation}
				\label{iq:power_passive_approx}
				\resizebox{0.95\textwidth}{!}{
					$2 \Re \Bigl\{\sum\limits_g \mathrm{tr}(\tilde{\mathbf{\Theta}}_g^\mathsf{H} \mathbf{H}_{\mathrm{B},g}^\mathsf{H} \mathbf{H}_\mathrm{D} \mathbf{H}_{\mathrm{F},g}^\mathsf{H}) + \sum\limits_{g_1,g_2} \mathrm{tr}(\tilde{\mathbf{\Theta}}_{g_1}^\mathsf{H} \mathbf{H}_{\mathrm{B},g_1}^\mathsf{H} \mathbf{H}_{\mathrm{B},g_2} \mathbf{\Theta}_{g_2} \mathbf{H}_{\mathrm{F},g_2} \mathbf{H}_{\mathrm{F},g_1}^\mathsf{H})\Bigr\} \ge 2 \Re \Bigl\{\sum\limits_g \mathrm{tr}({\mathbf{\Theta}}_g^\mathsf{H} \mathbf{H}_{\mathrm{B},g}^\mathsf{H} \mathbf{H}_\mathrm{D} \mathbf{H}_{\mathrm{F},g}^\mathsf{H}) + \sum\limits_{g_1,g_2} \mathrm{tr}({\mathbf{\Theta}}_{g_1}^\mathsf{H} \mathbf{H}_{\mathrm{B},g_1}^\mathsf{H} \mathbf{H}_{\mathrm{B},g_2} \mathbf{\Theta}_{g_2} \mathbf{H}_{\mathrm{F},g_2} \mathbf{H}_{\mathrm{F},g_1}^\mathsf{H})\Bigr\}$
				}
			\end{equation}
			\begin{equation}
				\label{iq:nonnegative_norm_square}
				\resizebox{0.95\textwidth}{!}{
					$\sum\limits_{g_1,g_2} \mathrm{tr}(\mathbf{H}_{\mathrm{F},g_1}^\mathsf{H} \tilde{\mathbf{\Theta}}_{g_1}^\mathsf{H} \mathbf{H}_{\mathrm{B},g_1}^\mathsf{H} \mathbf{H}_{\mathrm{B},g_2} \tilde{\mathbf{\Theta}}_{g_2} \mathbf{H}_{\mathrm{F},g_2}) - 2 \Re \Bigl\{\sum\limits_{g_1,g_2} \mathrm{tr}(\mathbf{H}_{\mathrm{F},g_1}^\mathsf{H} \tilde{\mathbf{\Theta}}_{g_1}^\mathsf{H} \mathbf{H}_{\mathrm{B},g_1}^\mathsf{H} \mathbf{H}_{\mathrm{B},g_2} \mathbf{\Theta}_{g_2} \mathbf{H}_{\mathrm{F},g_2})\Bigr\} + \sum\limits_{g_1,g_2} \mathrm{tr}(\mathbf{H}_{\mathrm{F},g_1}^\mathsf{H} {\mathbf{\Theta}}_{g_1}^\mathsf{H} \mathbf{H}_{\mathrm{B},g_1}^\mathsf{H} \mathbf{H}_{\mathrm{B},g_2} {\mathbf{\Theta}}_{g_2} \mathbf{H}_{\mathrm{F},g_2}) \ge 0$
				}
			\end{equation}
			\hrulefill
		\end{figure*}

		Finally, we prove that the converging point of \eqref{op:power_passive_approx}, denoted by $\tilde{\mathbf{\Theta}}^?$, is a stationary point of \eqref{op:power_passive}.
		The \gls{kkt} conditions of \eqref{op:power_passive} and \eqref{op:power_passive_approx} are equivalent in terms of primal/dual feasibility and complementary slackness, while the stationary conditions are respectively, $\forall g$,
		\begin{gather}
			\mathbf{H}_{\mathrm{B},g}^\mathsf{H} (\mathbf{H}_\mathrm{D} + \mathbf{H}_\mathrm{B} \mathbf{\Theta}^\star \mathbf{H}_\mathrm{F}) \mathbf{H}_{\mathrm{F},g}^\mathsf{H} - \mathbf{\Theta}_g^\star \mathbf{\Lambda}_g^\mathsf{H} = 0,\label{eq:power_passive_stationary}\\
			\mathbf{M}_g - \mathbf{\Theta}_g^\star \mathbf{\Lambda}_g^\mathsf{H} = 0.\label{eq:power_passive_approx_stationary}
		\end{gather}
		On convergence, \eqref{eq:power_passive_approx_stationary} becomes $\mathbf{H}_{\mathrm{B},g}^\mathsf{H} (\mathbf{H}_\mathrm{D} + \mathbf{H}_\mathrm{B} \mathbf{\Theta}^? \mathbf{H}_\mathrm{F}) \mathbf{H}_{\mathrm{F},g}^\mathsf{H} - \mathbf{\Theta}_g^? \mathbf{\Lambda}_g^\mathsf{H} = 0$ and reduces to \eqref{eq:power_passive_stationary}.
		The proof is thus completed.
	\end{subsection}
\end{section}
