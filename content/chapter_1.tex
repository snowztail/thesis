%!TEX root = ../thesis.tex

\graphicspath{{assets/chapter_1/}}

\chapter{Introduction}
% \chapter{Summary of research problems and contribution}

\begin{section}{Motivation and Aims}
	The quest for reliable, high-speed, and ubiquitous wireless connectivity has been long-standing since Marconi's illuminating radio in 1895.
	Great successes have been made at the transmitter and receiver sides over the past century, and the communications society is unprecedentedly close to the Shannon limit \cite{Shannon1948}.
	Today we are witnessing a paradigm shift from \emph{connectivity} to \emph{intelligence}, where the wireless environment is no longer a chaotic medium but a conscious agent that can serve on demand.
	This is made possible by the latest advances in machine learning and programmable metamaterials --- the former enables the network to learn from the environment while the latter allows the environment to adapt to the network.
	Together, they form a symbiotic relationship that can revolutionize the way we communicate, sense, and interact with the world.
	One promising candidate to fulfill this vision is \gls{ris}, a programmable metasurface that recycles and redistributes the electromagnetic waves in the air for improved wireless performance.
	This disruptive technology can be integrated into the next-generation wireless systems for signal enhancement, interference suppression, scattering enrichment, blockage bypassing, coverage extension, and security control.
	Compared with traditional multi-antenna techniques, \gls{ris} has the potential to achieve a similar performance using fewer active components, lower power consumption, and lower hardware complexity.
	\gls{ris} is also different from conventional relays as it features no \gls{rf} chains, no symbol dependency, minimum signal processing, and negligible additional noise.
	Moreover, it can be easily deployed in various forms (e.g., walls, windows, ceilings, tables) to provide seamless coverage and powerful customization for indoor and outdoor environments.
	These unique characteristics make \gls{ris} a ubiquitous and cost-effective solution for future communication, sensing, and power transfer systems.

	Despite the great possibilities, the prototyping of \gls{ris} is still in its infancy and no commercial product has been released by the first quarter of 2024.
	The transition from theory to practice is hindered by many challenges, such as channel acquisition, response resolution, out-of-band response mitigation, placement optimization, and integration with existing systems.
	Most importantly, a comprehensive understanding of its potential is still missing and the precise role it shall play in 6G remains ambiguous.
	For example, \gls{ris} could be incorporated into the transmitter and receiver for \emph{beamforming}, employed as a free-rider information source for \emph{modulation}, or placed in space as a standalone device for \emph{channel shaping}.
	These applications have distinctive requirements and trade-offs, but they are not mutually exclusive and may be evolved into a versatile tool, which blurs the boundary between the network and environment.
	Imagine a future where everything can be ``smartened'' by coating with a metamaterial layer and attaching a microcontroller tag.
	Only a few electromagnetic sources are needed, while most objects can exploit the surrounding waves to energize themselves, sense the environment, communicate with others, and help those in need when idle.
	This vision motivates us to explore the fundamental limits of \gls{ris} and integrate it with state-of-the-art wireless technologies.
	In particular, Chapters \ref{ch:ris_aided_swipt}--\ref{ch:channel_shaping} respectively aim to:
	\begin{itemize}
		\item Investigate the impact of \gls{ris} on wireless information and power transfer and propose effective configurations for enhanced performance trade-off;
		\item Compare the principles of \gls{ris} with various backscatter applications and develop a novel protocol for beamforming-modulation integration;
		\item Rethink the architecture and modeling of \gls{ris} and characterize its ultimate channel shaping capabilities for further design reference.
	\end{itemize}
\end{section}

\begin{section}{Summary of Contributions}
	The key contributions of Chapter \ref{ch:ris_aided_swipt} are summarized as follows:
	\begin{itemize}
		\item Introduce \gls{ris} to a multi-antenna multi-carrier \gls{swipt} system with various receiver architectures;
		\item Consider joint waveform and beamforming design for the proposed system under a practical energy harvester model;
		\item Characterize the \gls{r-e} performance trade-off and provide insights into the transceiver design;
		\item Formulate the optimization problem as harvested energy maximization subject to minimum communication rate constraints;
		\item Propose local-optimal and low-complexity algorithms and evaluate their performance through numerical simulations;
		\item Discuss the array gain for communication and the scaling order for power transfer in terms of the number of transmit antennas and \gls{ris} elements.
	\end{itemize}

	The key contributions of Chapter \ref{ch:ris_modulation} are summarized as follows:
\end{section}

\begin{section}{Outline}

\end{section}

\begin{section}{Publications}
\end{section}
