%!TEX root = ../thesis.tex

\graphicspath{{assets/chapter_1/}}

\chapter{Introduction}
% \chapter{Summary of research problems and contribution}

\begin{section}{Motivation}
	The quest for reliable, high-speed, and ubiquitous wireless connectivity has been long-standing since Marconi's illuminating radio in 1895.
	Great successes have been made at the transmitter and receiver sides over the past century, and the communications society is unprecedentedly close to the Shannon limit \cite{Shannon1948}.
	Today we are witnessing a paradigm shift from \emph{connectivity} to \emph{intelligence}, where the wireless environment is no longer a chaotic medium but a conscious agent that can serve on demand.
	This is made possible by the latest advances in machine learning and programmable metamaterials --- the former enables the network to learn from the environment while the latter allows the environment to adapt to the network.
	Together, they form a symbiotic relationship that can revolutionize the way we communicate, sense, and interact with the world.
	One promising candidate to fulfill this vision is \gls{ris}, a programmable metasurface that recycles and redistributes the ambient electromagnetic waves for improved wireless performance.
	It is a disruptive technology that can be integrated into the next-generation wireless systems for signal enhancement, interference suppression, scattering enrichment, blockage bypassing, coverage extension, and security management.
	Apart from the communications domain, \gls{ris} has also found applications in radar, imaging, and sensing, where it can be used to manipulate the electromagnetic waves for better resolution, penetration, and detection.

\end{section}

\begin{section}{Objectives and Challenges}

\end{section}

\begin{section}{Outline and Contributions}

\end{section}

\begin{section}{Publications}

\end{section}
