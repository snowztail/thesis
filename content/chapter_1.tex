%!TEX root = ../thesis.tex

\graphicspath{{assets/chapter_1/}}

\chapter{Introduction}
% \chapter{Summary of research problems and contribution}

\begin{section}{Motivation}
	The quest for reliable, high-speed, and ubiquitous wireless connectivity has been long-standing since Marconi's illuminating radio in 1895.
	Great successes have been made at the transmitter and receiver sides over the past century, and the communications society is unprecedentedly close to the Shannon limit \cite{Shannon1948}.
	Today we are witnessing a paradigm shift from \emph{connectivity} to \emph{intelligence}, where the wireless environment is no longer a chaotic medium but a conscious agent that can serve on demand.
	This is made possible by the latest advances in machine learning and programmable metamaterials --- the former enables the network to learn from the environment while the latter allows the environment to adapt to the network.
	Together, they form a symbiotic relationship that can revolutionize the way we communicate, sense, and interact with the world.
	One promising candidate to fulfill this vision is \gls{ris}, a programmable metasurface that recycles and redistributes the ambient electromagnetic waves for improved wireless performance.
	This disruptive technology can be integrated into the next-generation wireless systems for signal enhancement, interference suppression, scattering enrichment, blockage bypassing, coverage extension, and security improvement.
	Compared with traditional multi-antenna techniques, \gls{ris} has the potential to achieve a similar performance using fewer active components, lower power consumption, and lower hardware complexity.
	\gls{ris} is also different from conventional relays as it features no \gls{rf} chains, no symbol dependency, minimum signal processing, and negligible additional noise.
	Moreover, it can be easily deployed in various forms, such as walls, windows, ceilings, and furniture, to provide seamless coverage and powerful customization for indoor and outdoor environments.
	These unique characteristics make \gls{ris} a ubiquitous and cost-effective solution for future wireless communication, sensing, and power transfer systems.

	Despite the great possibilities, the prototyping of \gls{ris} is still in its infancy and no commercial product has been released by the first quarter of 2024.
	The transition from theory to practice is hindered by many challenges including channel acquisition, scatter resolution, out-of-band response mitigation, placement optimization, and integration with existing systems.
	Most importantly, a comprehensive understanding of its potential is still missing and the precise role it shall play in 6G remains ambiguous.
	For example, \gls{ris} could be incorporated into the transmitter and receiver for \emph{beamforming}, employed as a free-rider information source for \emph{modulation}, or placed in space as a standalone device for \emph{channel shaping}.
	These applications have distinctive requirements and trade-offs, but they are not mutually exclusive and can be evolved into a versatile tool that blurs the boundary between the network and environment.
	Imagine a truly wireless future where most objects become aware and responsive by smartly exploiting the surrounding waves to energize itself, communicate with others, sense the environment, and support other devices when idle.
	% can recycle the
	% only a few ``source'' devices radiate and other objects can exploit the surrounding waves to energize itself, communicate with others, sense the environment, and support other devices when idle.
	% the most devices can exploit the existing
\end{section}

\begin{section}{Objectives}

\end{section}

\begin{section}{Outline and Contributions}

\end{section}

\begin{section}{Publications}
\end{section}
