%!TEX root = ../thesis.tex

\graphicspath{{assets/chapter_1/}}

\chapter{Getting started}

\begin{section}{Introduction}
	\begin{subsection}{\glsfmtlong{swipt}}
		With the great advance in communication performance, a bottleneck of wireless networks has come to energy supply. \gls{swipt} is a promising solution to connect and power mobile devices via \gls{rf} waves. It provides low power at \si{\uW} level but broad coverage up to hundreds of meters in a sustainable and controllable manner, bringing more opportunities to the \gls{iot} and \gls{m2m} networks. The upsurge in wireless devices, together with the decrease of electronics power consumption, calls for a re-thinking of future wireless networks based on \gls{wpt} and \gls{swipt} \cite{Clerckx2019}.

		The concept of \gls{swipt} was first cast in \cite{Varshney2008}, where the authors investigated the \gls{r-e} tradeoff for a flat Gaussian channel and typical discrete channels. \cite{Zhou2013} proposed two practical co-located information and power receivers, i.e., \gls{ts} and \gls{ps}. Dedicated information and energy beamforming were then investigated in \cite{Zhang2013,Park2014} to characterize the \gls{r-e} region for multi-antenna broadcast and interference channels. On the other hand, \cite{Trotter2009} pointed out that the \gls{rf}-to-\gls{dc} conversion efficiency of rectifiers depends on the input power and waveform shape. It implies that the modeling of the energy harvester, particularly its nonlinearity, has a crucial impact on the waveform preference, resource allocation, and system design of any wireless-powered systems \cite{Trotter2009,Clerckx2018,Clerckx2019}. Motivated by this, \cite{Clerckx2016a} derived a tractable nonlinear harvester model based on the Taylor expansion of diode I-V characteristics, and performed joint waveform and beamforming design for \gls{wpt}. Simulation and experiments showed the benefit of modeling energy harvester nonlinearity in real system design \cite{Kim2019,Kim2020a} and demonstrated the joint waveform and beamforming strategy as a key technique to expand the operation range \cite{Kim2021}. A low-complexity adaptive waveform design by \gls{smf} was proposed in \cite{Clerckx2017} to exploit the rectifier nonlinearity, whose advantage was then demonstrated in a prototype with channel acquisition \cite{Kim2017}. Beyond \gls{wpt}, \cite{Clerckx2018b} uniquely showed that the rectifier nonlinearity brings radical changes to \gls{swipt} design, namely (i) modulated and unmodulated waveforms are not equally suitable for wireless power delivery; (ii) a multi-carrier unmodulated waveform superposed to a multi-carrier modulated waveform can enlarge the \gls{r-e} region; (iii) a combination of \gls{ps} and \gls{ts} is generally the best strategy; (iv) the optimal input distribution is not the conventional \gls{cscg}; (v) modeling rectifier nonlinearity is beneficial to system performance and essential to efficient \gls{swipt} design. Those observations, validated experimentally in \cite{Kim2019}, led to the question: \emph{What is the optimal input distribution for \gls{swipt} under nonlinearity?} This question was answered in \cite{Varasteh2020} for single-carrier \gls{swipt}, and some attempts were further made in \cite{Varasteh2019d} for multi-carrier \gls{swipt}. The answers shed new light to the fundamental limits of \gls{swipt} and practical signaling (e.g., modulation and waveform) strategies. It is now well understood from \cite{Clerckx2018b,Varasteh2020,Varasteh2019d} that, due to harvester nonlinearity, a combination of \gls{cscg} and on-off keying in single-carrier setting and non-zero mean asymmetric inputs in multi-carrier setting lead to significantly larger \gls{r-e} region compared to conventional \gls{cscg}. Recently, \cite{Varasteh2020a} used machine learning techniques to design \gls{swipt} signaling under nonlinearity to complement the information-theoretic results of \cite{Varasteh2020}, and new modulation schemes were subsequently invented.
	\end{subsection}





	\begin{subsection}{\glsfmtlong{ris}}
		\gls{ris} has recently emerged as a promising technique that adapts the propagation environment to enhance the spectrum and energy efficiency. In practice, a \gls{ris} consists of multiple individual sub-wavelength reflecting elements to adjust the amplitude and phase of the incoming signal (i.e., passive beamforming). Different from the relay, backscatter and frequency-selective surface \cite{Anwar2018}, the \gls{ris} assists the primary transmission using passive components with negligible thermal noise but is limited to frequency-dependent reflection.

		Inspired by the development of real-time reconfigurable metamaterials \cite{Cui2014}, the authors of \cite{Liaskos2018} introduced a programmable metasurface that steers or polarizes the electromagnetic wave at a specific frequency to mitigate signal attenuation. \cite{Wu2018} proposed a \gls{ris}-assisted \gls{miso} system and jointly optimized the precoder at the \gls{ap} and the phase shifts at the \gls{ris} to minimize the transmit power. The active and passive beamforming problem was then extended to the discrete phase shift case \cite{Wu2019a} and the multi-user case \cite{Wu2019}. In \cite{Abeywickrama2020}, the authors investigated the impact of non-zero resistance on the reflection pattern and emphasized the coupling between reflection amplitude and phase shift in practice. To estimate the cascaded \gls{ap}-\gls{ris}-\gls{ue} link without \gls{rf}-chains at the \gls{ris}, practical protocols were developed based on element-wise on/off switching \cite{Nadeem2019}, training sequence and reflection pattern design \cite{You2019,Kang2020}, and compressed sensing \cite{Wang2020}. The hardware architecture, design challenges, and application opportunities of practical \gls{ris} were covered in \cite{Wu2020}. In \cite{Dai2020}, a prototype \gls{ris} with \num{256} \num{2}-bit elements based on \gls{pin} diodes was developed to support real-time video transmission at \si{GHz} and mmWave frequency.
	\end{subsection}


	\begin{subsection}{\glsfmtshort{ris}-Aided \glsfmtshort{swipt}}
		By integrating \gls{ris} with \gls{swipt}, the constructive reflection can boost the end-to-end power efficiency and improve the \gls{r-e} tradeoff. In multi-user cases, dedicated energy beams were proved unnecessary for the weighted sum-power maximization \cite{Wu2020b} but essential when fairness issue is considered \cite{Tang2019}. It was also claimed that \gls{los} links could boost the power efficiency since rank-deficient channels require fewer energy beams \cite{Wu2020a}. However, \cite{Wu2020b,Tang2019,Wu2020a} were based on a linear energy harvester model that is known in both the \gls{rf} and the communication literature to be inefficient and inaccurate \cite{Clerckx2019,Trotter2009,Clerckx2018,Clerckx2016a,Kim2019,Kim2020a,Kim2021,Clerckx2017,Kim2017,Clerckx2018b,Varasteh2020,Varasteh2019d,Varasteh2020a}. Based on practical \gls{ris} and harvester models, \cite{Xu2021c} introduced a scalable resource allocation framework for a large-scale tile-based \gls{ris}-assisted \gls{swipt} system, where the optimization consists of a reflection design stage and a joint reflection selection and precoder design stage. The proposed framework provides a flexible tradeoff between performance and complexity. To the best of our knowledge, all existing papers considered resource allocation and beamforming design for dedicated information and energy users in a single-carrier network. In this paper, we instead build our design based on a proper nonlinear harvester model that captures the dependency of the output \gls{dc} power on both the power and shape of the input waveform, and marry the benefits of joint multi-carrier waveform and active beamforming optimization for \gls{swipt} with the passive beamforming capability of \gls{ris}, to investigate the \gls{r-e} tradeoff for one \gls{swipt} user with co-located information decoder and energy harvester. We ask ourselves the important question: \emph{How to jointly exploit the spatial domain and the frequency domain efficiently through joint waveform and beamforming design to enlarge the \gls{r-e} region of \gls{ris}-aided \gls{swipt}?} The contributions of this paper are summarized as follows.

		\emph{First,} we propose a novel \gls{ris}-aided \gls{swipt} architecture based on joint waveform, active and passive beamforming design under the diode nonlinear model \cite{Clerckx2016a}. Although this tractable harvester model accurately reveals how the input power level and waveform shape influence the output \gls{dc} power, it also introduces design challenges such as frequency coupling (i.e., components of different frequencies compensate and produce \gls{dc}), waveform coupling (i.e., different waveforms jointly contribute to \gls{dc}), and high-order objective function. To make an efficient use of the rectifier nonlinearity, we superpose a multi-carrier unmodulated power waveform (deterministic multisine) to a multi-carrier modulated information waveform and evaluate the performance under the \gls{ts} and \gls{ps} receiving modes. The proposed joint waveform, active and passive beamforming architecture exploits the rectifier nonlinearity, the channel selectivity, and a beamforming gain across frequency and spatial domains to enlarge the achievable \gls{r-e} region. This is the first paper to propose a joint waveform, active and passive beamforming architecture for \gls{ris}-aided \gls{swipt}.

		\emph{Second,} we characterize each \gls{r-e} boundary point by energy maximization under a rate constraint. The problem is solved by a \gls{bcd} algorithm based on the \gls{csit}. For active beamforming, we prove that the global optimal active information and power precoders coincide at \gls{mrt} even with rectifier nonlinearity. For passive beamforming, we propose a \gls{sca} algorithm and retrieve the \gls{ris} phase shift by eigen decomposition with optimality proof. Finally, the superposed waveform and the splitting ratio are optimized by the \gls{gp} technique. The \gls{ris} phase shift, active precoder, and waveform amplitude are updated iteratively until convergence. This is the first paper to jointly optimize waveform and active/passive beamforming in \gls{ris}-aided \gls{swipt}.

		\emph{Third,} we introduce two closed-form adaptive waveform schemes to avoid the exponential complexity of the \gls{gp} algorithm. To facilitate practical \gls{swipt} implementation, the \gls{wf} strategy for modulated waveform and the \gls{smf} strategy for multisine waveform are combined in time and power domains, respectively. The passive beamforming design is also adapted to accommodate the low-complexity waveform schemes. The proposed low-complexity \gls{bcd} algorithm achieves a good balance between performance and complexity.

		\emph{Fourth,} we provide numerical results to evaluate the proposed algorithms. It is concluded that (i) \gls{ris} enables constructive reflection and flexible subchannel design in the frequency domain that is essential for \gls{swipt} systems; (ii) \gls{ris} mainly affects the effective channel instead of the waveform design; (iii) multisine waveform is beneficial to energy transfer especially when the number of subbands is large; (iv) \gls{ts} is preferred at low \gls{snr} while \gls{ps} is preferred at high \gls{snr}; (v) there exist two optimal \gls{ris} development locations, one close to the \gls{ap} and one close to the \gls{ue}; (vi) the output \gls{snr} scales linearly with the number of transmit antennas and quadratically with the number of \gls{ris} elements; (vii) due to the rectifier nonlinearity, the output \gls{dc} scales quadratically with the number of transmit antennas and quartically with the number of \gls{ris} elements; (viii) for narrowband \gls{swipt}, the optimal active and passive beamforming for any \gls{r-e} point are also optimal for the whole \gls{r-e} region; (ix) for broadband \gls{swipt}, the optimal active and passive beamforming depend on specific \gls{r-e} point and require adaptive designs; (x) the proposed algorithms are robust to practical impairments such as inaccurate cascaded \gls{csit} and finite \gls{ris} reflection states.
	\end{subsection}
\end{section}

\begin{section}{System Model}\label{se:system_model}
	\begin{figure}[!t]
		\centering
		\def\svgwidth{0.9\columnwidth}
		\input{assets/chapter_1/system.eps_tex}
		\caption{An \gls{ris}-aided multi-carrier \gls{miso} \gls{swipt} system.}
		\label{fi:system}
	\end{figure}

	As shown in Fig.~\ref{fi:system}, we propose a \gls{ris}-aided \gls{swipt} system where an $M$-antenna \gls{ap} delivers information and power simultaneously, through an $L$-element \gls{ris}, to a single-antenna \gls{ue} over $N$ orthogonal evenly-spaced subbands. We consider a quasi-static block fading model and assume the \gls{csit} of direct and cascaded channels is known. The signals reflected by two or more times are omitted, and the noise power is assumed too small to be harvested.


	\begin{subsection}{Transmitted Signal}
		Following \cite{Clerckx2018b}, we superpose a multi-carrier modulated information-bearing waveform to a multi-carrier unmodulated power-dedicated deterministic multisine to boost the spectrum and energy efficiency. The information signal transmitted over subband $n \in \{1, \dots, N\}$ at time $t$ is
		\begin{equation}
			\boldsymbol{x}_{\mathrm{I},n}(t) = \Re\left\{\boldsymbol{w}_{\mathrm{I},n} \tilde{x}_{\mathrm{I},n}(t) e^{j2{\pi}{f_n}{t}}\right\},
		\end{equation}
		where $\boldsymbol{w}_{\mathrm{I},n} \in \mathbb{C}^{M \times 1}$ is the information precoder at subband $n$, $\tilde{x}_{\mathrm{I},n}\sim\mathcal{CN}(0,1)$ is the information symbol at subband $n$, and $f_n$ is the frequency of subband $n$. On the other hand, the power signal transmitted over subband $n$ at time $t$ is
		\begin{equation}
			\boldsymbol{x}_{\mathrm{P},n}(t) = \Re\left\{\boldsymbol{w}_{\mathrm{P},n} e^{j2{\pi}{f_n}{t}}\right\},
		\end{equation}
		where $\boldsymbol{w}_{\mathrm{P},n} \in \mathbb{C}^{M \times 1}$ is the power precoder at subband $n$. Therefore, the superposed signal transmitted over all subbands at time $t$ is
		\begin{equation}
			\boldsymbol{x}(t) = \Re{\left\{\sum_{n=1}^N(\boldsymbol{w}_{\mathrm{I},n}\tilde{x}_{\mathrm{I},n}(t)+\boldsymbol{w}_{\mathrm{P},n}){e^{j2{\pi}{f_n}{t}}}\right\}}.
		\end{equation}
		We also define $\boldsymbol{w}_{\mathrm{I/P}} \triangleq [\boldsymbol{w}_{\mathrm{I/P},1}^T,\dots,\boldsymbol{w}_{\mathrm{I/P},N}^T]^T \in \mathbb{C}^{MN \times 1}$.
	\end{subsection}


	\begin{subsection}{Reflection Pattern and Composite Channel}
		According to Green's decomposition \cite{Hansen1989}, the backscattered signal of an antenna can be decomposed into the \emph{structural mode} component and the \emph{antenna mode} component. The former is fixed and can be regarded as part of the environment multipath, while the latter is adjustable and depends on the mismatch of the antenna and load impedance. \gls{ris} element $l \in \{1, \dots, L\}$ varies its impedance $Z_l = R_l + j X_l$ to reflect the incoming signal, and the reflection coefficient is defined as
		\begin{equation}
			\phi_l = \frac{Z_l - Z_0}{Z_l + Z_0} \triangleq \eta_l e^{j\theta_l},
		\end{equation}
		where $Z_0$ is the real-valued characteristic impedance, $\eta_l \in [0, 1]$ is the reflection amplitude,\footnote{Due to the non-zero power consumption at the \gls{ris}, $R_l > 0$ in practice such that $\eta_l < 1$ and is a function of $\theta_l$. This paper sticks to the most common \gls{ris} model where the reflection amplitude equals \num{1} so as to reduce the design complexity and provide a primary benchmark for practical \gls{ris}-aided \gls{swipt}.} and $\theta_l \in [0,2\pi)$ is the phase shift. We also define $\boldsymbol{\phi} \triangleq [\phi_1, \dots, \phi_L]^H \in \mathbb{C}^{L \times 1}$ and $\boldsymbol{\Theta} \triangleq \mathrm{diag}(\phi_1, \dots, \phi_L)=\mathrm{diag}(\boldsymbol{\Phi}^*) \in \mathbb{C}^{L \times L}$ as the \gls{ris} vector and matrix, respectively.

		\begin{remark}\label{re:reflection_coefficient}
			The element impedance $Z_l$ maps to the reflection coefficient $\phi_l$ uniquely. Since the reactance $X_l$ depends on the frequency, the reflection coefficient $\phi_l$ is also a function of frequency and cannot be designed independently at different subbands. In this paper, we assume the bandwidth is small compared to the operating frequency such that the reflection coefficient of each \gls{ris} element is the same at all subbands.
		\end{remark}

		At subband $n$, we denote the \gls{ap}-\gls{ue} direct channel as $\boldsymbol{h}_{\mathrm{D},n}^H \in \mathbb{C}^{1 \times M}$, the \gls{ap}-\gls{ris} incident channel as $\boldsymbol{H}_{\mathrm{I},n} \in \mathbb{C}^{L \times M}$, and the \gls{ris}-\gls{ue} reflected channel as $\boldsymbol{h}_{\mathrm{R},n}^H \in \mathbb{C}^{1 \times L}$. The auxiliary \gls{ap}-\gls{ris}-\gls{ue} link can be modeled as a concatenation of the incident channel, the \gls{ris} reflection, and the reflected channel. Hence, the composite equivalent channel reduces to
		\begin{equation}\label{eq:h_n}
			\boldsymbol{h}_{n}^H = \boldsymbol{h}_{\mathrm{D},n}^H + \boldsymbol{h}_{\mathrm{R},n}^H \boldsymbol{\Theta} \boldsymbol{H}_{\mathrm{I},n} = \boldsymbol{h}_{\mathrm{D},n}^H + \boldsymbol{\phi}^H \boldsymbol{V}_{n},
		\end{equation}
		where we define the cascaded incident-reflected channel at subband $n$ as $\boldsymbol{V}_{n} \triangleq \mathrm{diag}(\boldsymbol{h}_{\mathrm{R},n}^H)\boldsymbol{H}_{\mathrm{I},n} \in \mathbb{C}^{L \times M}$. We also define $\boldsymbol{h} \triangleq [\boldsymbol{h}_1^T,\dots,\boldsymbol{h}_N^T]^T \in \mathbb{C}^{MN \times 1}$.

		\begin{remark}\label{re:subband_tradeoff}
			The cascaded channel varies at different subbands, but the reflection cannot be designed independently at different frequencies. Therefore, there exists a tradeoff for the passive beamforming design in the frequency domain, and the composite subchannels should be tuned adaptively to meet the specific requirement of multi-carrier \gls{swipt}. For example, one can design the reflection pattern to either enhance the strongest subband (e.g., $\max_{\boldsymbol{\phi},n} \lVert \boldsymbol{h}_n \rVert$), or improve the fairness among subbands (e.g., $\max_{\boldsymbol{\phi}} \min_n \lVert \boldsymbol{h}_n \rVert$). That is to say, \gls{ris} essentially enables a flexible subchannel design. In the \gls{miso} case, a similar effect also exists in the spatial domain. Therefore, each reflection coefficient is indeed shared by $M$ antennas over $N$ subbands.
		\end{remark}
	\end{subsection}


	\begin{subsection}{Received Signal}
		The received superposed signal at the single-antenna \gls{ue} is\footnote{We assume that the time difference of signal arrival via direct and auxiliary link is negligible compared to the symbol period.}
		\begin{equation}
			y(t) = \Re\left\{\sum_{n=1}^N{\Bigl(\boldsymbol{h}_{n}^H}{(\boldsymbol{w}_{\mathrm{I},n}\tilde{x}_{\mathrm{I},n}(t)+\boldsymbol{w}_{\mathrm{P},n})+\tilde{n}_n(t)\Bigr)}{e^{j2{\pi}{f_n}{t}}}\right\},
		\end{equation}
		where $\tilde{n}_n(t)$ is the noise at \gls{rf} band $n$. Note that the modulated component can be used for energy harvesting if necessary, but the multisine component carries no information and cannot be used for information decoding.
	\end{subsection}


	\begin{subsection}{Receiving Modes}
		\begin{figure}[!t]
			\centering
			\subfloat[\gls{ts} receiver\label{fi:ts_receiver}]{
				\resizebox{0.9\columnwidth}{!}{
					\input{assets/chapter_1/ts_receiver.tex}
				}
			}
			\\
			\subfloat[\gls{ps} receiver\label{fi:ps_receiver}]{
				\resizebox{0.9\columnwidth}{!}{
					\input{assets/chapter_1/ps_receiver.tex}
				}
			}
			\caption{Diagrams of practical co-located receivers.}
			\label{fi:receiver}
		\end{figure}

		As illustrated in Fig.~\ref{fi:receiver}, there are two practical receiving modes for the co-located information decoder and energy harvester \cite{Zhou2013}. The \gls{ts} receiver divides each transmission block into orthogonal data and energy sessions with duration $1-\eta$ and $\eta$, respectively. During each session, the transmitter optimizes the waveform for either \gls{wit} or \gls{wpt}, while the receiver activates the information decoder or the energy harvester correspondingly. The duration ratio $\eta$ controls the \gls{r-e} tradeoff and is independent from the waveform and beamforming design. On the other hand, the \gls{ps} receiver splits the incoming signal into individual data and energy streams with power ratio $1-\rho$ and $\rho$, respectively. The data stream is fed into the information decoder while the energy stream is fed into the energy harvester. During each transmission block, the superposed waveform and splitting ratio are jointly designed to achieve different \gls{r-e} tradeoffs. In the following context, we consider the optimization with the \gls{ps} receiver, as the \gls{ts} receiver can be regarded as a special case (i.e., a time sharing between $\rho=0$ and $\rho=1$).
	\end{subsection}


	\begin{subsection}{Information Decoder}
		A major benefit of the superposed waveform is that the multisine is deterministic and creates no interference to the modulated waveform \cite{Clerckx2018b}. Therefore, the achievable rate is\footnote{It requires waveform cancellation or translated demodulation \cite{Clerckx2018b}.}
		\begin{equation}\label{eq:R}
			R(\boldsymbol{\phi},\boldsymbol{w}_{\mathrm{I}},\rho) = \sum_{n=1}^N{\log_2\left(1+\frac{(1-\rho)\lvert \boldsymbol{h}_{n}^H\boldsymbol{w}_{\mathrm{I},n} \rvert^2}{\sigma_n^2}\right)},
		\end{equation}
		where $\sigma_n^2$ is the noise variance at the \gls{rf}-band and during the \gls{rf}-to-baseband conversion on tone $n$.
	\end{subsection}


	\begin{subsection}{Energy Harvester}\label{se:energy_harvester}
		\begin{figure}[!t]
			\centering
			\subfloat[Antenna equivalent circuit\label{fi:antenna_equivalent_circuit}]{
				\resizebox{0.45\columnwidth}{!}{
					% \includegraphics{assets/antenna_equivalent_circuit.eps}
					\input{assets/chapter_1/antenna_equivalent_circuit.tex}
				}
			}
			\subfloat[A single diode rectifier\label{fi:single_diode_rectifier}]{
				\resizebox{0.45\columnwidth}{!}{
					% \includegraphics{assets/a_single_diode_rectifier.eps}
					\input{assets/chapter_1/a_single_diode_rectifier.tex}
				}
			}
			\caption{Equivalent circuits of receive antenna and energy harvester.}
		\end{figure}

		Taken from \cite{Clerckx2016a}, the rectenna model used in this section captures the dependency of the output \gls{dc} on both the power and shape of the received signal. Fig.~\subref*{fi:antenna_equivalent_circuit} illustrates the equivalent circuit of an ideal antenna, where the antenna has a resistance $R_{\mathrm{A}}$ and the incoming signal creates a voltage source $v_{\mathrm{S}}(t)$. Let $R_{\mathrm{in}}$ be the total input resistance of the rectifier and matching network, and we assume the voltage across the matching network is negligible. When perfectly matched (i.e., $R_{\mathrm{in}}=R_{\mathrm{A}}$), the rectifier input voltage is $v_{\mathrm{in}}(t)=y(t)\sqrt{\rho R_{\mathrm{A}}}$. Consider a simplified rectifier model in Fig.~\subref*{fi:single_diode_rectifier} where a single series diode is followed by a low-pass filter with a parallel load. As detailed in \cite{Clerckx2018b}, a truncated Taylor expansion of the diode I-V characteristic equation suggests that maximizing the average output \gls{dc} is equivalent to maximizing a monotonic function\footnote{This small-signal expansion model is only valid for the nonlinear operation region of the diode, and the I-V relationship would be linear if the diode behavior is dominated by the load \cite{Clerckx2016a}.}
		\begin{equation}\label{eq:z}
			z(\boldsymbol{\phi},\boldsymbol{w}_{\mathrm{I}},\boldsymbol{w}_{\mathrm{P}},\rho)=\sum_{i\,\mathrm{even},i\ge2}^{n_0}{k_i}{\rho^{i/2}}{R_{\mathrm{A}}^{i/2}}{\mathbb{E}\left\{\mathbb{A}\left\{y(t)^i\right\}\right\}},
		\end{equation}
		where $n_0$ is the truncation order and $k_i \triangleq i_{\mathrm{S}}/i!(n'v_{\mathrm{T}})^i$ is the diode coefficient ($i_{\mathrm{S}}$ is the reverse bias saturation current, $n'$ is the diode ideality factor, $v_{\mathrm{T}}$ is the thermal voltage). With a slight abuse of notation, we refer to $z$ as the average output \gls{dc} in this paper. It can be observed that the conventional linear harvester model, where the output \gls{dc} power equals the sum of the power harvested on each frequency, is a special case of \eqref{eq:z} with $n_0=2$. However, due to the coupling effect among different frequencies, some high-order frequency components compensate each other in frequency and further contribute to the output \gls{dc} power. In other words, even-order terms with $i \ge 4$ account for the nonlinear diode behavior. For simplicity, we choose $n_0=4$ to investigate the fundamental rectifier nonlinearity, and define $\beta_2 \triangleq {k_2}{R_{\mathrm{A}}}$, $\beta_4 \triangleq {k_4}{R_{\mathrm{A}}^2}$ to rewrite $z$ by \eqref{eq:z_expand}. Note that $\mathbb{E}\left\{\lvert\tilde{x}_{\mathrm{I},n}\rvert^2\right\}=1$ but $\mathbb{E}\left\{\lvert\tilde{x}_{\mathrm{I},n}\rvert^4\right\}=2$ applies a modulation gain on the fourth-order \gls{dc} terms. Let $\boldsymbol{W}_{\mathrm{I/P}} \triangleq \boldsymbol{w}_{\mathrm{I/P}}\boldsymbol{w}_{\mathrm{I/P}}^H \in \mathbb{C}^{MN \times MN}$. As illustrated by Fig.~\ref{fi:block_diagonal}, $\boldsymbol{W}_{\mathrm{I/P}}$ can be divided into $N \times N$ blocks of size $M \times M$, and we let $\boldsymbol{W}_{\mathrm{I/P},k}$ keep its block diagonal $k \in \{-N+1,\dots,N-1\}$ and set all other blocks to $\boldsymbol{0}$. Hence, the components of $z$ reduce to \eqref{eq:y_I2}--\eqref{eq:y_P4} \cite{Golub2013}.

		\begin{figure*}[!t]
			\begin{equation*}
				\boldsymbol{W}_{\mathrm{I/P}}=
				\begin{tikzpicture}[>=stealth,thick,baseline,every right delimiter/.append style={name=rd},]
					\matrix [matrix of math nodes,left delimiter=(,right delimiter=)] (m)
					{
						\boldsymbol{w}_{\mathrm{I/P},1}\boldsymbol{w}_{\mathrm{I/P},1}^H & \boldsymbol{w}_{\mathrm{I/P},1}\boldsymbol{w}_{\mathrm{I/P},2}^H & \dots & \boldsymbol{w}_{\mathrm{I/P},1}\boldsymbol{w}_{\mathrm{I/P},N}^H \\
						\boldsymbol{w}_{\mathrm{I/P},2}\boldsymbol{w}_{\mathrm{I/P},1}^H & \boldsymbol{w}_{\mathrm{I/P},2}\boldsymbol{w}_{\mathrm{I/P},2}^H & \ddots & \vdots \\
						\vdots & \ddots & \ddots & \boldsymbol{w}_{\mathrm{I/P},N-1}\boldsymbol{w}_{\mathrm{I/P},N}^H \\
						\boldsymbol{w}_{\mathrm{I/P},N}\boldsymbol{w}_{\mathrm{I/P},1}^H & \dots & \boldsymbol{w}_{\mathrm{I/P},N}\boldsymbol{w}_{\mathrm{I/P},N-1}^H & \boldsymbol{w}_{\mathrm{I/P},N}\boldsymbol{w}_{\mathrm{I/P},N}^H \\
					};
					\draw[dotted,thick] (m-4-1.north west) rectangle (m-4-1.south east);
					\draw[dotted,thick,fill=gray,opacity=0.125] (m-2-1.north west) rectangle (m-2-1.south east); \draw[dotted,thick,fill=gray,opacity=0.125] (m-3-2.north west) rectangle (m-3-2.south east); \draw[dotted,thick,fill=gray,opacity=0.125] (m-4-3.north west) rectangle (m-4-3.south east);
					\draw[dashed,thick,fill=gray,opacity=0.5] (m-1-1.north west) rectangle (m-1-1.south east); \draw[dashed,thick,fill=gray,opacity=0.5] (m-2-2.north west) rectangle (m-2-2.south east); \draw[dashed,thick,fill=gray,opacity=0.5] (m-3-3.north west) rectangle (m-3-3.south east); \draw[dashed,thick,fill=gray,opacity=0.5] (m-4-4.north west) rectangle (m-4-4.south east);
					\draw[solid,thick,fill=gray,opacity=0.25] (m-1-2.north west) rectangle (m-1-2.south east); \draw[solid,thick,fill=gray,opacity=0.25] (m-2-3.north west) rectangle (m-2-3.south east); \draw[solid,thick,fill=gray,opacity=0.25] (m-3-4.north west) rectangle (m-3-4.south east);
					\draw[solid,thick] (m-1-4.north west) rectangle (m-1-4.south east);
					\draw[<-] (m-4-3.south|-m.south) -- ++(0.5,-0.15) node[below]{$k=-1$};
					\draw[<-] (rd.east|-m.south) -- ++(0.5,-0.15) node[right]{$k=0$};
					\draw[<-] (rd.east|-m-3-4.east) -- ++(0.5,-0.15) node[right]{$k=1$};
				\end{tikzpicture}
			\end{equation*}
			\caption{$\boldsymbol{W}_{\mathrm{I/P}}$ consists of $N \times N$ blocks of size $M \times M$. $\boldsymbol{W}_{\mathrm{I/P},k}$ keeps the $k$-th block diagonal of $\boldsymbol{W}_{\mathrm{I/P}}$ and nulls all remaining blocks. Solid, dashed and dotted blocks correspond to $k>0$, $k=0$ and $k<0$, respectively. For $\boldsymbol{w}_{\mathrm{I/P},n_1}\boldsymbol{w}_{\mathrm{I/P},n_2}^H$, the $k$-th block diagonal satisfies $k=n_2-n_1$.}
			\label{fi:block_diagonal}
		\end{figure*}

		\begin{figure*}[!b]
			\hrule
			\begin{align}
				&z(\boldsymbol{\phi},\boldsymbol{w}_{\mathrm{I}},\boldsymbol{w}_{\mathrm{P}},\rho) = \beta_2\rho\Bigl(\mathbb{E}\left\{\mathbb{A}\left\{y_{\mathrm{I}}^2(t)\right\}\right\}+\mathbb{A}\left\{y_{\mathrm{P}}^2(t)\right\}\Bigr)+\beta_4\rho^2\Bigl(\mathbb{E}\left\{\mathbb{A}\left\{y_{\mathrm{I}}^4(t)\right\}\right\}+\mathbb{A}\left\{y_{\mathrm{P}}^4(t)\right\}+6\mathbb{E}\left\{\mathbb{A}\left\{y_{\mathrm{I}}^2(t)\right\}\right\}\mathbb{A}\left\{y_{\mathrm{P}}^2(t)\right\}\Bigr),\label{eq:z_expand}\\
				&\mathbb{E}\left\{\mathbb{A}\left\{y_{\mathrm{I}}^2(t)\right\}\right\} = \frac{1}{2}\sum_{n=1}^N{(\boldsymbol{h}_{n}^H\boldsymbol{w}_{\mathrm{I},n})(\boldsymbol{h}_{n}^H\boldsymbol{w}_{\mathrm{I},n})^*} = \frac{1}{2}\boldsymbol{h}^H\boldsymbol{W}_{\mathrm{I},0}\boldsymbol{h},\label{eq:y_I2}\\
				&\mathbb{E}\left\{\mathbb{A}\left\{y_{\mathrm{I}}^4(t)\right\}\right\} = \frac{3}{4}\left(\sum_{n=1}^N{(\boldsymbol{h}_{n}^H\boldsymbol{w}_{\mathrm{I},n})(\boldsymbol{h}_{n}^H\boldsymbol{w}_{\mathrm{I},n})^*}\right)^2 = \frac{3}{4}(\boldsymbol{h}^H\boldsymbol{W}_{\mathrm{I},0}\boldsymbol{h})^2,\label{eq:y_I4}\\
				&\mathbb{A}\left\{y_{\mathrm{P}}^2(t)\right\} = \frac{1}{2}\sum_{n=1}^N{(\boldsymbol{h}_{n}^H\boldsymbol{w}_{\mathrm{P},n})(\boldsymbol{h}_{n}^H\boldsymbol{w}_{\mathrm{P},n})^*} = \frac{1}{2}\boldsymbol{h}^H\boldsymbol{W}_{\mathrm{P},0}\boldsymbol{h},\label{eq:y_P2}\\
				&\mathbb{A}\left\{y_{\mathrm{P}}^4(t)\right\} = \frac{3}{8}\sum_{\substack{{n_1},{n_2},{n_3},{n_4}\\{n_1}+{n_2}={n_3}+{n_4}}}{(\boldsymbol{h}_{{n_1}}^H\boldsymbol{w}_{\mathrm{P},{n_1}})(\boldsymbol{h}_{{n_2}}^H\boldsymbol{w}_{\mathrm{P},{n_2}})(\boldsymbol{h}_{{n_3}}^H\boldsymbol{w}_{\mathrm{P},{n_3}})^*(\boldsymbol{h}_{{n_4}}^H\boldsymbol{w}_{\mathrm{P},{n_4}})^*} = \frac{3}{8}\sum_{k=-N+1}^{N-1}(\boldsymbol{h}^H\boldsymbol{W}_{\mathrm{P},k}\boldsymbol{h})(\boldsymbol{h}^H\boldsymbol{W}_{\mathrm{P},k}\boldsymbol{h})^*.\label{eq:y_P4}
			\end{align}
		\end{figure*}
	\end{subsection}


	\begin{subsection}{Rate-Energy Region}
		The achievable \gls{r-e} region is defined as
		\begin{align}
			\mathcal{C}_{\mathrm{\gls{r-e}}}
			&\triangleq \biggl\{(R_{\mathrm{ID}}, z_{\mathrm{EH}}) \in \mathbb{R}_+^2 \mid R_{\mathrm{ID}} \le R, z_{\mathrm{EH}} \le z,\nonumber\\
			&\quad \frac{1}{2}\left(\lVert{\boldsymbol{w}_{\mathrm{I}}}\rVert^2+\lVert{\boldsymbol{w}_{\mathrm{P}}}\rVert^2\right) \le P\biggr\},
		\end{align}
		where $P$ is the average transmit power budget and 1/2 converts the peak value of the sine waves to the average value.
	\end{subsection}
\end{section}


\begin{section}{Problem Formulation}\label{se:problem_formulation}
	We characterize each \gls{r-e} boundary point through a current maximization problem subject to sum rate, transmit power, and reflection amplitude constraints as
	\begin{maxi!}
		{\scriptstyle{\boldsymbol{\phi},\boldsymbol{w}_{\mathrm{I}},\boldsymbol{w}_{\mathrm{P}},\rho}}{z(\boldsymbol{\phi},\boldsymbol{w}_{\mathrm{I}},\boldsymbol{w}_{\mathrm{P}},\rho)}{\label{op:original}}{\label{ob:original}}
		\addConstraint{R(\boldsymbol{\phi},\boldsymbol{w}_{\mathrm{I}},\rho) \ge \bar{R}}\label{co:original_rate}
		\addConstraint{\frac{1}{2}\left(\lVert{\boldsymbol{w}_{\mathrm{I}}}\rVert^2+\lVert{\boldsymbol{w}_{\mathrm{P}}}\rVert^2\right)\le{P}}\label{co:original_power}
		\addConstraint{\lvert{\boldsymbol{\phi}}\rvert=\boldsymbol{1}}\label{co:original_modulus}
		\addConstraint{0 \le \rho \le 1.}
	\end{maxi!}
	Problem~\eqref{op:original} is intricate because of the coupled variables in \eqref{ob:original}, \eqref{co:original_rate} and the non-convex constraint \eqref{co:original_modulus}. To obtain a feasible solution, we propose a \gls{bcd} algorithm that iteratively updates (i) the \gls{ris} phase shift; (ii) the active precoder; (iii) the waveform amplitude and splitting ratio, until convergence.


	\begin{subsection}{Passive Beamforming}
		In this section, we optimize the \gls{ris} phase shift $\boldsymbol{\phi}$ for any given waveform $\boldsymbol{w}_{\mathrm{I/P}}$ and splitting ratio $\rho$. Note that
		\begin{align}
			\lvert \boldsymbol{h}_{n}^H\boldsymbol{w}_{\mathrm{I},n} \rvert^2
			& = \boldsymbol{w}_{\mathrm{I},n}^H\boldsymbol{h}_n\boldsymbol{h}_n^H\boldsymbol{w}_{\mathrm{I},n}\nonumber\\
			& = \boldsymbol{w}_{\mathrm{I},n}^H(\boldsymbol{h}_{\mathrm{D},n}+\boldsymbol{V}_n^H\boldsymbol{\phi})(\boldsymbol{h}_{\mathrm{D},n}^H+\boldsymbol{\phi}^H\boldsymbol{V}_n)\boldsymbol{w}_{\mathrm{I},n}\nonumber\\
			& = \boldsymbol{w}_{\mathrm{I},n}^H\boldsymbol{M}_n^H\boldsymbol{\Phi}\boldsymbol{M}_n\boldsymbol{w}_{\mathrm{I},n}\nonumber\\
			& = \mathrm{tr}(\boldsymbol{M}_n\boldsymbol{w}_{\mathrm{I},n}\boldsymbol{w}_{\mathrm{I},n}^H\boldsymbol{M}_n^H\boldsymbol{\Phi})\nonumber\\
			& = \mathrm{tr}(\boldsymbol{C}_n\boldsymbol{\Phi}),
		\end{align}
		where $\boldsymbol{M}_n \triangleq [\boldsymbol{V}_n^H, \boldsymbol{h}_{\mathrm{D},n}]^H \in \mathbb{C}^{(L+1) \times M}$, $t'$ is an auxiliary variable with unit modulus, $\bar{\boldsymbol{\phi}} \triangleq [\boldsymbol{\phi}^H, t']^H \in \mathbb{C}^{(L+1) \times 1}$, $\boldsymbol{\Phi} \triangleq \bar{\boldsymbol{\phi}}\bar{\boldsymbol{\phi}}^H \in \mathbb{C}^{(L+1) \times (L+1)}$, $\boldsymbol{C}_n \triangleq \boldsymbol{M}_n\boldsymbol{w}_{\mathrm{I},n}\boldsymbol{w}_{\mathrm{I},n}^H\boldsymbol{M}_n^H \in \mathbb{C}^{(L+1)\times(L+1)}$. On the other hand, we define $t_{\mathrm{I/P},k}$ as
		\begin{align}
			t_{\mathrm{I/P},k}
			& \triangleq \boldsymbol{h}^H\boldsymbol{W}_{\mathrm{I/P},k}\boldsymbol{h}\nonumber\\
			& = \mathrm{tr}(\boldsymbol{h}\boldsymbol{h}^H\boldsymbol{W}_{\mathrm{I/P},k})\nonumber\\
			& = \mathrm{tr}\left((\boldsymbol{h}_{D}+\boldsymbol{V}^H\boldsymbol{\phi})(\boldsymbol{h}_{D}^H+\boldsymbol{\phi}^H\boldsymbol{V})\boldsymbol{W}_{\mathrm{I/P},k}\right)\nonumber\\
			& = \mathrm{tr}(\boldsymbol{M}^H\boldsymbol{\Phi}\boldsymbol{M}\boldsymbol{W}_{\mathrm{I/P},k})\nonumber\\
			& = \mathrm{tr}(\boldsymbol{M}\boldsymbol{W}_{\mathrm{I/P},k}\boldsymbol{M}^H\boldsymbol{\Phi})\nonumber\\
			& = \mathrm{tr}(\boldsymbol{C}_{\mathrm{I/P},k}\boldsymbol{\Phi})\label{eq:t_k},
		\end{align}
		where $\boldsymbol{V} \triangleq [\boldsymbol{V}_1,\dots,\boldsymbol{V}_N] \in \mathbb{C}^{L \times MN}$, $\boldsymbol{M} \triangleq [\boldsymbol{V}^H, \boldsymbol{h}_{D}]^H \in \mathbb{C}^{(L+1) \times MN}$, $\boldsymbol{C}_{\mathrm{I/P},k} \triangleq \boldsymbol{M}\boldsymbol{W}_{\mathrm{I/P},k}\boldsymbol{M}^H \in \mathbb{C}^{(L+1)\times(L+1)}$. On top of this, \eqref{eq:R} and \eqref{eq:z_expand} reduce respectively to
		\begin{align}
			R(\boldsymbol{\Phi})
			& = \sum_{n=1}^{N}{\log_2\left(1+\frac{(1-\rho)\mathrm{tr}(\boldsymbol{C}_n\boldsymbol{\Phi})}{\sigma_n^2}\right)},\label{eq:R_irs}\\
			z(\boldsymbol{\Phi})
			& = \frac{1}{2}{\beta_2}{\rho}(t_{\mathrm{I},0}+t_{\mathrm{P},0}) + \frac{3}{8}{\beta_4}{\rho^2} \left(2t_{\mathrm{I},0}^2 + \sum_{k=-N+1}^{N-1}{t_{\mathrm{P},k}t_{\mathrm{P},k}^*}\right)\nonumber\\
			& \quad + \frac{3}{2}{\beta_4}{\rho^2}t_{\mathrm{I},0}t_{\mathrm{P},0}.\label{eq:z_irs}
		\end{align}
		To maximize the non-concave expression \eqref{eq:z_irs}, we successively lower bound the second-order terms by their first-order Taylor expansions \cite{Adali2010}. Based on the solution at iteration $i - 1$, the approximations at iteration $i$ are
		\begin{align}
			(t_{\mathrm{I},0}^{(i)})^2
			& \ge 2 t_{\mathrm{I},0}^{(i)}t_{\mathrm{I},0}^{(i-1)} - (t_{\mathrm{I},0}^{(i-1)})^2,\label{eq:taylor_1}\\
			t_{\mathrm{P},k}^{(i)} (t_{\mathrm{P},k}^{(i)})^*
			& \ge 2 \Re\left\{t_{\mathrm{P},k}^{(i)} (t_{\mathrm{P},k}^{(i-1)})^*\right\} - t_{\mathrm{P},k}^{(i-1)} (t_{\mathrm{P},k}^{(i-1)})^*,\label{eq:taylor_2}\\
			t_{\mathrm{I},0}^{(i)} t_{\mathrm{P},0}^{(i)}
			& \ge t_{\mathrm{I},0}^{(i)} t_{\mathrm{P},0}^{(i-1)} + t_{\mathrm{P},0}^{(i)} t_{\mathrm{I},0}^{(i-1)} - t_{\mathrm{I},0}^{(i-1)} t_{\mathrm{P},0}^{(i-1)}.\label{eq:taylor_3}
		\end{align}
		Note that $t_{\mathrm{I/P},0}=\mathrm{tr}(\boldsymbol{C}_{\mathrm{I/P},0}\boldsymbol{\Phi})$ is real-valued because $\boldsymbol{C}_{\mathrm{I/P},0}$ and $\boldsymbol{\Phi}$ are Hermitian matrices. Due to symmetry \cite{Golub2013}, we have
		\begin{equation}\label{eq:coupled_terms}
			\sum_{k=-N+1}^{N-1} \Re\left\{t_{\mathrm{P},k}^{(i)} (t_{\mathrm{P},k}^{(i-1)})^*\right\} = \sum_{k=-N+1}^{N-1} t_{\mathrm{P},k}^{(i)} (t_{\mathrm{P},k}^{(i-1)})^*.
		\end{equation}
		Plugging \eqref{eq:taylor_1}--\eqref{eq:coupled_terms} into \eqref{eq:z_irs}, we obtain the \gls{dc} approximation $\tilde{z}$ as \eqref{eq:z_irs_approx} and transform problem~\eqref{op:original} to
		\begin{figure*}[!b]
			\hrule
			\begin{align}
				\tilde{z}(\boldsymbol{\Phi}^{(i)})
				& = \frac{1}{2}{\beta_2}{\rho}(t_{\mathrm{I},0}^{(i)}+t_{\mathrm{P},0}^{(i)}) + \frac{3}{8}{\beta_4}{\rho^2} \left(4 t_{\mathrm{I},0}^{(i)}t_{\mathrm{I},0}^{(i-1)} - 2 (t_{\mathrm{I},0}^{(i-1)})^2 + \sum_{k=-N+1}^{N-1}{2 t_{\mathrm{P},k}^{(i)} (t_{\mathrm{P},k}^{(i-1)})^* - t_{\mathrm{P},k}^{(i-1)} (t_{\mathrm{P},k}^{(i-1)})^*}\right)\nonumber\\
				& \quad + \frac{3}{2}{\beta_4}{\rho^2} \left(t_{\mathrm{I},0}^{(i)} t_{\mathrm{P},0}^{(i-1)} + t_{\mathrm{P},0}^{(i)} t_{\mathrm{I},0}^{(i-1)} - t_{\mathrm{I},0}^{(i-1)} t_{\mathrm{P},0}^{(i-1)}\right).\label{eq:z_irs_approx}
			\end{align}
		\end{figure*}
		\begin{maxi!}
			{\scriptstyle{\boldsymbol{\Phi}}}{\tilde{z}(\boldsymbol{\Phi})}{\label{op:irs}}{\label{ob:irs}}
			\addConstraint{R(\boldsymbol{\Phi}) \ge \bar{R}}\label{co:irs_rate}
			\addConstraint{\mathrm{diag}^{-1}(\boldsymbol{\Phi})=\boldsymbol{1}}\label{co:irs_modulus}
			\addConstraint{\boldsymbol{\Phi}\succeq{\boldsymbol{0}}}\label{co:irs_sd}
			\addConstraint{\mathrm{rank}(\boldsymbol{\Phi})=1.\label{co:irs_rank}}
		\end{maxi!}
		We then apply \gls{sdr} to the unit-rank constraint \eqref{co:irs_rank} and formulate a \gls{sdp} with approximation accuracy no greater than $\pi/4$ \cite{Luo2010b}. In this specific case, we found the solution provided by \texttt{CVX toolbox} \cite{Grant2016} to \eqref{ob:irs}-\eqref{co:irs_sd} is always rank-\num{1}. This conclusion is summarized below.

		\begin{proposition}\label{pr:relaxation}
			Any optimal solution $\boldsymbol{\Phi}^\star$ to the relaxed passive beamforming problem~\eqref{ob:irs}--\eqref{co:irs_sd} is rank-\num{1} such that \eqref{co:irs_rank} is tight and no loss is introduced by \gls{sdr}.
		\end{proposition}

		\begin{proof}\label{pf:relaxation}
			Please refer to Appendix~\ref{ap:relaxation}.
		\end{proof}

		In summary, we update $\boldsymbol{\Phi}^{(i)}$ until convergence, extract $\hat{\boldsymbol{\phi}}^\star$ by eigen decomposition, and retrieve the	\gls{ris} vector by $\boldsymbol{\phi}^{\star}=e^{j \arg\left([\hat{\boldsymbol{\phi}}^\star]_{(1:L)} \middle/ [\hat{\boldsymbol{\phi}}^\star]_{(L+1)}\right)}$. The passive beamforming design is summarized in the \gls{sca} Algorithm~\ref{al:sca}, where the relaxed problem \eqref{ob:irs}--\eqref{co:irs_sd} involves a $(L+1)$-order positive semi-definite matrix variable and $(L+2)$ linear constraints. Given a solution accuracy $\epsilon_{\mathrm{IPM}}$ for the interior-point method, the computational complexity of Algorithm~\ref{al:sca} is $\mathcal{O}\left(I_{\mathrm{\gls{sca}}}(L+2)^4 (L+1)^{0.5} \log(\epsilon_{\mathrm{IPM}}^{-1})\right)$, where $I_{\mathrm{\gls{sca}}}$ denotes the number of \gls{sca} iterations \cite{Luo2010b}.

		\begin{algorithm}[!t]
			\caption{\gls{sca}: \gls{ris} Phase Shift.}
			\label{al:sca}
			\begin{algorithmic}[1]
				\State \textbf{Input} $\beta_2$, $\beta_4$, $\boldsymbol{h}_{\mathrm{D},n}$, $\boldsymbol{V}_{n}$, $\sigma_n$, $\boldsymbol{w}_{\mathrm{I/P},n}$, $\rho$, $\bar{R}$, $\epsilon$, $\forall n$
				\State Construct $\boldsymbol{V}$, $\boldsymbol{M}$, $\boldsymbol{M}_n$, $\boldsymbol{C}_{n}$, $\boldsymbol{C}_{\mathrm{I/P},k}$, $\forall n,k$
				\State \textbf{Initialize} $i \gets 0$, $\boldsymbol{\Phi}^{(0)}$
				\State Set $t_{\mathrm{I/P},k}^{(0)}$, $\forall k$ by \eqref{eq:t_k}
				\State Compute $z^{(0)}$ by \eqref{eq:z_irs}
				\Repeat
					\State $i \gets i + 1$
					\State Get $\boldsymbol{\Phi}^{(i)}$ by solving \eqref{ob:irs}--\eqref{co:irs_sd}
					\State Update $t_{\mathrm{I/P},k}^{(i)}$, $\forall k$ by \eqref{eq:t_k}
					\State Compute $z^{(i)}$ by \eqref{eq:z_irs}
				\Until $\lvert z^{(i)}-z^{(i-1)} \rvert \le \epsilon$
				\State Set $\boldsymbol{\Phi}^{\star} \gets \boldsymbol{\Phi}^{(i)}$
				\State Get $\hat{\boldsymbol{\phi}}^\star$ by eigen decomposition, $\boldsymbol{\Phi}^{\star}=\hat{\boldsymbol{\phi}}^\star(\hat{\boldsymbol{\phi}}^\star)^H$
				\State Set $\boldsymbol{\phi}^{\star} \gets e^{j \arg\left([\hat{\boldsymbol{\phi}}^\star]_{(1:L)} \middle/ [\hat{\boldsymbol{\phi}}^\star]_{(L+1)}\right)}$
				\State \textbf{Output} $\boldsymbol{\phi}^{\star}$
			\end{algorithmic}
		\end{algorithm}

		\begin{proposition}\label{pr:sca}
			For any feasible initial point with given waveform and splitting ratio, the \gls{sca} Algorithm~\ref{al:sca} is guaranteed to converge to local optimal points of the original problem~\eqref{op:original}.
		\end{proposition}

		\begin{proof}\label{pf:sca}
			Please refer to Appendix~\ref{ap:sca}.
		\end{proof}
	\end{subsection}

	\begin{subsection}{Active Beamforming}
		The original waveform and active beamforming problem~\eqref{op:original} is over complex vectors $\boldsymbol{w}_{\mathrm{I/P}}$ of size $MN \times 1$. Next, we decouple the design in spatial and frequency domains, enable independent optimizations correspondingly, and reduce the size of variables from $2MN$ to $2(M+N)$. The weight on subband $n$ is essentially
		\begin{equation}\label{eq:w}
			\boldsymbol{w}_{\mathrm{I/P}, n} = s_{\mathrm{I/P}, n} \boldsymbol{b}_{\mathrm{I/P}, n},
		\end{equation}
		where $s_{\mathrm{I/P},n}$ denotes the amplitude of the modulated/multisine waveform at tone $n$, and $\boldsymbol{b}_{\mathrm{I/P}, n}$ denotes the corresponding information/power precoder. Define $\boldsymbol{s}_{\mathrm{I/P}} \triangleq [s_{\mathrm{I/P},1},\dots,s_{\mathrm{I/P},N}]^T \in \mathbb{R}_+^{N \times 1}$. The \gls{mrt} precoder at subband $n$ is given by
		\begin{equation}\label{eq:b_n}
			\boldsymbol{b}_{\mathrm{I/P}, n}^\star = \frac{\boldsymbol{h}_n}{\lVert{\boldsymbol{h}_n}\rVert}.
		\end{equation}

		\begin{proposition}\label{pr:mrt}
			For single-user \gls{swipt}, the global optimal information and power precoders coincide at the \gls{mrt}.
		\end{proposition}

		\begin{proof}\label{pf:mrt}
			Please refer to Appendix~\ref{ap:mrt}.
		\end{proof}
	\end{subsection}


	\begin{subsection}{Waveform and Splitting Ratio}
		Next, we jointly optimize the waveform amplitude $\boldsymbol{s}_{\mathrm{I/P}}$ and the splitting ratio $\rho$ for any given \gls{ris} phase shift $\boldsymbol{\phi}$ and active precoder $\boldsymbol{b}_{\mathrm{I/P},n}$, $\forall n$. On top of \eqref{eq:b_n}, the equivalent channel strength at subband $n$ is $\lVert{\boldsymbol{h}_n}\rVert$. Hence, the rate \eqref{eq:R} reduces to
		\begin{equation}\label{eq:R_waveform}
			R(\boldsymbol{s}_{\mathrm{I}},\rho) = \log_2\prod_{n=1}^N\left(1+\frac{(1-\rho)\lVert{\boldsymbol{h}_n}\rVert^2 s_{\mathrm{I},n}^2}{\sigma_n^2}\right),
		\end{equation}
		and the \gls{dc} \eqref{eq:z_expand} rewrites as \eqref{eq:z_waveform}, so that problem~\eqref{op:original} boils down to
		\begin{figure*}[!b]
			\hrule
			\begin{align}
				z(\boldsymbol{s}_{\mathrm{I}},\boldsymbol{s}_\mathrm{P},\rho)
				& = \frac{1}{2}{\beta_2}{\rho} \sum_{n=1}^N \lVert{\boldsymbol{h}_n}\rVert^2(s_{\mathrm{I},n}^2+s_{\mathrm{P},n}^2) + \frac{3}{8}{\beta_4}{\rho^2} \left( 2\sum_{n_1,n_2} \prod_{j=1}^2 \lVert{\boldsymbol{h}_{n_j}}\rVert^2 s_{\mathrm{I},{n_j}}^2 + \sum_{\substack{{n_1},{n_2},{n_3},{n_4}\\{n_1}+{n_2}={n_3}+{n_4}}} \prod_{j=1}^4 \lVert{\boldsymbol{h}_{n_j}}\rVert s_{\mathrm{P},{n_j}} \right)\nonumber\\
				& \quad + \frac{3}{2}{\beta_4}{\rho^2} \left( \sum_{n_1,n_2} \lVert{\boldsymbol{h}_{n_1}}\rVert^2 \lVert{\boldsymbol{h}_{n_2}}\rVert^2 s_{\mathrm{I},{n_1}}^2 s_{\mathrm{P},{n_2}}^2 \right).\label{eq:z_waveform}
			\end{align}
		\end{figure*}
		\begin{maxi!}
			{\scriptstyle{\boldsymbol{s}_{\mathrm{I}},\boldsymbol{s}_\mathrm{P},\rho}}{z(\boldsymbol{s}_{\mathrm{I}},\boldsymbol{s}_\mathrm{P},\rho)}{\label{op:waveform}}{}
			\addConstraint{R(\boldsymbol{s}_{\mathrm{I}},\rho) \ge \bar{R}}
			\addConstraint{\frac{1}{2}\left(\lVert{\boldsymbol{s}_{\mathrm{I}}}\rVert^2+\lVert{\boldsymbol{s}_\mathrm{P}}\rVert^2\right)\le{P}.}
		\end{maxi!}
		Following \cite{Clerckx2018b}, we introduce auxiliary variables $t'',\bar{\rho}$ and transform problem~\eqref{op:waveform} into a reversed \gls{gp}
		\begin{mini!}
			{\scriptstyle{\boldsymbol{s}_{\mathrm{I}},\boldsymbol{s}_\mathrm{P},\rho,\bar{\rho},t''}}{\frac{1}{t''}}{\label{op:waveform_rgp}}{}
			\addConstraint{\frac{t''}{z(\boldsymbol{s}_{\mathrm{I}},\boldsymbol{s}_\mathrm{P},\rho)} \le 1}\label{co:waveform_objective}
			\addConstraint{\frac{2^{\bar{R}}}{\prod_{n=1}^N \left(1+{\bar{\rho}\lVert{\boldsymbol{h}_n}\rVert^2 s_{\mathrm{I},n}^2}\big/{\sigma_n^2}\right)} \le 1}\label{co:waveform_rate}
			\addConstraint{\frac{1}{2}\left(\lVert{\boldsymbol{s}_{\mathrm{I}}}\rVert^2+\lVert{\boldsymbol{s}_\mathrm{P}}\rVert^2\right) \le P}\label{co:waveform_power}
			\addConstraint{\rho + \bar{\rho} \le 1.}\label{co:waveform_splitting_ratio}
		\end{mini!}
		It can be concluded that $\bar{\rho}^{\star}=1-\rho^{\star}$ as no power is wasted at the receiver. The denominators of \eqref{co:waveform_rate} and \eqref{co:waveform_objective} consist of posynomials \cite{Boyd2007} that can be decomposed as sums of monomials
		\begin{align}
			1+\frac{\bar{\rho}\lVert{\boldsymbol{h}_n}\rVert^2 s_{\mathrm{I},n}^2}{\sigma_n^2} &\triangleq \sum_{m_{\mathrm{I},n}}g_{m_{\mathrm{I},n}}(s_{\mathrm{I},n},\bar{\rho})\label{eq:g_I},\\
			z(\boldsymbol{s}_{\mathrm{I}},\boldsymbol{s}_\mathrm{P},\rho) &\triangleq \sum_{m_\mathrm{P}}{g_{m_\mathrm{P}}(\boldsymbol{s}_{\mathrm{I}},\boldsymbol{s}_\mathrm{P},\rho)}\label{eq:g_P}.
		\end{align}
		We upper bound \eqref{eq:g_I} and \eqref{eq:g_P} by the \gls{gm}-\gls{am} inequality \cite{Chiang2005} and transform problem~\eqref{op:waveform_rgp} to
		\begin{mini!}
			{\scriptstyle{\boldsymbol{s}_{\mathrm{I}},\boldsymbol{s}_\mathrm{P},\rho,\bar{\rho},t''}}{\frac{1}{t''}}{\label{op:waveform_gp}}{}
			\addConstraint{{t''}\prod_{m_\mathrm{P}}{\left(\frac{g_{{m_\mathrm{P}}}(\boldsymbol{s}_{\mathrm{I}},\boldsymbol{s}_\mathrm{P},\rho)}{\gamma_{{m_\mathrm{P}}}}\right)^{-\gamma_{{m_\mathrm{P}}}}}\le{1}}
			\addConstraint{2^{\bar{R}}\prod_{n}\prod_{m_{\mathrm{I},n}}\left(\frac{g_{m_{\mathrm{I},n}}(s_{\mathrm{I},n},\bar{\rho})}{\gamma_{m_{\mathrm{I},n}}}\right)^{-\gamma_{m_{\mathrm{I},n}}}\le{1}}
			\addConstraint{\frac{1}{2}\left(\lVert{\boldsymbol{s}_{\mathrm{I}}}\rVert^2+\lVert{\boldsymbol{s}_\mathrm{P}}\rVert^2\right)\le{P}}
			\addConstraint{\rho + \bar{\rho} \le 1,}
		\end{mini!}
		where $\gamma_{m_{\mathrm{I},n}},\gamma_{m_\mathrm{P}} \ge 0$ and $\sum_{m_{\mathrm{I},n}}\gamma_{m_{\mathrm{I},n}}=\sum_{m_\mathrm{P}}\gamma_{m_\mathrm{P}}=1$. The tightness of the AM-GM inequality depends on $\{\gamma_{m_{\mathrm{I},n}},\gamma_{m_\mathrm{P}}\}$, and a feasible choice at iteration $i$ is
		\begin{align}
			\gamma_{m_{\mathrm{I},n}}^{(i)} & = \frac{g_{m_{\mathrm{I},n}}(s_{\mathrm{I},n}^{(i-1)},\bar{\rho}^{(i-1)})}{1+{\bar{\rho}^{(i-1)}\lVert{\boldsymbol{h}_n}\rVert^2 (s_{\mathrm{I},n}^{(i-1)})^2}\big/{\sigma_n^2}}\label{eq:gamma_I},\\
			\gamma_{m_\mathrm{P}}^{(i)} & = \frac{g_{m_\mathrm{P}}(\boldsymbol{s}_{\mathrm{I}}^{(i-1)},\boldsymbol{s}_\mathrm{P}^{(i-1)},\rho^{(i-1)})}{z(\boldsymbol{s}_{\mathrm{I}}^{(i-1)},\boldsymbol{s}_\mathrm{P}^{(i-1)},\rho^{(i-1)})}\label{eq:gamma_P}.
		\end{align}
		With \eqref{eq:gamma_I} and \eqref{eq:gamma_P}, problem~\eqref{op:waveform_gp} can be solved by existing optimization tools such as \texttt{CVX toolbox} \cite{Grant2016}. We update $\boldsymbol{s}_{\mathrm{I}}^{(i)},\boldsymbol{s}_\mathrm{P}^{(i)},\rho^{(i)}$ iteratively until convergence. The joint waveform amplitude and splitting ratio design is summarized in the \gls{gp} Algorithm~\ref{al:gp}, which achieves local optimality at the cost of exponential computational complexity \cite{Chiang2005}.

		\begin{algorithm}[!t]
			\caption{\gls{gp}: Waveform Amplitude and Splitting Ratio.}
			\label{al:gp}
			\begin{algorithmic}[1]
				\State \textbf{Input} $\beta_2$, $\beta_4$, $\boldsymbol{h}_n$, $P$, $\sigma_n$, $\bar{R}$, $\epsilon$, $\forall n$
				\State \textbf{Initialize} $i \gets 0$, $\boldsymbol{s}_{\mathrm{I/P}}^{(0)}$, $\rho^{(0)}$
				\State Compute $R^{(0)}$, $z^{(0)}$ by \eqref{eq:R_waveform}, \eqref{eq:z_waveform}
				\State Set $g_{m_{\mathrm{I},n}}^{(0)}$, $g_{m_\mathrm{P}}^{(0)}$, $\forall n$ by \eqref{eq:g_I}, \eqref{eq:g_P}
				\Repeat
					\State $i \gets i + 1$
					\State Update $\gamma_{m_{\mathrm{I},n}}^{(i)}$, $\gamma_{m_\mathrm{P}}^{(i)}$, $\forall n$ by \eqref{eq:gamma_I}, \eqref{eq:gamma_P}
					\State Get $\boldsymbol{s}_{\mathrm{I/P}}^{(i)}$, $\rho^{(i)}$ by solving problem~\eqref{op:waveform_gp}
					\State Compute $R^{(i)}$, $z^{(i)}$ by \eqref{eq:R_waveform}, \eqref{eq:z_waveform}
					\State Update $g_{m_{\mathrm{I},n}}^{(i)}$, $g_{m_\mathrm{P}}^{(i)}$, $\forall n$ by \eqref{eq:g_I}, \eqref{eq:g_P}
				\Until $\lvert z^{(i)} - z^{(i-1)} \rvert \le \epsilon$
				\State Set $\boldsymbol{s}_{\mathrm{I/P}}^{\star} \gets \boldsymbol{s}_{\mathrm{I/P}}^{(i)}$, $\rho^{\star} \gets \rho^{(i)}$
				\State \textbf{Output} $\boldsymbol{s}_{\mathrm{I}}^{\star}$, $\boldsymbol{s}_{\mathrm{P}}^{\star}$, $\rho^{\star}$
			\end{algorithmic}
		\end{algorithm}

		\begin{proposition}\label{pr:gp}
			For any feasible initial point, the \gls{gp} Algorithm~\ref{al:gp} is guaranteed to converge to local optimal points of the waveform amplitude and splitting ratio design problem \eqref{op:waveform}.
		\end{proposition}

		\begin{proof}\label{pf:gp}
			Please refer to \cite{Clerckx2016a,Clerckx2018b}.
		\end{proof}
	\end{subsection}


	\begin{subsection}{Low-Complexity Adaptive Design}
		To facilitate practical \gls{swipt} implementation, we propose two closed-form adaptive waveform amplitude schemes by combining \gls{wf} and \gls{smf} in time and power domains, respectively. For \gls{wit}, the optimal \gls{wf} strategy assigns the amplitude of modulated tone $n$ by
		\begin{equation}\label{eq:wf}
			s_{\mathrm{I}, n} = \sqrt{2\left(\lambda - \frac{\sigma_n^2}{P \lVert{\boldsymbol{h}_n}\rVert^2}\right)^+},
		\end{equation}
		where $\lambda$ is chosen to satisfy the power constraint $\lVert{\boldsymbol{s}_I}\rVert^2 / 2 \le P$. The closed-form solution can be obtained by iterative power allocation \cite{Tse2005}, and the details are omitted here. On the other hand, \gls{smf} was proposed in \cite{Clerckx2017} as a suboptimal \gls{wpt} resource allocation scheme that assigns the amplitude of sinewave $n$ by
		\begin{equation}\label{eq:smf}
			s_{\mathrm{P}, n} = \sqrt{\frac{2 P}{\sum_{n=1}^N \lVert{\boldsymbol{h}_n \rVert^{2 \alpha}}}}\lVert{\boldsymbol{h}_n}\rVert^\alpha,
		\end{equation}
		where the scaling ratio $\alpha \ge 1$ is predetermined to exploit the rectifier nonlinearity and frequency selectivity. When the receiver works in \gls{ts} mode, there is no superposition in the suboptimal waveform design (modulated waveform with amplitude \eqref{eq:wf} is used in the data session while multisine waveform with amplitude \eqref{eq:smf} is used in the energy session). When the receiver works in \gls{ps} mode, we jointly design the combining ratio $\delta$ with the splitting ratio $\rho$, and assign the superposed waveform amplitudes as
		\begin{align}
			s_{\mathrm{I}, n} &= \sqrt{2(1 - \delta)\left(\lambda - \frac{\sigma_n^2}{P \lVert{\boldsymbol{h}_n}\rVert^2}\right)^+}, \label{eq:s_i}\\
			s_{\mathrm{P}, n} &= \sqrt{\frac{2 \delta P}{\sum_{n=1}^N \lVert{\boldsymbol{h}_n \rVert^{2 \alpha}}}}\lVert{\boldsymbol{h}_n}\rVert^\alpha, \label{eq:s_p}
		\end{align}
		where the $\delta$ determines the power ratio of multisine waveform at the transmitter, and $\rho$ determines the power ratio of the energy harvester at the receiver.\footnote{We notice that $\delta^{\star}=\rho^{\star}=0$ at the \gls{wit} point and $\delta^{\star}=\rho^{\star}=1$ at the \gls{wpt} point when $N$ is relatively large. Intuitively, $\delta^{\star}$ and $\rho^{\star}$ should be positively correlated for efficient \gls{swipt} design.}

		Besides, minor modifications are required for passive beamforming to accommodate the low-complexity waveform schemes. Specifically, the rate constraint \eqref{co:irs_rate} should be dropped as the achievable rate is controlled by $\eta$ or $\{\delta,\rho\}$. To achieve the \gls{wit} point ($\rho=0$), the rate \eqref{eq:R_irs} should be maximized, the current expression \eqref{eq:z_irs_approx} is not needed and no \gls{sca} is involved. The Modified-\gls{sca} (M-\gls{sca}) Algorithm~\ref{al:m_sca} summarizes the modified passive beamforming design when the receiver works in \gls{ps} mode. Similarly, no loss is introduced by \gls{sdr} and local optimality is guaranteed. The proofs are omitted here. Since each \gls{sdp} involves $(L+1)$ linear constraints, the computational complexity of Algorithm~\ref{al:m_sca} is $\mathcal{O}\left(I_{\mathrm{M-\gls{sca}}}(L+1)^{4.5} \log(\epsilon_{\mathrm{IPM}}^{-1})\right)$, where $I_{\mathrm{M-\gls{sca}}}$ denotes the number of M-\gls{sca} iterations \cite{Luo2010b}. Note that no \gls{sca} is involved at the \gls{wit} point where $I_{\mathrm{M-\gls{sca}}}=1$.

		\begin{algorithm}[!t]
			\caption{M-\gls{sca}: \gls{ris} Phase Shift.}
			\label{al:m_sca}
			\begin{algorithmic}[1]
				\State \textbf{Input} $\beta_2$, $\beta_4$, $\boldsymbol{h}_{\mathrm{D},n}$, $\boldsymbol{V}_{n}$, $\sigma_n$, $\boldsymbol{w}_{\mathrm{I/P},n}$, $\rho$, $\epsilon$, $\forall n$
				\State Construct $\boldsymbol{V}$, $\boldsymbol{M}$, $\boldsymbol{M}_n$, $\boldsymbol{C}_{n}$, $\boldsymbol{C}_{\mathrm{I/P},k}$, $\forall n,k$
				\State \textbf{Initialize} $i \gets 0$, $\boldsymbol{\Phi}^{(0)}$
				\If{$\rho=0$}
					\State Get $\boldsymbol{\Phi}^{\star}$ by maximizing \eqref{eq:R_irs} s.t. \eqref{co:irs_modulus}, \eqref{co:irs_sd}
				\Else
					\State Set $t_{\mathrm{I/P},k}^{(0)}$, $\forall k$ by \eqref{eq:t_k}
					\State Compute $z^{(0)}$ by \eqref{eq:z_irs}
					\Repeat
						\State $i \gets i + 1$
							\State Get $\boldsymbol{\Phi}^{(i)}$ by maximizing \eqref{eq:z_irs_approx} s.t. \eqref{co:irs_modulus}, \eqref{co:irs_sd}
							\State Update $t_{\mathrm{I/P},k}^{(i)}$, $\forall k$ by \eqref{eq:t_k}
							\State Compute $z^{(i)}$ by \eqref{eq:z_irs}
					\Until $\lvert z^{(i)}-z^{(i-1)} \rvert \le \epsilon$
					\State Set $\boldsymbol{\Phi}^{\star} \gets \boldsymbol{\Phi}^{(i)}$
				\EndIf
				\State Get $\hat{\boldsymbol{\phi}}^\star$ by eigen decomposition, $\boldsymbol{\Phi}^{\star}=\hat{\boldsymbol{\phi}}^\star(\hat{\boldsymbol{\phi}}^\star)^H$
				\State Set $\boldsymbol{\phi}^{\star} \gets e^{j \arg\left([\hat{\boldsymbol{\phi}}^\star]_{(1:L)} \middle/ [\hat{\boldsymbol{\phi}}^\star]_{(L+1)}\right)}$
				\State \textbf{Output} $\boldsymbol{\phi}^{\star}$
			\end{algorithmic}
		\end{algorithm}
	\end{subsection}


	\begin{subsection}{Block Coordinate Descent}
		Based on the direct and cascaded \gls{csit}, we iteratively update the passive beamforming $\boldsymbol{\phi}$ by Algorithm~\ref{al:sca}, the active precoder $\boldsymbol{b}_{\mathrm{I/P},n}$, $\forall n$ by equation \eqref{eq:b_n}, and the waveform amplitude $\boldsymbol{s}_{\mathrm{I/P}}$ and splitting ratio $\rho$ by Algorithm~\ref{al:gp}, until convergence. The steps are summarized in the \gls{bcd} Algorithm~\ref{al:bcd}, whose computational complexity is exponential as inherited from Algorithm~\ref{al:gp}. It is guaranteed to converge, but may end up with a suboptimal solution because variables are coupled in constraint~\eqref{co:original_rate} \cite{Grippo2000}.

		\begin{algorithm}[!t]
			\caption{\gls{bcd}: Waveform, Beamforming and Splitting Ratio.}
			\label{al:bcd}
			\begin{algorithmic}[1]
				\State \textbf{Input} $\beta_2$, $\beta_4$, $\boldsymbol{h}_{\mathrm{D},n}$, $\boldsymbol{V}_{n}$, $P$, $\sigma_n$, $\bar{R}$, $\epsilon$, $\forall n$
				\State \textbf{Initialize} $i \gets 0$, $\boldsymbol{\phi}^{(0)}$, $\boldsymbol{b}_{\mathrm{I/P},n}^{(0)}$, $\boldsymbol{s}_{\mathrm{I/P}}^{(0)}$, $\rho^{(0)}$, $\forall n$
				\State Set $\boldsymbol{w}_{\mathrm{I/P},n}^{(0)}$, $\forall n$ by \eqref{eq:w}
				\State Compute $z^{(0)}$ by \eqref{eq:z_waveform}
				\Repeat
					\State $i \gets i + 1$
					\State Get $\boldsymbol{\phi}^{(i)}$ based on $\boldsymbol{w}_{\mathrm{I/P}}^{(i-1)}$, $\rho^{(i-1)}$ by Algorithm~\ref{al:sca}
					\State Update $\boldsymbol{h}_n^{(i)}$, $\boldsymbol{b}_n^{(i)}$, $\forall n$ by \eqref{eq:h_n}, \eqref{eq:b_n}
					\State Get $\boldsymbol{s}_{\mathrm{I/P}}^{(i)}$, $\rho^{(i)}$ by Algorithm~\ref{al:gp}
					\State Update $\boldsymbol{w}_{\mathrm{I/P},n}^{(i)}$, $\forall n$ by \eqref{eq:w}
					\State Compute $z^{(i)}$ by \eqref{eq:z_waveform}
				\Until $\lvert z^{(i)} - z^{(i-1)} \rvert \le \epsilon$
				\State Set $\boldsymbol{\phi}^{\star} \gets \boldsymbol{\phi}^{(i)}$, $\boldsymbol{w}_{\mathrm{I/P}}^{\star} \gets \boldsymbol{w}_{\mathrm{I/P}}^{(i)}$, $\rho^{\star} \gets \rho^{(i)}$
				\State \textbf{Output} $\boldsymbol{\phi}^{\star}$, $\boldsymbol{w}_{\mathrm{I}}^{\star}$, $\boldsymbol{w}_{\mathrm{P}}^{\star}$, $\rho^{\star}$
			\end{algorithmic}
		\end{algorithm}

		For the low-complexity design under \gls{ps} mode, we obtain the phase shift by Algorithm~\ref{al:m_sca}, the active precoder $\boldsymbol{b}_{\mathrm{I/P},n}$, $\forall n$ by equation \eqref{eq:b_n}, and the waveform amplitude by \eqref{eq:s_i} and \eqref{eq:s_p}. To achieve the \gls{wit} point ($\rho=0$), the rate \eqref{eq:R_irs} should be maximized to obtain the maximum capacity $C_{\max}$.\footnote{Recall in Remark~\ref{re:subband_tradeoff} that different subchannel designs lead to different capacities.} Note that the \gls{bcd} algorithm obtains the \gls{r-e} region by varying the rate constraint from \num{0} to $C_{\max}$, while the achievable \gls{r-e} region of the \gls{lc}-\gls{bcd} algorithm can be obtained by performing a two-dimensional search over $(\delta, \rho)$ from $(0, 0)$ to $(1, 1)$. The steps are summarized in Algorithm~\ref{al:lc_bcd}. The computational complexity of Algorithm~\ref{al:lc_bcd} is $\mathcal{O}\left(I_{\mathrm{\gls{lc}-\gls{bcd}}}I_{\mathrm{M-\gls{sca}}}(L+1)^{4.5} \log(\epsilon_{\mathrm{IPM}}^{-1})\right)$, where $I_{\mathrm{\gls{lc}-\gls{bcd}}}$ denotes the number of \gls{lc}-\gls{bcd} iterations \cite{Luo2010b}.

		\begin{algorithm}[!t]
			\caption{\gls{lc}-\gls{bcd}: Waveform and Beamforming.}
			\label{al:lc_bcd}
			\begin{algorithmic}[1]
				\State \textbf{Input} $\beta_2$, $\beta_4$, $\boldsymbol{h}_{\mathrm{D},n}$, $\boldsymbol{V}_{n}$, $P$, $\sigma_n$, $\delta$, $\rho$, $\epsilon$, $\forall n$
				\State \textbf{Initialize} $i \gets 0$, $\boldsymbol{\phi}^{(0)}$, $\boldsymbol{b}_{\mathrm{I/P},n}^{(0)}$, $\boldsymbol{s}_{\mathrm{I/P}}^{(0)}$, $\forall n$
				\State Set $\boldsymbol{w}_{\mathrm{I/P},n}^{(0)}$, $\forall n$ by \eqref{eq:w}
				\State Compute $R^{(0)}$, $z^{(0)}$ by \eqref{eq:R_waveform}, \eqref{eq:z_waveform}
				\Repeat
					\State $i \gets i + 1$
					\State Get $\boldsymbol{\phi}^{(i)}$ based on $\boldsymbol{w}_{\mathrm{I/P}}^{(i-1)}$ by Algorithm~\ref{al:m_sca}
					\State Update $\boldsymbol{h}_n^{(i)}$, $\boldsymbol{b}_n^{(i)}$, $\forall n$ by \eqref{eq:h_n}, \eqref{eq:b_n}
					\State Update $\boldsymbol{s}_{\mathrm{I}}^{(i)}$, $\boldsymbol{s}_{\mathrm{P}}^{(i)}$ by \eqref{eq:s_i}, \eqref{eq:s_p}
					\State Update $\boldsymbol{w}_{\mathrm{I/P},n}^{(i)}$, $\forall n$ by \eqref{eq:w}
					\State Compute $R^{(i)}$, $z^{(i)}$ by \eqref{eq:R_waveform}, \eqref{eq:z_waveform}
					\If{$\rho=0$}
						\State $\Delta \gets R^{(i)} - R^{(i-1)}$
					\Else
						\State $\Delta \gets z^{(i)} - z^{(i-1)}$
					\EndIf
				\Until $\lvert \Delta \rvert \le \epsilon$
				\State Set $\boldsymbol{\phi}^{\star} \gets \boldsymbol{\phi}^{(i)}$, $\boldsymbol{w}_{\mathrm{I/P}}^{\star} \gets \boldsymbol{w}_{\mathrm{I/P}}^{(i)}$
				\State \textbf{Output} $\boldsymbol{\phi}^{\star}$, $\boldsymbol{w}_{\mathrm{I}}^{\star}$, $\boldsymbol{w}_{\mathrm{P}}^{\star}$
			\end{algorithmic}
		\end{algorithm}
	\end{subsection}
\end{section}


\begin{section}{Performance Evaluations}\label{se:performance_evaluation}
	\begin{figure}[!t]
		\centering
		\def\svgwidth{0.9\columnwidth}
		% \import{assets/}{layout.eps_tex}
		\input{assets/chapter_1/layout.eps_tex}
		\caption{System layout in simulation.}
		\label{fi:layout}
	\end{figure}

	To evaluate the proposed \gls{ris}-aided \gls{swipt} system, we consider the layout in Fig.~\ref{fi:layout} where the \gls{ris} moves along a line parallel to the \gls{ap}-\gls{ue} path. Let $d_{\mathrm{H}}$, $d_{\mathrm{V}}$ be the horizontal and vertical distances from the \gls{ap} to the \gls{ris}, and denote respectively $d_{\mathrm{D}}$, $d_{\mathrm{I}}=\sqrt{d_{\mathrm{H}}^2+d_{\mathrm{V}}^2}$, $d_{\mathrm{R}}=\sqrt{(d_{\mathrm{D}}-d_{\mathrm{H}})^2+d_{\mathrm{V}}^2}$ as the distance of direct, incident and reflected links. $d_{\mathrm{D}}=\qty{12}{\meter}$ and $d_{\mathrm{H}}=d_{\mathrm{V}}=\qty{2}{\meter}$ are chosen as reference. The path loss of direct, incident and reflected links are denoted by $\Lambda_{\mathrm{D}}$, $\Lambda_{\mathrm{I}}$ and $\Lambda_{\mathrm{R}}$, respectively. We consider a large open space Wi-Fi-like environment at center frequency \qty{2.4}{\GHz} where the channel follows IEEE TGn channel model D \cite{Erceg2004}. Specifically, the path loss exponent is \num{2} (i.e., free-space model) up to \qty{10}{\meter}, and \num{3.5} onwards to further penalize the channels with large distance. All fadings are modeled as \gls{nlos} with tap delays and powers specified in model D, and the tap gains are modeled as i.i.d. \gls{cscg} variables. Rectenna parameters are set to $k_2=0.0034$, $k_4=0.3829$, $R_{\mathrm{A}}=\qty{50}{\ohm}$ \cite{Clerckx2016a} such that $\beta_2=0.17$ and $\beta_4=957.25$. We also choose the average \gls{eirp} as $P=\qty{40}{dBm}$, the receive antenna gain as \qty{3}{dBi}, the scaling ratio as $\alpha=2$, and the tolerance as $\epsilon=10^{-8}$. To further reduce the complexity, we assume $\delta=\rho$ for simplicity and perform a one-dimensional search from \num{0} to \num{1} to obtain a inner \gls{r-e} bound for the \gls{lc}-\gls{bcd} algorithm. Each \gls{r-e} point is averaged over \num{200} channel realizations, and the $x$-axis is normalized to per-subband rate $R/N$.

	\begin{figure}[!t]
		\centering
		\resizebox{0.8\columnwidth}{!}{
			% \includegraphics{assets/channel_amplitude.eps}
			\input{assets/chapter_1/channel_amplitude.tex}
		}
		\caption{Sorted equivalent subchannel amplitude with and without \gls{ris} versus $N$ for $M=1$, $L=100$, $\sigma_n^2=\qty{-40}{dBm}$, $B=\qty{10}{\MHz}$ and $d_{\mathrm{H}}=d_{\mathrm{V}}=\qty{2}{\meter}$.}
		\label{fi:channel_amplitude}
	\end{figure}

	Fig.~\ref{fi:channel_amplitude} reveals how \gls{ris} influences the sorted equivalent subchannel amplitude for one channel realization. Due to the flexible subchannel design enabled by passive beamforming, the optimal amplitude distribution for \gls{wit} and \gls{wpt} are dissimilar. Under the specified configuration, the \gls{wpt}-optimized \gls{ris} aligns the strong subbands to exploit the rectifier nonlinearity. On the other hand, the \gls{wit}-optimized \gls{ris} provides a fair gain over all subchannels when $L$ is sufficiently large. This is reminiscent of the \gls{wf} scheme at high \gls{snr}, but is realized by channel alignment instead of resource allocation. Nevertheless, the amplitude of modulated and multisine waveforms remains approximately unchanged when adding a \gls{ris} (the plots are not attached). In other words, \gls{ris} has a subtle impact on the waveform design.

	\begin{figure}[!t]
		\centering
		\subfloat[\gls{r-e} region\label{fi:re_subband}]{
			\resizebox{0.45\columnwidth}{!}{
				% \includegraphics{assets/re_subband.eps}
				\input{assets/chapter_1/re_subband.tex}
			}
		}
		\subfloat[\gls{wpt} waveform amplitude\label{fi:waveform_subband}]{
			\resizebox{0.45\columnwidth}{!}{
				% \includegraphics{assets/waveform_subband.eps}
				\input{assets/chapter_1/waveform_subband.tex}
			}
		}
		\caption{Average \gls{r-e} region and \gls{wpt} waveform amplitude versus $N$ for $M=1$, $L=20$, $\sigma_n^2=\qty{-40}{dBm}$, $B=\qty{1}{\MHz}$ and $d_{\mathrm{H}}=d_{\mathrm{V}}=\qty{2}{\meter}$.}
	\end{figure}

	Fig.~\subref*{fi:re_subband} illustrates the average \gls{r-e} region versus the number of subband $N$. First, it is observed that increasing $N$ reduces the per-subband rate but boosts the harvested energy. This is because less power is allocated to each subband but more balanced \gls{dc} terms are introduced by frequency coupling to boost the harvested energy. On the other hand, Fig.~\subref*{fi:waveform_subband} presents the sorted modulated/multisine amplitude $\boldsymbol{s}_{\mathrm{I/P}}$ for \gls{wpt}. It demonstrates that a dedicated multisine waveform is unnecessary for a small $N$ but is required for a large $N$. This observation origins from the rectifier nonlinearity. Although both waveforms have equivalent second-order \gls{dc} terms \eqref{eq:y_I2} and \eqref{eq:y_P2}, for the fourth-order terms \eqref{eq:y_I4} and \eqref{eq:y_P4}, the modulated waveform has $N^2$ monomials with a modulation gain of \num{2}, while the multisine has $(2N^3+N)/3$ monomials as the components of different frequencies compensate and produce \gls{dc}. Second, the \gls{r-e} region is convex for $N \in \{2,4\}$ and concave-convex for $N \in \{8,16\}$, such that \gls{ps} outperforms \gls{ts} for a small $N$ and is outperformed for a large $N$. When $N$ is in between, the optimal strategy is a combination of both, i.e., a time sharing between the \gls{wpt} point and the saddle \gls{ps} \gls{swipt} point (as denoted by the red curve in Fig.~\subref*{fi:re_subband}). When $N$ is relatively small, only modulated waveform is used at both \gls{wit} and \gls{wpt} points, and one can infer that no multisine waveform is needed for the entire \gls{r-e} region. It aligns with the conclusion based on the conventional linear harvester model, namely the \gls{r-e} region is convex, \gls{ps} outperforms \gls{ts}, and dedicated power waveform is unnecessary. As $N$ becomes sufficiently large, the multisine waveform further boosts \gls{wpt} and creates some concavity in the high-power region, which accounts for the superiority of \gls{ts} under the nonlinear harvester model. Therefore, we conclude that the rectifier nonlinearity enlarges the \gls{r-e} region by favoring a different waveform and receiving mode, both heavily depending on $N$.

	\begin{figure}[!t]
		\centering
		\subfloat[\gls{r-e} region\label{fi:re_noise}]{
			\resizebox{0.45\columnwidth}{!}{
				% \includegraphics{assets/re_noise.eps}
				\input{assets/chapter_1/re_noise.tex}
			}
		}
		\subfloat[Splitting ratio\label{fi:splitting_ratio_noise}]{
			\resizebox{0.45\columnwidth}{!}{
				% \includegraphics{assets/splitting_ratio_noise.eps}
				\input{assets/chapter_1/splitting_ratio_noise.tex}
			}
		}
		\caption{Average \gls{r-e} region and splitting ratio versus $\sigma_n^2$ for $M=1$, $N=16$, $L=20$, $B=\qty{1}{\MHz}$ and $d_{\mathrm{H}}=d_{\mathrm{V}}=\qty{2}{\meter}$.}
	\end{figure}

	The average noise power influences the \gls{r-e} region as shown in Fig.~\subref*{fi:re_noise}. First, we note that the \gls{r-e} region is roughly concave/convex at low/high \gls{snr} such that \gls{ts}/\gls{ps} are preferred correspondingly. At low \gls{snr}, the power is allocated to the modulated waveform on a few strongest subbands to achieve a high rate. As the rate constraint $\bar{R}$ decreases, Algorithm~\ref{al:gp} activates more subbands that further boosts the harvested \gls{dc} power because of frequency coupling and harvester nonlinearity. Second, there exists a turning point in the \gls{r-e} region, especially for a low noise level ($\sigma_n^2 \le \qty{-40}{dBm}$). The reason is that when $\bar{R}$ departs slightly from the maximum value, the algorithm tends to adjust the splitting ratio $\rho$ rather than allocate more power to the multisine waveform, since a small amplitude multisine could be inefficient for energy purpose. As $\bar{R}$ further decreases, thanks to the advantage of multisine, a superposed waveform with a small $\rho$ can outperform a modulated waveform with a large $\rho$. The result proves the benefit of superposed waveform and the necessity of joint waveform and splitting ratio optimization. Besides, the \gls{lc}-\gls{bcd} algorithm achieves a good balance between performance and complexity even if one-dimensional search is considered for $\delta=\rho$ from \num{0} to \num{1}.

	\begin{figure}[!t]
		\centering
		\subfloat[\gls{r-e} region\label{fi:re_distance}]{
			\resizebox{0.45\columnwidth}{!}{
				% \includegraphics{assets/re_distance.eps}
				\input{assets/chapter_1/re_distance.tex}
			}
		}
		\subfloat[Path loss product\label{fi:path_loss}]{
			\resizebox{0.45\columnwidth}{!}{
				% \includegraphics{assets/path_loss.eps}
				\input{assets/chapter_1/path_loss.tex}
			}
		}
		\caption{Average \gls{r-e} region and path loss versus $d_{\mathrm{H}}$ for $M=1$, $N=16$, $L=20$, $\sigma_n^2=\qty{-40}{dBm}$, $B=\qty{1}{\MHz}$ and $d_{\mathrm{V}}=\qty{2}{\meter}$.}
	\end{figure}

	In Fig.~\subref*{fi:re_distance}, we compare the average \gls{r-e} region achieved by different \gls{ap}-\gls{ris} horizontal distance $d_{\mathrm{H}}$. Different from the active Amplify-and-Forward (AF) relay that favors midpoint development \cite{Li2017}, the \gls{ris} should be placed close to either the \gls{ap} or the \gls{ue} based on the product path loss model that applies to finite-size element reflection \cite{Ozdogan2020,Tang2021}. Moreover, there exist two optimal \gls{ris} coordinates around $d_{\mathrm{H}}=0.6$ and \qty{11.4}{\meter} that minimize the path loss product $\Lambda_{\mathrm{I}}\Lambda_R$ and maximize the \gls{r-e} tradeoff. It suggests that equipping the \gls{ap} with a \gls{ris} can potentially extend the operation range of \gls{swipt} systems.

	\begin{figure}[!t]
		\centering
		\subfloat[\gls{r-e} region\label{fi:re_tx}]{
			\resizebox{0.45\columnwidth}{!}{
				% \includegraphics{assets/re_tx.eps}
				\input{assets/chapter_1/re_tx.tex}
			}
		}
		\subfloat[\gls{wit} \gls{snr} and \gls{wpt} \gls{dc}\label{fi:scaling_tx}]{
			\resizebox{0.45\columnwidth}{!}{
				% \includegraphics{assets/scaling_tx.eps}
				\input{assets/chapter_1/scaling_tx.tex}
			}
		}
		\caption{Average \gls{r-e} region, \gls{wit} \gls{snr} and \gls{wpt} \gls{dc} versus $M$ for $N=16$, $L=20$, $\sigma_n^2=\qty{-40}{dBm}$, $B=\qty{1}{\MHz}$, $d_{\mathrm{H}}=d_{\mathrm{V}}=\qty{0.2}{\meter}$.}
	\end{figure}

	\begin{figure}[!t]
		\centering
		\subfloat[\gls{r-e} region\label{fi:re_reflector}]{
			\resizebox{0.425\columnwidth}{!}{
				% \includegraphics{assets/re_reflector.eps}
				\input{assets/chapter_1/re_reflector.tex}
			}
		}
		\subfloat[\gls{wit} \gls{snr} and \gls{wpt} \gls{dc}\label{fi:scaling_reflector}]{
			\resizebox{0.475\columnwidth}{!}{
				% \includegraphics{assets/scaling_reflector.eps}
				\input{assets/chapter_1/scaling_reflector.tex}
			}
		}
		\caption{Average \gls{r-e} region, \gls{wit} \gls{snr} and \gls{wpt} \gls{dc} versus $L$ for $M=1$, $N=16$, $\sigma_n^2=\qty{-40}{dBm}$, $B=\qty{1}{\MHz}$ and $d_{\mathrm{H}}=d_{\mathrm{V}}=\qty{0.2}{\meter}$.}
	\end{figure}

	The impacts of the number of transmit antennas $M$ and the \gls{ris} elements $L$ on the \gls{r-e} behavior are revealed in Figs.~\subref*{fi:re_tx} and \subref*{fi:re_reflector}. First, it is observed that adding either active or passive elements can improve the equivalent \gls{snr}, which produces a nearly concave \gls{r-e} region and favors the \gls{ps} receiver. Second, the conventional Linear Energy Harvester (LEH) model leads to a power-inefficient design. To investigate the performance loss, we truncate the \gls{dc} objective function \eqref{eq:z} at $n_0=2$ such that (i) in the passive beamforming problem, $z(\boldsymbol{\Phi}) = {\beta_2}{\rho}(t_{\mathrm{I},0}+t_{\mathrm{P},0})/2$ and no \gls{sca} is required; (ii) in the waveform design problem, the \gls{wpt}-optimal strategy is the adaptive single sinewave that allocates all power to the multisine at the strongest subband \cite{Clerckx2016a}. As shown in Figs.~\subref*{fi:scaling_tx} and \subref*{fi:scaling_reflector}, those conventional designs do not exploit the harvester nonlinearity and end up with a nearly \qty{20}{dBA} gap compared to the nonlinear model-based \gls{smf} and \gls{gp} designs. Third, doubling $M$ brings a \qty{3}{\dB} gain at the output \gls{snr} and a \qty{12}{dBA} increase at the harvested \gls{dc}, which verified that active beamforming has an array gain of $M$ \cite{Tse2005} with power scaling order $M^2$ under the truncated nonlinear harvester model \cite{Clerckx2016a,Clerckx2018b}. Fourth, when the \gls{ris} is very close to the \gls{ap} or \gls{ue}, doubling $L$ can bring a \qty{6}{\dB} gain at the output \gls{snr} and a \qty{24}{dBA} increase at the harvested \gls{dc}. From the perspective of \gls{wit}, it suggests that passive beamforming can reach an array gain of $L^2$, as indicated by \cite{Wu2019}. An interpretation is that the \gls{ris} coherently combines the incoming signal with a receive array gain $L$, then performs an equal gain reflection with a transmit array gain $L$. From the perspective of \gls{wpt}, it suggests that passive beamforming comes with a power scaling order $L^4$ under the truncated nonlinear harvester model. We then verify this novel observation in a simplified case where the power is uniformly allocated over multisine, all channels are frequency-flat, and $L$ is sufficiently large such that the direct channel becomes negligible. Let $X$ be the cascaded small-scale fading coefficient. The \gls{dc} in such case reduces to
	\begin{equation}
		z = \beta_2 \Lambda_{\mathrm{R}}^2 \Lambda_{\mathrm{I}}^2 \lvert X \rvert^2 L^2 P + \beta_4 \frac{2N^2 + 1}{2N} \Lambda_{\mathrm{R}}^4 \Lambda_{\mathrm{I}}^4 \lvert X \rvert^4 L^4 P^2,
	\end{equation}
	which scales quartically with $L$. Compared with active antennas, \gls{ris} elements achieve higher array gain and power scaling order, but a very large $L$ is required to compensate the double fading of the auxiliary link. These observations demonstrate the \gls{r-e} benefit of passive beamforming and emphasize the importance of accounting for the harvester nonlinearity in the waveform and beamforming design.

	\begin{figure}[!t]
		\centering
		\subfloat[$B=\qty{1}{\MHz}$\label{fi:re_irs_1mhz}]{
			\resizebox{0.45\columnwidth}{!}{
				% \includegraphics{assets/re_irs_1mhz.eps}
				\input{assets/chapter_1/re_irs_1mhz.tex}
			}
		}
		\subfloat[$B=\qty{10}{\MHz}$\label{fi:re_irs_10mhz}]{
			\resizebox{0.45\columnwidth}{!}{
				% \includegraphics{assets/re_irs_10mhz.eps}
				\input{assets/chapter_1/re_irs_10mhz.tex}
			}
		}
		\caption{Average \gls{r-e} region for ideal, adaptive, fixed and no \gls{ris} versus $B$ for $M=1$, $N=16$, $L=20$, $\sigma_n^2=\qty{-40}{dBm}$ and $d_{\mathrm{H}}=d_{\mathrm{V}}=\qty{2}{\meter}$.}
	\end{figure}

	Figs.~\subref*{fi:re_irs_1mhz} and \subref*{fi:re_irs_10mhz} explore the \gls{r-e} region with different \gls{ris} strategies for narrowband and broadband \gls{swipt}. The ideal Frequency-Selective (FS) \gls{ris} assumes the reflection coefficient of each element is independent and controllable at different frequencies. The adaptive \gls{ris} adjusts the passive beamforming for different \gls{r-e} points by Algorithm~\ref{al:sca}. The \gls{wit}/\gls{wpt}-optimized \gls{ris} is retrieved by Algorithm~\ref{al:m_sca} then fixed for the whole \gls{r-e} region. The random \gls{ris} models the phase shift of all elements as i.i.d. uniform random variables over $[0, 2\pi)$. First, random \gls{ris} and no \gls{ris} perform worse than other schemes since no passive beamforming is exploited. Their \gls{r-e} boundaries coincide as the antenna mode reflection of the random \gls{ris} is canceled out after averaging. Second, when the bandwidth is small, the performance of ideal, adaptive, and \gls{wit}/\gls{wpt}-optimized \gls{ris} are similar; when the bandwidth is large, the adaptive \gls{ris} outperforms the \gls{wit}/\gls{wpt}-optimized \gls{ris} but is outperformed by the ideal FS \gls{ris}. In the former case, the subband responses are close to each other such that the tradeoff in Remark~\ref{re:subband_tradeoff} becomes insignificant, and the auxiliary link can be roughly maximized at all subbands. It suggests that for narrowband \gls{swipt}, the optimal passive beamforming for any \gls{r-e} point is optimal for the whole \gls{r-e} region, and the corresponding composite channel and active precoder are also optimal for the whole \gls{r-e} region. Hence, the achievable \gls{r-e} region is obtained by optimizing the waveform amplitude and splitting ratio. On the other hand, since the channel frequency selectivity affects the performance of the information decoder and energy harvester differently, the optimal \gls{ris} reflection coefficient varies at different \gls{r-e} tradeoffs points for broadband \gls{swipt}. As shown in Fig.~\ref{fi:channel_amplitude}, the subchannel amplification can be either spread evenly to improve the rate at high \gls{snr}, or focused on a few strongest subbands to boost the output \gls{dc}, thanks to adaptive passive beamforming.

	\begin{figure}[!t]
		\centering
		\subfloat[Imperfect cascaded \gls{csit}\label{fi:re_csi}]{
			\resizebox{0.45\columnwidth}{!}{
				% \includegraphics{assets/re_csi.eps}
				\input{assets/chapter_1/re_csi.tex}
			}
		}
		\subfloat[Quantized \gls{ris}\label{fi:re_quantization}]{
			\resizebox{0.45\columnwidth}{!}{
				% \includegraphics{assets/re_quantization.eps}
				\input{assets/chapter_1/re_quantization.tex}
			}
		}
		\caption{Average \gls{r-e} region with imperfect cascaded \gls{csit} and quantized \gls{ris} for $M=1$, $N=16$, $L=20$, $\sigma_n^2=\qty{-40}{dBm}$, $B=\qty{10}{\MHz}$ and $d_{\mathrm{H}}=d_{\mathrm{V}}=\qty{2}{\meter}$. $\epsilon_{n}=0$ and $\epsilon_{n}=\infty$ correspond respectively to perfect \gls{csit} and no \gls{csit} (and random \gls{ris}); $b=0$ and $b \to \infty$ correspond respectively to no \gls{ris} and continuous \gls{ris}.}
	\end{figure}

	We then explore the impacts of imperfect cascaded \gls{csit} and quantized \gls{ris} on the \gls{r-e} performance. Due to the general lack of \gls{rf}-chains at the \gls{ris}, it can be challenging to acquire accurate cascaded \gls{csit} on a short-term basis. We assume the cascaded channel at subband $n$ is
	\begin{equation}
		\boldsymbol{V}_{n} = \hat{\boldsymbol{V}}_{n} + \tilde{\boldsymbol{V}}_{n},
	\end{equation}
	where $\hat{\boldsymbol{V}}_{n}$ is the estimated cascaded \gls{csit} and $\tilde{\boldsymbol{V}}_{n}$ is the estimation error with entries following i.i.d. \gls{cscg} distribution $\mathcal{CN}(0, \epsilon_{n}^2)$.\footnote{Note that the subchannel responses are correlated but the estimations can be independent.} Figure~\subref*{fi:re_csi} shows that the proposed passive beamforming Algorithm~\ref{al:sca} is robust to cascaded \gls{csit} inaccuracy for broadband \gls{swipt} with different $L$. On the other hand, since the practical reflection coefficient depends on the available element impedances, we consider a discrete \gls{ris} codebook $\mathcal{C}_\phi = \{e^{j 2 \pi i / 2^b} \mid i = 1, \dots, 2^b\}$ and uniformly quantize the continuous reflection coefficients obtained by Algorithm~\ref{al:bcd} to reduce the circuit complexity and control overhead.\footnote{This relax-then-quantize approach can bring notable performance loss compared with direct optimization over the discrete phase shift set, especially for a small $b$ (i.e., low-resolution \gls{ris}) \cite{Wu2020c}.} Figure~\subref*{fi:re_quantization} suggests that even $b=1$ (i.e., two-state reflection) brings considerable \gls{r-e} gain over the benchmark scheme without \gls{ris}, and the performance gap between $b=4$ and unquantized \gls{ris} is negligible. These observations demonstrate the advantage of the proposed joint waveform, active and passive beamforming design in practical \gls{ris}-aided \gls{swipt} systems.
\end{section}


\begin{section}{Conclusion and Future Works}\label{se:conclusion_and_future_works}
	This paper investigated the \gls{r-e} tradeoff of a single user employing practical receiving strategies in a \gls{ris}-aided multi-carrier \gls{miso} \gls{swipt} system. Uniquely, we considered the joint waveform, active and passive beamforming design under rectifier nonlinearity to maximize the achievable \gls{r-e} region. A three-stage \gls{bcd} algorithm was proposed to solve the problem. In the first stage, the \gls{ris} phase shift was obtained by the \gls{sca} technique and eigen decomposition. In the second and third stages, the active precoder was derived in closed form, and the waveform amplitude and splitting ratio were optimized by the \gls{gp} method. We also proposed and combined closed-form adaptive waveform schemes with a modified passive beamforming strategy to formulate a low-complexity \gls{bcd} algorithm that achieves a good balance between performance and complexity. Numerical results revealed significant \gls{r-e} gains by modeling harvester nonlinearity in the \gls{ris}-aided \gls{swipt} design. Unlike active antennas, \gls{ris} elements cannot be designed independently across frequencies, but can integrate coherent combining and equal gain transmission to enable constructive reflection and flexible subchannel design. Compared to the conventional no-\gls{ris} system, the \gls{ris} mainly affects the effective channel instead of the waveform design.

	One particular unanswered question of this paper is how to design waveform, active and passive beamforming in a multi-user multi-carrier \gls{ris}-aided \gls{swipt} system. Also, harvester saturation effect and practical \gls{ris} models with amplitude-phase coupling \cite{Abeywickrama2020}, angle-dependent reflection \cite{Tang2021}, frequency-dependent reflection, and/or partially/fully-connected architecture \cite{Shen2020a} could be considered in future works.
\end{section}
