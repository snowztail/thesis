%!TEX root = ../thesis.tex

\graphicspath{{assets/chapter_5/}}

\chapter{Channel Shaping using \glsfmtshort{ris}: From Diagonal Model to Beyond}\label{ch:channel_shaping}




\begin{section}{Introduction}
	Today we are witnessing a paradigm shift from connectivity to intelligence, where the wireless environment is no longer a chaotic medium but a conscious agent that can serve on demand.
	This is empowered by the recent advances in \gls{ris}, a programmable metasurface that recycles and redistributes ambient electromagnetic waves for improved wireless performance.
	A typical \gls{ris} consists of numerous low-power sub-wavelength non-resonant scattering elements, whose response can be engineered in real-time to manipulate the amplitude, phase, frequency, and polarization of the scattered waves \cite{Basar2019}.
	% It not only experiences negligible noise and supports full-duplex transmissions, but also features better flexibility than reflectarrays, lighter footprint than various relays, and greater scalability than conventional multi-antenna techniques.
	It enables low-noise, full-duplex operation, and also features better flexibility than reflectarrays, lighter footprint than various relays, and greater scalability than conventional multi-antenna techniques.
	The most popular \gls{ris} research direction is \emph{joint beamforming} design with transceivers for a specific performance measure, which has attracted significant attention in wireless communication \cite{Wu2019,Guo2020,Liu2022}, sensing \cite{He2022,Luo2022,Hua2023}, and power transfer literature \cite{Wu2020a,Feng2022,Zhao2022}.
	While passive beamforming suffers severe attenuation from double fading, it usually offers a squared asymptotic behavior than active beamforming (e.g., second-order array gain and fourth-order harvested power \cite{Zhao2022}).
	On the other hand, \gls{ris} can also be used for \emph{backscatter modulation} by periodically switching its reflection pattern within channel coherence time.
	This creates a free-ride message stream (similar to index modulation \cite{Basar2017}) with dual benefits: integrating with the legacy transmitter for enhanced channel capacity \cite{Karasik2020,Basar2020,Ye2022} or serving as a dedicated source for low-power uplink communication \cite{Liang2020,Zhao2024,Yang2024}.
	Different from above directions, \emph{channel shaping} exploits the \gls{ris} as a stand-alone device to modify the inherent properties of the propagation environment.
	This provides a ubiquitous wave scattering benchmark for different wireless applications and helps to decouple the \gls{ris}-transceiver design.
	Relevant shaping metrics can be classified into two categories:
	\begin{itemize}
		\item \emph{Singular value centric:} Closely related to the performance measures (e.g., achievable rate and harvested power \cite{Shen2021}) but sensitive to numerical perturbations. The impact of \gls{ris} has been studied in terms of minimum singular value \cite{ElMossallamy2021}, effective rank \cite{ElMossallamy2021,Meng2023}, condition number \cite{Zheng2022,Huang2023}, and degree of freedom \cite{Bafghi2022,Zheng2023,Chae2023}.
		\item \emph{Power centric:} The second-order statistics are less informative in \gls{mimo} but easier to analyze and optimize. The impact of \gls{ris} has been studied in terms of channel power gain \cite{Wu2019,Shen2020a,Nerini2023,Nerini2024,Santamaria2023} and leakage interference \cite{Santamaria2023a}.
	\end{itemize}

	While those works offer initial glimpses into the channel shaping potential, one critical question remains unaddressed: \emph{To what extent can a passive \gls{ris} reshape the \gls{mimo} channel in terms of singular values?}
	% \begin{enumerate}
	% 	\item What is the entire singular value region of a \gls{mimo} channel with \gls{ris}?
	% \end{enumerate}
	% they neither discuss the limitation
	% depict the entire singular value region nor provide a universal optimization framework.
	% It motivates the ultimate question: \emph{To what extent can a passive \gls{ris} reshape \gls{mimo} channels in terms of singular values?}
	% Despite the great potentials and fruitful outcomes, one critical unanswered question is the {channel shaping} capability: \emph{To what extent can a passive \gls{ris} reshape \gls{mimo} channels?}
	The answer depends heavily on the scattering model and hardware architecture.
	Most relevant works \cite{ElMossallamy2021,Meng2023,Zheng2022,Huang2023,Bafghi2022,Zheng2023,Chae2023,Wu2019,Santamaria2023a} assumed that each \gls{ris} element is tuned by a dedicated impedance and acts as an \emph{individual} scatterer \cite{Wu2020}.
	This ideally translates to a scattering matrix with unit-magnitude entries on the main diagonal and zeros elsewhere, applying merely a phase shift to the incoming signal.
	The concept was soon generalized to \gls{bd}-\gls{ris} \cite{Shen2020a} that physically groups adjacent elements using passive reconfigurable components.\footnote{Those components can be either symmetric (e.g., capacitors and inductors) or asymmetric (e.g., ring hybrids and branch-line hybrids) \cite{Ahn2006}, resulting in symmetric and asymmetric scattering matrices, respectively.}
	This allows \emph{cooperative} scattering --- wave impinging on one element can propagate within the circuit and depart partially from any element in the same group.
	It can thus redistribute both amplitude and phase of the scattered wave with zero power loss, generalizing the scattering matrix to block-diagonal with unitary blocks.
	Such a powerful model can be realized at reduced hardware cost using tree- and forest-connected architectures inspired by graph theory \cite{Nerini2024}.
	\gls{bd}-\gls{ris} can also function in hybrid transmitting-and-reflecting mode \cite{Li2023b} and multi-sector mode \cite{Li2023c} to provide full-space coverage and multi-user support.
	Many practical design challenges have been addressed including channel estimation \cite{Li2024}, mutual coupling \cite{Li2023f}, and wideband modelling \cite{Li2024a}.
	Its beamforming superiority has been studied extensively in \gls{siso} and \gls{miso} systems, where the problems were cast as single-user \gls{snr} maximization \cite{Shen2020a,Nerini2023,Nerini2024,Santamaria2023} and multi-user \gls{wsr} maximization \cite{Fang2023,Zhou2023,Li2023c,Soleymani2024}.
	However, the interplay between \gls{bd}-\gls{ris} and \gls{mimo} systems is still in the infancy stage.
	The authors of \cite{Bartoli2023} investigated the rate-optimal joint beamforming design for a specific \gls{bd}-\gls{ris}-aided \gls{mimo} system with blocked direct link and unitary (a.k.a. fully-connected) \gls{ris}.
	Similar constraints were also adopted in \cite{Mishra2024}, which introduces a transmitter-side \gls{bd}-\gls{ris} to massive \gls{mimo} that exploits statistical \gls{csi} for improved spectral efficiency.
	Received power maximization problem was studied over continuous-valued \cite{Nerini2023} and discrete-valued \cite{Nerini2023b} \gls{bd}-\gls{ris}, but the proposed single-stream transceiver is rate-suboptimal and the passive beamforming problem is equivalent to \gls{siso}.
	A practical frequency-dependent \gls{bd}-\gls{ris} model has been recently proposed for multi-band multi-cell \gls{mimo} networks to facilitate practical deployments \cite{Sena2024}.
	Although the results are promising, the channel shaping capability from off-diagonal entries deserves further investigation, and there lacks a universal yet efficient \gls{bd}-\gls{ris} design framework.
	Furthermore, no previous work has attempted to characterize the singular value region of a \gls{mimo} channel manipulated by any type of \gls{ris}, and the \gls{bd}-\gls{ris}-aided \gls{mimo} rate maximization problem remains unsolved.
	This paper aims for a comprehensive answer to the channel shaping question through theoretical analysis and numerical optimization.
	The contributions are summarized below.


	% 1. need a general study on BD-RIS x MIMO and explain the potential from off-diagonal entries
	% 2. need a general optimization framework for BD-RIS
	% 3. need a comprehensive answer to the channel shaping question
	% 4. need a low-complexity solution from shaping

	First, we discuss the shaping potential from off-diagonal entries of \gls{bd}-\gls{ris} in terms of subchannel rearrangement and subspace alignment.
	Subchannel rearrangement enables a flexible in-group permutation and combination of backward and forward channels by their strength.
	This unique feature of \gls{bd}-\gls{ris} provides a higher shaping freedom and exploits spatial diversity more effectively than diagonal \gls{ris}.
	On the other hand, subspace alignment in \gls{mimo} generalizes phase matching in \gls{siso} and \gls{miso} to the high-dimensional singular vector space.
	The scattering matrix simultaneously affects the multiplicative intra-group alignment and additive inter-group alignment.
	However, both subspaces cannot be perfectly aligned at the same time and a balance needs to be struck in between.
	Increasing \gls{mimo} dimensions highlights the advantage of \gls{bd}-\gls{ris} for subchannel rearrangement while exposing the limitation of diagonal \gls{ris} in subspace alignment.
	This is the first paper to study \gls{bd}-\gls{ris} in general \gls{mimo} systems and discuss its shaping potential.
	% unveil and interpret the potential of \gls{bd}-\gls{ris} in general \gls{mimo} systems.

	Second, we exploit the Riemannian geometry of the Stiefel manifold and propose an efficient \gls{bd}-\gls{ris} design framework based on geodesic\footnote{A geodesic refers to the shortest path between two points in a Riemannian manifold.} \gls{rcg}.
	This method modified from \cite{Abrudan2008,Abrudan2009} not only provides better objective value and faster convergence than general non-geodesic approach \cite{Absil2009,Pan2022d}, but also works for arbitrary group size and any smooth optimization problem.
	Specifically, group-wise multiplicative rotational updates are performed along the geodesics of the Stiefel manifold and compactly evaluated as the exponential map \cite{Edelman1998}.
	By exploiting the inherent structure of unitary matrices, this method avoids retractions from the Euclidean space and facilitates the step size selection, which improves the computational efficiency and stability.
	This is the first work to tailor an efficient and universal optimization framework for \gls{bd}-\gls{ris}.

	% pioneer
	Third, we quantify the capability of a \gls{bd}-\gls{ris} to redistribute the singular values of a \gls{mimo} \gls{pc} channel.
	The {Pareto frontiers} are characterized by optimizing the {weighted sum of singular values}, where the weights can be positive, zero, or negative.
	This problem is solved by the proposed geodesic \gls{rcg} algorithm.
	The resulting singular value region generalizes most relevant metrics and provides an intuitive channel shaping benchmark.
	% We also and derive some analytical singular value bounds for rank-deficient \gls{mimo} and unitary \gls{ris}.
	To explore the shaping limits of different \gls{ris} architectures, we also derive individual and collective singular value bounds for rank-deficient channels and unitary \gls{ris} via matrix analysis.
	Results validate those bounds and show that increasing \gls{bd}-\gls{ris} group size significantly enlarges the singular value region, providing wider dynamic range and better trade-off.
	% dynamic range and trade-off of singular values are significantly improved by \gls{bd}-\gls{ris}, especially in large-scale \gls{mimo} systems.
	This is the first work to comprehensively answer the channel shaping capability question from both numerical and analytical perspectives in the most general \gls{bd}-\gls{ris} setup.
	% from a Pareto perspective.

	Fourth, we tackle \gls{bd}-\gls{ris}-aided \gls{mimo} achievable rate maximization problem with two beamforming solutions: a local-optimal approach via \gls{ao} and a low-complexity approach over channel shaping.
	The former iteratively updates active beamforming by eigenmode transmission and passive beamforming by geodesic \gls{rcg} until convergence.
	The latter suboptimally decouples the joint design into a channel power gain maximization subproblem and a conventional \gls{mimo} transmission subproblem, then propose a closed-form two-stage solution.
	We observe that \gls{bd}-\gls{ris} yields higher channel power and achievable rate than diagonal \gls{ris}, while the relative gains scale with the group size and \gls{mimo} dimensions.
	% especially with large group sizes and \gls{mimo} dimensions.
	Moreover, the rate difference between the optimal and shaping-inspired designs diminishes as the \gls{ris} evolves from diagonal to unitary.
	% Interestingly, the rate difference between the optimal and low-complexity designs diminishes as the \gls{ris} evolves from diagonal to unitary.
	Those results emphasize the importance of using \gls{bd}-\gls{ris} in large-scale \gls{mimo} systems and suggest channel shaping offers a promising path to decouple joint \gls{ris}-transceiver designs.

	Fifth, we extend the aforementioned approaches to \gls{mimo}-\gls{ic} where the \gls{bd}-\gls{ris} is used for leakage interference minimization and \gls{wsr} maximization.
	In the former case, we update the receive combiner, transmit precoder, and scattering matrix iteratively in closed forms with optimality proof.
	In the latter case, we alternatively updates the transmit precoder by bisection and the scattering matrix by \gls{rcg} algorithm.
	We also compare the impact of \gls{ris} on \gls{pc} and \gls{ic}.
	% provide a comprehensive comparison between the two approaches and discuss the implications of channel shaping in \gls{ic}.



	% Fifth, extensive simulations reveal that the performance gain from \gls{bd}-\gls{ris} increases with group size and \gls{mimo} dimensions.
	% In terms of channel power, fully-connected \gls{bd}-\gls{ris} boosts up to 62\% and 270\% over diagonal \gls{ris} in $1 \times 1$ and $4 \times 4$ \gls{mimo} under independent Rayleigh fading, respectively.
	% The superiority stems from stronger \emph{subchannel rearrangement} and \emph{subspace alignment} capabilities empowered by in-group cooperation.
	% It emphasizes the importance of using \gls{bd}-\gls{ris} in large-scale \gls{mimo} systems.


\end{section}


\begin{section}{\glsfmtshort{mimo}-\glsfmtshort{pc}}

	\begin{subsection}{System Model}
		Consider a \gls{bd}-\gls{ris} aided \gls{mimo} \gls{pc} system with $N_\mathrm{T}$, $N_\mathrm{S}$, $N_\mathrm{R}$ transmit, scatter, and receive antennas, respectively.
		This configuration is denoted as $N_\mathrm{T} \times N_\mathrm{S} \times N_\mathrm{R}$ in the following context.
		The \gls{bd}-\gls{ris} can be modeled as an $N_\mathrm{S}$-port network \cite{Ivrlac2010} that further divides into $G$ individual groups, each containing $L \triangleq N_\mathrm{S} / G$ elements interconnected by real-time reconfigurable components \cite{Shen2020a}.
		To simplify the analysis and explore the performance limits, we assume a lossless asymmetric network without mutual coupling between scattering elements, as previously considered in \cite{Li2023b,Li2023c,Bartoli2023}.
		The overall scattering matrix of the \gls{bd}-\gls{ris} is block-unitary
		\begin{equation}
			\mathbf{\Theta} = \mathrm{diag}(\mathbf{\Theta}_1,\ldots,\mathbf{\Theta}_G),
			\label{eq:bd_ris}
		\end{equation}
		where $\mathbf{\Theta}_g \in \mathbb{U}^{L \times L}$ is the $g$-th unitary block (i.e., $\mathbf{\Theta}_g^\mathsf{H} \mathbf{\Theta}_g = \mathbf{I}$) that describes the response of group $g \in \mathcal{G} \triangleq \{1, \ldots, G\}$.
		Note that diagonal and unitary \gls{ris} can be regarded as its extreme cases with group size $L=1$ and $L=N_\mathrm{S}$, respectively.
		Some potential physical architectures of \gls{bd}-\gls{ris} are illustrated in \cite[Fig. 3]{Shen2020a}, \cite[Fig. 5]{Li2023c}, and \cite[Fig. 2]{Nerini2024}, where the radiation pattern and circuit topology need to be modeled in the scattering matrix.

		Let $\mathbf{H}_\mathrm{D} \in \mathbb{C}^{N_\mathrm{R} \times N_\mathrm{T}}$, $\mathbf{H}_\mathrm{B} \in \mathbb{C}^{N_\mathrm{R} \times N_\mathrm{S}}$, $\mathbf{H}_\mathrm{F} \in \mathbb{C}^{N_\mathrm{S} \times N_\mathrm{T}}$ denote the direct (transmitter-receiver), backward (\gls{ris}-receiver), and forward (transmitter-\gls{ris}) channels, respectively.
		The equivalent channel is a function of the scattering matrix
		\begin{equation}
			\mathbf{H} = \mathbf{H}_\mathrm{D} + \mathbf{H}_\mathrm{B} \mathbf{\Theta} \mathbf{H}_\mathrm{F} = \mathbf{H}_\mathrm{D} + \sum_g \underbrace{\mathbf{H}_{\mathrm{B},g} \mathbf{\Theta}_g \mathbf{H}_{\mathrm{F},g}}_{\triangleq \mathbf{H}_g},
			\label{eq:channel_equivalent}
		\end{equation}
		where $\mathbf{H}_{\mathrm{B},g} \in \mathbb{C}^{N_\mathrm{R} \times L}$ and $\mathbf{H}_{\mathrm{F},g} \in \mathbb{C}^{L \times N_\mathrm{T}}$ are the backward and forward subchannels for \gls{ris} group $g$, corresponding to the $(g{-}1)L$ to $gL$ columns of $\mathbf{H}_\mathrm{B}$ and rows of $\mathbf{H}_\mathrm{F}$, respectively.
		Let $\mathbf{H}_g \triangleq \mathbf{H}_{\mathrm{B},g} \mathbf{\Theta}_g \mathbf{H}_{\mathrm{F},g}$ be the indirect channel via \gls{bd}-\gls{ris} group $g$.
		Since unitary matrices constitute an algebraic group with respect to multiplication, the scattering matrix of group $g$ can be decomposed as
		\begin{equation}
			\mathbf{\Theta}_g = \mathbf{L}_g \mathbf{R}_g^\mathsf{H},
		\end{equation}
		where $\mathbf{L}_g, \mathbf{R}_g \in \mathbb{U}^{L \times L}$ are two unitary factor matrices.
		Let $\mathbf{H}_{\mathrm{B},g} = \mathbf{U}_{\mathrm{B},g} \mathbf{\Sigma}_{\mathrm{B},g} \mathbf{V}_{\mathrm{B},g}^\mathsf{H}$ and $\mathbf{H}_{\mathrm{F},g} = \mathbf{U}_{\mathrm{F},g} \mathbf{\Sigma}_{\mathrm{F},g} \mathbf{V}_{\mathrm{F},g}^\mathsf{H}$ be the compact \gls{svd} of the backward and forward channels, respectively.
		The equivalent channel can thus be rewritten as
		\begin{equation}
			\mathbf{H} = \overbrace{\mathbf{H}_\mathrm{D} + \sum_g \mathbf{U}_{\mathrm{B},g} \mathbf{\Sigma}_{\mathrm{B},g} \underbrace{\mathbf{V}_{\mathrm{B},g}^\mathsf{H} \mathbf{L}_g \mathbf{R}_g^\mathsf{H} \mathbf{U}_{\mathrm{F},g}}_\text{backward-forward} \mathbf{\Sigma}_{\mathrm{F},g} \mathbf{V}_{\mathrm{F},g}^\mathsf{H}}^\text{direct-indirect}.
			\label{eq:channel_equivalent_svd}
		\end{equation}

		% \begin{remark}
		By analyzing \eqref{eq:channel_equivalent_svd}, we conclude that the off-diagonal entries of the \gls{bd}-\gls{ris} scattering matrix provide two key potentials for \gls{mimo} channel shaping:
		\begin{itemize}
			\item \emph{Subchannel rearrangement:} This unique feature of \gls{bd}-\gls{ris} exploits the spatial diversity by rearranging and combining the backward and forward channel branches within each group. In \gls{siso}, diagonal \gls{ris} with perfect phase matching provides a maximum indirect channel amplitude of $\sum_{n=1}^{N_\mathrm{S}} \lvert h_{\mathrm{B},n} \rvert \lvert h_{\mathrm{F},n} \rvert$ while \gls{bd}-\gls{ris} can generalize it to $\sum_{g=1}^{G} \sum_{l=1}^{L} \lvert h_{\mathrm{B},\pi_{\mathrm{B},g}(l)} \rvert \lvert h_{\mathrm{F},\pi_{\mathrm{F},g}(l)} \rvert$, where $\pi_{\mathrm{B},g}$ and $\pi_{\mathrm{F},g}$ are permutations of $\mathcal{L} \triangleq \{1, \ldots, L\}$.
			Note the first summation is over groups and the second summation is over permuted subchannels.
			By rearrangement inequality, the maximum channel gain is attained by pairing the $l$-th strongest backward and forward branches.
			% , providing up to \qty{62.5}{\percent} asymptotic power gain over diagonal \gls{ris} in \gls{siso} Rayleigh fading \cite{Shen2020a}.
			Since the number of subchannels associated with each group is proportional to $N_\mathrm{T} N_\mathrm{R}$, we conclude the advantage of \gls{bd}-\gls{ris} in subchannel rearrangement scales with \gls{mimo} dimensions,
			\item \emph{Subspace alignment:} Each group can align the singular vectors of the associated backward-forward (intra-group, multiplicative) channels and direct-indirect (inter-group, additive) channels. In \gls{siso}, subspace alignment boils down to phase matching and the optimal scattering matrix of group $g$ that maximizes the channel gain is
			\begin{equation}
				\mathbf{\Theta}_g^\star = \exp \bigl(\jmath \mathrm{arg}(h_\mathrm{D})\bigr) \mathbf{V}_{\mathrm{B},g} \mathbf{U}_{\mathrm{F},g}^\mathsf{H},
				\label{eq:scattering_siso}
			\end{equation}
			where $\mathbf{V}_{\mathrm{B},g} = \bigl[\mathbf{h}_{\mathrm{B},g}/\lVert \mathbf{h}_{\mathrm{B},g} \rVert, \mathbf{N}_{\mathrm{B},g}\bigr] \in \mathbb{U}^{L \times L}$, $\mathbf{U}_{\mathrm{F},g} = \bigl[\mathbf{h}_{\mathrm{F},g}/\lVert \mathbf{h}_{\mathrm{F},g} \rVert, \mathbf{N}_{\mathrm{F},g}\bigr] \in \mathbb{U}^{L \times L}$, and $\mathbf{N}_{\mathrm{B},g}, \mathbf{N}_{\mathrm{F},g} \in \mathbb{C}^{L \times (L-1)}$ are the orthonormal bases of the null spaces of $\mathbf{h}_{\mathrm{B},g}$ and $\mathbf{h}_{\mathrm{F},g}$, respectively.
			Diagonal \gls{ris} ($L=1$, empty null spaces) thus suffices for perfect phase matching in \gls{siso}.
			When it comes to \gls{mimo}, each individual scattering element can only apply a common phase shift to the ``pinhole'' indirect channel $\mathbf{H}_g \in \mathbb{C}^{N_\mathrm{R} \times N_\mathrm{T}}$ passing through itself.
			That is, the disadvantage of diagonal \gls{ris} in subspace alignment scales with \gls{mimo} dimensions.
			As will be shown later, even if the \gls{bd}-\gls{ris} is unitary, there still exists a tradeoff between the alignment of direct-indirect and backward-forward subspaces.
		\end{itemize}
		% \end{remark}

		% TODO: add illustrations for both

		% In \gls{siso}, the equivalent channel can be further simplified as
		% \begin{equation}
		% 	h = h_\mathrm{D} + \sum_g \mathbf{h}_{\mathrm{B},g} \mathbf{\Theta}_g \mathbf{h}_{\mathrm{F},g},
		% \end{equation}
		% subspace alignment boils down to phase matching




		% For the \gls{siso} case in Fig. \subref*{fg:power_bond_txrx1_nd}, the maximum channel power is approximately \num{4e-6} by diagonal \gls{ris} and \num{6.5e-6} by fully-connected \gls{bd}-\gls{ris}, corresponding to a \qty{62.5}{\percent} gain.
		% This aligns with the asymptotic \gls{bd}-\gls{ris} scaling law derived for \gls{siso} in \cite{Shen2020a}.
		% Interestingly, the gain surges to \qty{270}{\percent} in 4T4R \gls{mimo} as shown in Fig. \subref*{fg:power_bond_txrx4_nd}.
		% This is because subspace alignment boils down to phase matching in \gls{siso} such that both triangular and Cauchy-Schwarz inequalities in \cite[(50)]{Shen2020a} can be simultaneously tight regardless of the group size.
		% That is, diagonal \gls{ris} is sufficient for subspace alignment in \gls{siso} while the \qty{62.5}{\percent} gain from \gls{bd}-\gls{ris} comes purely from subchannel rearrangement (i.e., pairing the forward and backward channels from strongest to weakest).
		% Now consider a diagonal \gls{ris} in \gls{mimo}.
		% Each element can only apply a common phase shift to the associated rank-1 $N_\mathrm{R} \times N_\mathrm{T}$ indirect channel.
		% Therefore, perfect subspace alignment of indirect channels through different elements is generally impossible.
		% It means the disadvantage of diagonal \gls{ris} in subspace alignment and subchannel rearrangement scales with \gls{mimo} dimensions.
		% We thus conclude that the power gain of \gls{bd}-\gls{ris} scales with group size and \gls{mimo} dimensions.


		% \begin{equation}
		% 	\mathbf{H} = \underbrace{\mathbf{H}_\mathrm{D} + \underbrace{\sum_g \mathbf{U}_{\mathrm{B},g} \mathbf{\Sigma}_{\mathrm{B},g} \underbrace{\mathbf{V}_{\mathrm{B},g}^\mathsf{H} \mathbf{L}_g \mathbf{R}_g^\mathsf{H} \mathbf{U}_{\mathrm{F},g}}_\text{backward-forward} \mathbf{\Sigma}_{\mathrm{F},g} \mathbf{V}_{\mathrm{F},g}^\mathsf{H}}_\text{group-wise}}_\text{direct-indirect},
		% \end{equation}
		% which involves three subspace alignment problems:
		% which involves three subspace alignment problems
		% \begin{equation}
		% 	\mathbf{H} = \mathbf{H}_\mathrm{D} + \sum_g \mathbf{U}_{\mathrm{B},g} \mathbf{\Sigma}_{\mathrm{B},g} \underbrace{\mathbf{V}_{\mathrm{B},g}^\mathsf{H} \mathbf{L}_g \mathbf{R}_g^\mathsf{H} \mathbf{U}_{\mathrm{F},g}}_{\substack{\text{backward-forward}\\\text{additive alignment}}} \mathbf{\Sigma}_{\mathrm{F},g} \mathbf{V}_{\mathrm{F},g}^\mathsf{H}.
		% \end{equation}
		% \begin{equation}
		% 	\mathbf{H} = \mathbf{H}_\mathrm{D} + \sum_g \mathbf{U}_{\mathrm{B},g} \mathbf{\Sigma}_{\mathrm{B},g} \underbrace{\mathbf{V}_{\mathrm{B},g}^\mathsf{H} \mathbf{L}_g}_{\substack{\text{backward}\\\text{align}}} \underbrace{\mathbf{R}_g^\mathsf{H} \mathbf{U}_{\mathrm{F},g}}_{\substack{\text{forward}\\\text{align}}} \mathbf{\Sigma}_{\mathrm{F},g} \mathbf{V}_{\mathrm{F},g}^\mathsf{H}.
		% \end{equation}
		% Then the equivalent channel can be rewritten as
		% Since the field of (block) unitary matrices formulate a field that is closed under multiplication
		% Since block unitary matrices satisfy the axioms of a field under matrix multiplication
		% \begin{remark}
		% 	% To understand the \gls{b} gain from in-group connections
		% \end{remark}




		% \begin{remark}
		% 	From \eqref{eq:scattering_fc} and \eqref{eq:channel_equivalent_fc} in the proof of Proposition \ref{pp:fully_connected}, we notice that \eqref{iq:sv_bound_fc}--\eqref{iq:sv_bound_fc_power} are simultaneously tight when
		% 	% Interestingly, \eqref{iq:sv_bound_fc}--\eqref{iq:sv_bound_fc_power} are simultaneously tight when
		% 	\begin{equation}
		% 		\mathbf{\Theta} = \mathbf{V}_\mathrm{B} \mathbf{U}_\mathrm{F}^\mathsf{H}.
		% 		\label{eq:scattering_fc_tight}
		% 	\end{equation}
		% 	An interpretation is that the off-diagonal entries can enhance the capabilities of
		% 	\begin{itemize}
		% 		\item subspace alignment: $\mathbf{V}_\mathrm{B}$ and $\mathbf{U}_\mathrm{F}^\mathsf{H}$ in \eqref{eq:scattering_fc} fully align the subspaces of $\mathbf{H}_\mathrm{B}$ and $\mathbf{H}_\mathrm{F}$ by rotation;
		% 		\item subchannel rearrangement: $\mathbf{X} = \mathbf{I}$ in \eqref{eq:channel_equivalent_fc} pairs the subchannels of $\mathbf{H}_\mathrm{B}$ and $\mathbf{H}_\mathrm{F}$ from strongest to weakest, which attains the maximum in rearrangement inequality.
		% 	\end{itemize}
		% \end{remark}



		% \begin{remark}
		% 	\gls{bd}-\gls{ris} reduces to diagonal \gls{ris} and unitary \gls{ris} with group size $L$ 1 and $N_\mathrm{S}$, respectively.
		% \end{remark}
		% \begin{remark}
		% 	Individual forward and backward \gls{csi} are required for \gls{bd}-\gls{ris} designs.
		% 	This is different from diagonal \gls{ris} where estimating their product is usually sufficient.
		% 	% Later we will show the potential benefits from the \gls{csi} overhead.
		% \end{remark}
	\end{subsection}

	\begin{subsection}{Group-Wise Geodesic \glsfmtshort{rcg}}
		In this section, we first provide an overview on signal processing techniques for general \gls{bd}-\gls{ris} design problems, then propose a novel group-wise geodesic \gls{rcg} method that exploits the properties of unitary group to operate directly on the Stiefel manifold.
		% The proposed method is applicable to any smooth optimization problem on the Stiefel manifold and is particularly suitable for \gls{bd}-\gls{ris} design problems.
		We will later show that the proposed method not only provides better objective value and faster convergence than other approaches, but also works for arbitrary group size and any smooth optimization problem.
		\begin{subsubsection}{Conventional Techniques}
			General \gls{bd}-\gls{ris} design techniques can be classified into two categories based on the optimization variable and hardware architecture:
			\begin{itemize}
				\item \emph{Scattering matrix $\mathbf{\Theta}$:} This approach is often exploited for asymmetric architectures where the feasible domain of each group is a $L$-dimensional Stiefel manifold $\mathbf{\Theta}_g \in \mathbb{U}^{L \times L}$. Due to its non-convexity, relevant problems are usually solved by general non-geodesic manifold optimization \cite{Li2023b,Li2023c,Bartoli2023} or relax-then-project methods \cite{Fang2023}. The former will be discussed in the next subsection.
				\item \emph{Reactance matrix $\mathbf{X}$:} This approach is often exploited for symmetric architectures where every pair of elements in the same group are connected by capacitors and inductors. According to network theory \cite{Pozar2011}, it maps to the scattering matrix by $\mathbf{\Theta}_g = (\jmath \mathbf{X}_g + Z_0 \mathbf{I})^{-1} (\jmath \mathbf{X}_g - Z_0 \mathbf{I})$, which formulates an unconstrained optimization problem on the upper triangular entries of $\{\mathbf{X}_g\}_{g \in \mathcal{G}}$ that is solvable by quasi-Newton methods \cite{Shen2020a}.
			\end{itemize}
		\end{subsubsection}

		\begin{subsubsection}{Geodesic vs Non-Geodesic \glsfmtshort{rcg}}
			We first revisit the general non-geodesic \gls{rcg} method that is applicable to optimization problems over arbitrary manifolds \cite{Absil2009,Pan2022d}.
			The main idea is to perform {additive} updates along the conjugate direction guided by the Riemannian gradient, then {project} the solution back onto the manifold.
			% update the feasible point by progressing along conjugate directions then project back to the manifold.
			% The main idea is to update the feasible point by progressing along conjugate directions then project back to the manifold.
			For maximization problem with smooth objective $f$ and block-unitary constraint \eqref{eq:bd_ris}, the steps for \gls{bd}-\gls{ris} group $g$ at iteration $r$ are summarized below:
			\begin{enumerate}
				\item \emph{Compute the Euclidean gradient \cite{Hjorungnes2007}:} The gradient of $f$ with respect to $\mathbf{\Theta}_g^*$ in the Euclidean space is
				\begin{equation}
					\nabla_{\mathrm{E},g}^{(r)} = \frac{\partial f(\mathbf{\Theta}_g^{(r)})}{\partial \mathbf{\Theta}_g^*};
					\label{eq:gradient_euclidean}
				\end{equation}
				\item \emph{Translate to the Riemannian gradient \cite{Absil2009}:} At point $\mathbf{\Theta}^{(r)}$, the Riemannian gradient lies in the tangent space of the Stiefel manifold $\mathcal{T}_{\mathbf{\Theta}_g^{(r)}}\mathbb{U}^{L \times L} \triangleq \{\mathbf{M} \in \mathbb{C}^{L \times L} \mid \mathbf{M}^\mathsf{H} \mathbf{\Theta}_g^{(r)} + {\mathbf{\Theta}_g^{(r)\mathsf{H}}} \mathbf{M} = \mathbf{0}\}$. It gives the steepest ascent direction of the objective on the manifold can be obtained by projecting the Euclidean gradient onto the tangent space:
				\begin{equation}
					% \nabla_{\mathrm{R},g}^{(r)} = \nabla_{\mathrm{E},g}^{(r)} {\mathbf{\Theta}_g^{(r)\mathsf{H}}} - \mathbf{\Theta}_g^{(r)} {\nabla_{\mathrm{E},g}^{(r)\mathsf{H}}};
					\nabla_{\mathrm{R},g}^{(r)} = \nabla_{\mathrm{E},g}^{(r)} - \mathbf{\Theta}_g^{(r)} {\nabla_{\mathrm{E},g}^{(r)\mathsf{H}}} \mathbf{\Theta}_g^{(r)};
					\label{eq:gradient_riemannian}
				\end{equation}
				\item \emph{Determine the conjugate direction \cite{Nocedal2006}:} The conjugate direction is obtained over the Riemannian gradient and previous direction as
				\begin{equation}
					\mathbf{D}_g^{(r)} = \nabla_{\mathrm{R},g}^{(r)} + \gamma_g^{(r)} \mathbf{D}_g^{(r-1)}, % $\mathcal{O}(N_\mathrm{S}^2)$
					\label{eq:direction_cg}
				\end{equation}
				where $\gamma_g^{(r)}$ is the parameter that deviates the conjugate direction from the tangent space for accelerated convergence. A popular choice is the Polak-Ribi\`{e}re formula
				\begin{equation}
					\gamma_g^{(r)} = \frac{\mathrm{tr}\bigl((\nabla_{\mathrm{R},g}^{(r)} - \nabla_{\mathrm{R},g}^{(r-1)}) {\nabla_{\mathrm{R},g}^{(r)\mathsf{H}}}\bigr)}{\mathrm{tr}\bigl(\nabla_{\mathrm{R},g}^{(r-1)} {\nabla_{\mathrm{R},g}^{(r-1)\mathsf{H}}}\bigr)}; % $\mathcal{O}(2 N_\mathrm{S}^3 + N_\mathrm{S}^2 + 2 N_\mathrm{S})$
					\label{eq:parameter_cg}
				\end{equation}
				\item \emph{Perform additive update \cite{Pan2022d}:} The point is updated by moving along a straight path in the conjugate direction
				\begin{equation}
					\bar{\mathbf{\Theta}}_g^{(r+1)} = \mathbf{\Theta}_g^{(r)} + \mu \mathbf{D}_g^{(r)},
					\label{eq:update_additive}
				\end{equation}
				where $\mu$ is the step size refinable by the Armijo rule \cite{Armijo1966};
				\item \emph{Retract for feasibility \cite{Absil2009,Li2023b}:} The resulting point needs to be projected back onto the Stiefel manifold by
				\begin{equation}
					\mathbf{\Theta}_g^{(r+1)} = \bar{\mathbf{\Theta}}_g^{(r+1)} \bigl({\bar{\mathbf{\Theta}}_g^{(r+1)\mathsf{H}}} \bar{\mathbf{\Theta}}_g^{(r+1)}\bigr)^{-1/2}.
					\label{eq:retraction}
				\end{equation}
				One can also combine the addition \eqref{eq:update_additive} and retraction \eqref{eq:retraction} in one step
				\begin{equation}
					\mathbf{\Theta}_g^{(r+1)} = \bigl(\mathbf{\Theta}_g^{(r)} + \mu \mathbf{D}_g^{(r)}\bigr) \bigl( \mathbf{I} + \mu^2 {\mathbf{D}_g^{(r)\mathsf{H}}} \mathbf{D}_g^{(r)} \bigr)^{-1/2},
					\label{eq:add_then_retract}
				\end{equation}
				and determine the step size therein.
			\end{enumerate}

			A geodesic is a curve representing the shortest path between two points in a Riemannian manifold, whose tangent vectors remain parallel when transported along the curve.
			The above method is called non-geodesic since the points are updated in the linear embedding spaces of the manifold by addition and retraction, instead of on the Stiefel manifold itself.
			Next, we revisit some concepts in differential geometry and introduce a group-wise geodesic \gls{rcg} method on top of \cite{Abrudan2008,Abrudan2009}.

			A Lie group is simultaneously a continuous group and a differentiable manifold.
			Lie algebra refers to the tangent space of the Lie group at the identity element.
			The exponential map acts as a bridge between the Lie algebra and Lie group, which allows one to recapture the local group structure using linear algebra techniques.
			The set of unitary matrices $\mathbb{U}^{L \times L}$ forms a Lie group $U(L)$ under multiplication, and the corresponding Lie algebra $\mathfrak{u}(L) \triangleq \mathcal{T}_{\mathbf{I}}\mathbb{U}^{L \times L} = \{\mathbf{M} \in \mathbb{C}^{L \times L} \mid \mathbf{M}^\mathsf{H} + \mathbf{M} = \mathbf{0}\}$ consists of skew-Hermitian matrices.
			A geodesic emanating from the identity with velocity $\mathbf{D} \in \mathfrak{u}(L)$ can be described by \cite{Edelman1998}
			\begin{equation}
				\mathbf{G}_\mathbf{I}(\mu) = \exp(\mu \mathbf{D}),
				\label{eq:geodesic_identity}
			\end{equation}
			where $\exp(\mathbf{A}) = \sum_{k=0}^\infty (\mathbf{A}^k/k!)$ is the matrix exponential and $\mu$ is the step size (i.e., magnitude of the tangent vector).
			Note that the right translation is an isometry in $U(L)$.
			During the optimization of group $g$, the geodesic evaluated at the identity \eqref{eq:geodesic_identity} should be translated to $\mathbf{\Theta}_g^{(r)}$ for successive updates \cite{Abrudan2008}
			\begin{equation}
				\mathbf{G}_g^{(r)}(\mu) = \mathbf{G}_\mathbf{I}(\mu) \mathbf{\Theta}_g^{(r)} = \exp(\mu \mathbf{D}_g^{(r)}) \mathbf{\Theta}_g^{(r)},
				\label{eq:geodesic_translated}
			\end{equation}
			while the Riemannian gradient evaluated at $\mathbf{\Theta}_g^{(r)}$ \eqref{eq:gradient_riemannian} should be translated back to the identity for exploiting the Lie algebra \cite{Abrudan2008}
			\begin{equation}
				\tilde{\nabla}_{\mathrm{R},g}^{(r)} = \nabla_{\mathrm{R},g}^{(r)} \mathbf{\Theta}_g^{(r)\mathsf{H}} = \nabla_{\mathrm{E},g}^{(r)} \mathbf{\Theta}_g^{(r)\mathsf{H}} - \mathbf{\Theta}_g^{(r)} {\nabla_{\mathrm{E},g}^{(r)\mathsf{H}}}.
				% $\mathcal{O}(2 L^3)$
				\label{eq:gradient_translated}
			\end{equation}
			After gradient translation, the deviation parameter and conjugate direction can be determined similarly to \eqref{eq:parameter_cg} and \eqref{eq:direction_cg}
			\begin{equation}
				\tilde{\gamma}_g^{(r)} = \frac{\mathrm{tr}\bigl((\tilde{\nabla}_{\mathrm{R},g}^{(r)} - \tilde{\nabla}_{\mathrm{R},g}^{(r-1)}) {\tilde{\nabla}_{\mathrm{R},g}^{(r)\mathsf{H}}}\bigr)}{\mathrm{tr}\bigl(\tilde{\nabla}_{\mathrm{R},g}^{(r-1)} {\tilde{\nabla}_{\mathrm{R},g}^{(r-1)\mathsf{H}}}\bigr)}. % $\mathcal{O}(2 N_\mathrm{S}^3 + N_\mathrm{S}^2 + 2 N_\mathrm{S})$
				\label{eq:parameter_cg_geodesic}
			\end{equation}
			\begin{equation}
				{\mathbf{D}}_g^{(r)} = \tilde{\nabla}_{\mathrm{R},g}^{(r)} + \tilde{\gamma}_g^{(r)} {\mathbf{D}}_g^{(r-1)},
				\label{eq:direction_cg_geodesic}
			\end{equation}
			The solution can thus be updated along the geodesic in a multiplicative rotational manner
			\begin{equation}
				\mathbf{\Theta}_g^{(r+1)} = \mathbf{G}_g^{(r)}(\mu) = \exp(\mu \mathbf{D}_g^{(r)}) \mathbf{\Theta}_g^{(r)},
				\label{eq:update_geodesic}
			\end{equation}
			where an appropriate $\mu$ may be obtained by the Armijo rule.
			To double the step size, one can simply square the rotation matrix instead of recomputing the matrix exponential, since $\exp^2(\mu \mathbf{D}_g^{(r)}) = \exp(2 \mu \mathbf{D}_g^{(r)})$.

			\begin{algorithm}[!t]
				\caption{Group-wise geodesic \gls{rcg} for \gls{bd}-\gls{ris} design}
				\label{ag:rcg}
				\begin{algorithmic}[1]
					\Require $f(\mathbf{\Theta})$, $G$
					\Ensure $\mathbf{\Theta}^\star$
					\Initialize {$r \gets 0$, $\mathbf{\Theta}^{(0)}$}
					\Repeat
						\For {$g \gets 1$ to $G$}
							\State $\nabla_{\mathrm{E},g}^{(r)} \gets$ \eqref{eq:gradient_euclidean} \label{ln:gradient_euclidean}
							\State $\tilde{\nabla}_{\mathrm{R},g}^{(r)} \gets$ \eqref{eq:gradient_translated}
							\State $\tilde{\gamma}_g^{(r)} \gets$ \eqref{eq:parameter_cg_geodesic}
							\State $\mathbf{D}_g^{(r)} \gets$ \eqref{eq:direction_cg_geodesic}
							\If {$\Re\bigl\{\mathrm{tr}({\mathbf{D}_g^{(r)\mathsf{H}}} \tilde{\nabla}_{\mathrm{R},g}^{(r)})\bigr\} < 0$} \Comment{not an ascent direction}
								\State $\mathbf{D}_g^{(r)} \gets \tilde{\nabla}_{\mathrm{R},g}^{(r)}$
							\EndIf
							\State $\mu \gets 1$
							\State $\mathbf{G}_g^{(r)}(\mu) \gets$ \eqref{eq:geodesic_translated}
							\While {$f\bigl(\mathbf{G}_g^{(r)}(2\mu)\bigr) - f(\mathbf{\Theta}_g^{(r)}) \ge \mu \cdot \mathrm{tr}(\mathbf{D}_g^{(r)} {\mathbf{D}_g^{(r)\mathsf{H}}}) / 2$} \label{ln:armijo_start}
								\State $\mu \gets 2 \mu$
							\EndWhile
							\While {$f\bigl(\mathbf{G}_g^{(r)}(\mu)\bigr) - f(\mathbf{\Theta}_g^{(r)}) < \mu / 2 \cdot \mathrm{tr}(\mathbf{D}_g^{(r)} {\mathbf{D}_g^{(r)\mathsf{H}}}) / 2$}
								\State $\mu \gets \mu / 2$
							\EndWhile \label{ln:armijo_end}
							\State $\mathbf{\Theta}_g^{(r+1)} \gets$ \eqref{eq:update_geodesic}
						\EndFor
						\State $r \gets r+1$
					\Until $\lvert f(\mathbf{\Theta}^{(r)}) - f(\mathbf{\Theta}^{(r-1)}) \rvert / f(\mathbf{\Theta}^{(r-1)}) \le \epsilon$
				\end{algorithmic}
			\end{algorithm}

			Algorithm~\ref{ag:rcg} summarizes the proposed \gls{bd}-\gls{ris} design framework based on group-wise geodesic \gls{rcg}.
			Compared to the non-geodesic approach, it leverages the Lie group properties to replace the add-then-retract update \eqref{eq:add_then_retract} with a multiplicative rotational update \eqref{eq:update_geodesic} along the geodesic.
			This leads to faster convergence and simplifies the step size tuning thanks to appropriate parameter space.
			% Despite the difference in
			Convergence to a local optimum is still guaranteed if not initialized at a stationary point.
			Note that the group-wise updates can be performed in parallel to facilitate large-scale \gls{bd}-\gls{ris} optimization problems.
			Since block-unitary matrices are also closed under multiplication, one can avoid group-wise updates by directly operating on $\mathbf{\Theta}$ and pinching (i.e., keeping the block diagonal and nulling the other entries) the Euclidean gradient \eqref{eq:gradient_euclidean}, with potentially higher computational complexity and slower convergence.
		\end{subsubsection}
	\end{subsection}


	\begin{subsection}{Channel Singular Values Redistribution}
		\begin{subsubsection}{Toy Example}\label{sc:toy_example}
			We first illustrate the channel shaping capabilities of different \gls{ris} models by a toy example.
			Consider a $2 \times 2 \times 2$ setup where the direct link is blocked.
			The diagonal \gls{ris} is modeled by $\mathbf{\Theta}_\mathrm{D} = \mathrm{diag}(e^{\jmath \theta_1}, e^{\jmath \theta_2})$ while the unitary \gls{bd}-\gls{ris} has 4 independent angular parameters
			\begin{equation}
				\mathbf{\Theta}_\mathrm{U} = e^{\jmath \phi} \begin{bmatrix}
					e^{\jmath \alpha} \cos \psi  & e^{\jmath \beta} \sin \psi   \\
					-e^{-\jmath \beta} \sin \psi & e^{-\jmath \alpha} \cos \psi
				\end{bmatrix}.
				\label{eq:unitary_ris}
			\end{equation}
			It is worth noting that $\phi$ has no impact on the singular value because $\mathrm{sv}(e^{\jmath \phi} \mathbf{A}) = \mathrm{sv}(\mathbf{A})$.
			For a fair comparison, we also enforce symmetry $\mathbf{\Theta}_\mathrm{U} = \mathbf{\Theta}_\mathrm{U}^\mathsf{T}$ by $\beta = \pi / 2$ such that both architectures have the same number of angular parameters.
			\begin{figure}
				\centering
				\includegraphics[width=\columnwidth]{assets/chapter_5/singular_trend.eps}
				\caption{$2 \times 2 \times 2$ (no direct) channel singular value shaping by diagonal and symmetry unitary \gls{ris}.}
				\label{fg:singular_trend}
			\end{figure}
			Fig.~\ref{fg:singular_trend} shows the channel singular values achieved by an exhaustive grid search over $(\theta_1, \theta_2)$ for diagonal \gls{ris} and $(\alpha, \psi)$ for symmetric unitary \gls{ris}.
			It is observed that both singular values can be manipulated up to $9\%$ using diagonal \gls{ris} and $42\%$ using symmetric \gls{bd}-\gls{ris}, despite both architectures have the same number of scattering elements.
			A larger performance gap is expected when the symmetric constraint on \eqref{eq:unitary_ris} can be relaxed.
			This example shows \gls{bd}-\gls{ris} can provide a wider dynamic range of channel singular values and motivates further studies on channel shaping.
		\end{subsubsection}

		\begin{subsubsection}{Pareto Frontier Characterization}\label{sc:pareto_frontier}
			We then characterize the Pareto frontier of singular values of a general $N_\mathrm{T} \times N_\mathrm{S} \times N_\mathrm{R}$ channel \eqref{eq:channel_equivalent} by maximizing their weighted sum
			\begin{maxi!}
				{\scriptstyle{\mathbf{\Theta}}}{\sum_n \rho_n \sigma_n(\mathbf{H})}{\label{op:pareto}}{\label{ob:pareto}}
				\addConstraint{\mathbf{\Theta}_g^\mathsf{H} \mathbf{\Theta}_g=\mathbf{I},}{\quad \forall g,}{\label{co:pareto_unitary}}
			\end{maxi!}
			where $n \in \mathcal{N} \triangleq \{1,\ldots,N\}$, $N \triangleq \min(N_\mathrm{T}, N_\mathrm{R})$ is the maximum channel rank, and $\rho_n$ is the weight of the $n$-th singular value that can be positive, zero, or negative.
			Varying $\{\rho_n\}_{n \in \mathcal{N}}$ characterizes the Pareto frontier that encloses the entire singular value region.
			Thus, we claim problem \eqref{op:pareto} generalizes most singular value shaping problems.
			It can be solved optimally by Algorithm~\ref{ag:rcg} with the Euclidean gradient given by Lemma \ref{lm:pareto_gradient}.
			% It can be solved optimally by Algorithm~\ref{ag:rcg} where line \ref{ln:gradient_euclidean} uses the following Euclidean gradient expression \eqref{eq:pareto_gradient}.

			\begin{lemma}\label{lm:pareto_gradient}
				The Euclidean gradient of \eqref{ob:pareto} with respect to \gls{bd}-\gls{ris} group $g$ is
				\begin{equation}
					\frac{\partial \sum_n \rho_n \sigma_n(\mathbf{H})}{\partial \mathbf{\Theta}_g^*} = \mathbf{H}_{\mathrm{B},g}^\mathsf{H} \mathbf{U} \mathrm{diag}(\rho_1,\ldots,\rho_N) \mathbf{V}^\mathsf{H} \mathbf{H}_{\mathrm{F},g}^\mathsf{H},
					\label{eq:pareto_gradient}
				\end{equation}
				where $\mathbf{U}$ and $\mathbf{V}$ are the left and right compact singular matrices of $\mathbf{H}$, respectively.
			\end{lemma}
			\begin{proof}
				Please refer to Appendix~\ref{ap:pareto_gradient}.
			\end{proof}

			We then analyze the computational complexity of solving Pareto singular value problem \eqref{op:pareto} by Algorithm~\ref{ag:rcg}.
			To update each \gls{bd}-\gls{ris} group, compact \gls{svd} of $\mathbf{H}$ requires $\mathcal{O}(N N_\mathrm{T} N_\mathrm{R})$, Euclidean gradient \eqref{eq:pareto_gradient} requires $\mathcal{O}\bigl(L N (N_\mathrm{T}+N_\mathrm{R}+L+1) \bigr)$, Riemannian gradient translation \eqref{eq:gradient_translated} requires $\mathcal{O}(L^3)$, deviation parameter \eqref{eq:parameter_cg_geodesic} and conjugate direction \eqref{eq:direction_cg_geodesic} together require $\mathcal{O}(L^2)$, and matrix exponential \eqref{eq:update_geodesic} requires $\mathcal{O}(L^3)$ operations \cite{Moler2003}.
			The overall complexity is thus $\mathcal{O}\bigl(I_\text{RCG} G \bigl(N N_\mathrm{T} N_\mathrm{R} + L N (N_\mathrm{T}+N_\mathrm{R}+L+1) + I_\text{BLS} L^3\bigr)\bigr)$, where $I_\text{RCG}$ and $I_\text{BLS}$ are the number of iterations for geodesic \gls{rcg} and backtracking line search (line \ref{ln:armijo_start}--\ref{ln:armijo_end} of Algorithm \ref{ag:rcg}), respectively.
		\end{subsubsection}

		\begin{subsubsection}{Some Analytical Bounds}\label{sc:bounds}
			We then discuss some analytical bounds related to channel singular values.
			\begin{proposition}[Degree of freedom]\label{pp:dof}
				In point-to-point \gls{mimo}, \gls{bd}-\gls{ris} cannot achieve a higher \gls{dof} than diagonal \gls{ris}.
			\end{proposition}
			\begin{proof}
				Please refer to Appendix~\ref{ap:dof}.
			\end{proof}

			\begin{proposition}[Rank-deficient channel]\label{pp:rank_deficient}
				If the forward or backward channel is rank-$k$ ($k \le N$), then regardless of the passive \gls{ris} size and architecture, the $n$-th singular value of the equivalent channel is bounded by
				\begin{subequations}
					\begin{align}
						\sigma_n(\mathbf{H}) & \le \sigma_{n-k}(\mathbf{T}), &  & \text{if } n > k, \label{iq:sv_bound_enlarge}          \\
						\sigma_n(\mathbf{H}) & \ge \sigma_n(\mathbf{T}),     &  & \text{if } n < N - k + 1, \label{iq:sv_bound_suppress}
					\end{align}
					\label{iq:sv_bound_rank_deficient}
				\end{subequations}
				where
				\begin{equation}
					\mathbf{T} \mathbf{T}^\mathsf{H} =
					\begin{cases}
						\mathbf{H}_\mathrm{D} (\mathbf{I} - \mathbf{V}_\mathrm{F} \mathbf{V}_\mathrm{F}^\mathsf{H}) \mathbf{H}_\mathrm{D}^\mathsf{H}, & \text{if } \mathrm{rank}(\mathbf{H}_\mathrm{F}) = k, \\
						\mathbf{H}_\mathrm{D}^\mathsf{H} (\mathbf{I} - \mathbf{U}_\mathrm{B} \mathbf{U}_\mathrm{B}^\mathsf{H}) \mathbf{H}_\mathrm{D}, & \text{if } \mathrm{rank}(\mathbf{H}_\mathrm{B}) = k,
					\end{cases}
					\label{eq:auxiliary_matrix}
				\end{equation}
				and $\mathbf{V}_\mathrm{F}$ and $\mathbf{U}_\mathrm{B}$ are the right and left compact singular matrices of $\mathbf{H}_\mathrm{F}$ and $\mathbf{H}_\mathrm{B}$, respectively.
			\end{proposition}
			\begin{proof}
				Please refer to Appendix~\ref{ap:rank_deficient}.
			\end{proof}

			\begin{corollary}[Extreme singular values\label{co:extreme}]
				With a sufficiently large passive \gls{ris} of arbitrary architecture, the first $k$ channel singular values are unbounded above\footnote{The energy conservation law $\sum_n \sigma_n^2(\mathbf{H}) \le 1$ still has to be respected. This constraint is omitted in the following context for brevity.} while the last $k$ channel singular values can be suppressed to zero.
			\end{corollary}

			\begin{proof}
				This is a direct result of \eqref{iq:sv_bound_rank_deficient}.
			\end{proof}

			\begin{corollary}[\gls{los} channel\footnote{A similar result has been derived for diagonal \gls{ris} \cite{Semmler2023}.}\label{co:los}]
				% If the forward or backward channel is \gls{los}, then a \gls{ris} can at most enlarge (resp. suppress) the $n$-th ($n \ge 2$) channel singular value to the $(n-1)$-th (resp. $n$-th) singular value of $\mathbf{T}$, that is,
				If the forward or backward channel is \gls{los}, then a passive \gls{ris} of arbitrary architecture can at most enlarge the $n$-th ($n \ge 2$) channel singular value to the $(n-1)$-th singular value of $\mathbf{T}$, or suppress the $n$-th channel singular value to the $n$-th singular value of $\mathbf{T}$.
				That is,
				\begin{equation}
					\sigma_1(\mathbf{H}) \ge \sigma_1(\mathbf{T}) \ge {\sigma_2(\mathbf{H})} \ge \ldots \ge \sigma_{N-1}(\mathbf{T}) \ge {\sigma_N(\mathbf{H})} \ge \sigma_N(\mathbf{T}).
					\label{iq:sv_bound_los}
				\end{equation}
			\end{corollary}

			\begin{proof}
				This is a direct result of \eqref{iq:sv_bound_rank_deficient} with $k = 1$.
			\end{proof}

			In Section~\ref{sc:simulation}, we will show by simulation that a finite-size \gls{bd}-\gls{ris} can better approach those bounds than diagonal \gls{ris}.

			\begin{proposition}[Unitary \gls{ris} without direct link]\label{pp:fully_connected}
				% If the \gls{bd}-\gls{ris} is unitary and the direct link is absent, then the singular value bounds on $\mathbf{H}$ are equivalent to the singular value bounds on $\mathbf{BF}$, where $\mathbf{B}$ and $\mathbf{F}$ are arbitrary matrices with the same singular values as $\mathbf{H}_\mathrm{B}$ and $\mathbf{H}_\mathrm{F}$, respectively,
				If the \gls{bd}-\gls{ris} is unitary and the direct link is absent, then the channel singular values can be manipulated up to
				% If the \gls{bd}-\gls{ris} is unitary and the direct link is absent, then the bounds on the channel singular values are equivalent to the singular values of the equivalent channel, that is,
				% only bounds that apply here are those that apply to the singular values of
				\begin{equation}
					\mathrm{sv}(\mathbf{H}) = \mathrm{sv}(\mathbf{BF}),
				\end{equation}
				where $\mathbf{B}$ and $\mathbf{F}$ are arbitrary matrices with the same singular values as $\mathbf{H}_\mathrm{B}$ and $\mathbf{H}_\mathrm{F}$, respectively,
				% Our focus thus shifts to how the singular values of matrix product are bounded by the singular values of its individual factors.
			\end{proposition}

			\begin{proof}
				Please refer to Appendix~\ref{ap:fully_connected}.
			\end{proof}

			The problem now becomes, how the singular values of matrix product are bounded by the singular values of its individual factors.
			Let $\bar{N} = \max(N_\mathrm{T},N_\mathrm{S},N_\mathrm{R})$ and $\sigma_n(\mathbf{H})=\sigma_n(\mathbf{H}_\mathrm{F})=\sigma_n(\mathbf{H}_\mathrm{B})=0$ for $N < n \le \bar{N}$.
			We have the following corollaries.

			\begin{corollary}[Generic singular value bounds\label{co:sv_bound_horn}]
				\begin{equation}
					\prod_{k \in {K}} \sigma_k(\mathbf{H}) \le \prod_{i \in {I}} \sigma_i(\mathbf{H}_\mathrm{B}) \prod_{j \in {J}} \sigma_j(\mathbf{H}_\mathrm{F}),
					\label{iq:sv_bound_fc}
				\end{equation}
				for all admissible triples $(I, J, K) \in T_r^{\bar{N}}$ with $r < \bar{N}$, where
				\begin{equation*}
					\begin{gathered}
						% T_r^{\bar{N}} \triangleq \Bigl\{(I, J, K) \in U_r^{\bar{N}} \mid \text{for all } p < r \text{ and all } (F, G, H) \text{ in } T_p^r,\\
						% \sum_{f \in F} i_f + \sum_{g \in G} j_g \le \sum_{h \in H} k_h + p(p+1)/2 \Bigr\},
						T_r^{\bar{N}} \triangleq \Bigl\{(I, J, K) \in U_r^{\bar{N}} \mid \forall p < r, (F, G, H) \in T_p^r,\\
						\sum_{f \in F} i_f + \sum_{g \in G} j_g \le \sum_{h \in H} k_h + p(p+1)/2 \Bigr\},
					\end{gathered}
				\end{equation*}
				\begin{equation*}
					U_r^{\bar{N}} \triangleq \Bigl\{(I, J, K) \mid \sum_{i \in I} i + \sum_{j \in J} j = \sum_{k \in K} k + r(r+1)/2\Bigr\}.
				\end{equation*}
			\end{corollary}

			\begin{proof}
				Please refer to \cite[Theorem~8]{Fulton2000}.
			\end{proof}

			Corollary \eqref{co:sv_bound_horn} is by far the most comprehensive singular value bound over Proposition~\ref{pp:fully_connected}, which is also recognized as a variation of Horn's inequality \cite{Bhatia2001}.
			It is worth mentioning that the number of admissible triples (and bounds) grows exponentially with $\bar{N}$.
			For example, the number of inequalities described by \eqref{iq:sv_bound_fc} grows from 12 to 2062 when $\bar{N}$ increases from 3 to 7.
			This renders it computationally expensive for applications in large-scale \gls{mimo} systems.
			Next, we showcase some useful inequalities enclosed by \eqref{iq:sv_bound_fc}.
			Readers are referred to \cite[Chapter 16, 24]{Hogben2013} for more examples.

			\begin{corollary}[Upper bound on the largest singular value\label{co:sv_largest}]
				\begin{equation}
					\sigma_1(\mathbf{H}) \le \sigma_1(\mathbf{H}_\mathrm{B}) \sigma_1(\mathbf{H}_\mathrm{F}).
				\end{equation}
			\end{corollary}

			\begin{proof}
				This is a direct result of \eqref{iq:sv_bound_fc} with $r = 1$.
			\end{proof}

			\begin{corollary}[Lower bound on the smallest singular value\label{co:sv_smallest}]
				\begin{equation}
					\sigma_{\bar{N}}(\mathbf{H}) \ge \sigma_{\bar{N}}(\mathbf{H}_\mathrm{B}) \sigma_{\bar{N}}(\mathbf{H}_\mathrm{F}).
				\end{equation}
			\end{corollary}

			\begin{proof}
				This can be deducted from \eqref{iq:sv_bound_fc} with $r_1 = \bar{N}{-}1$ and $r_2 = \bar{N}$.
			\end{proof}

			\begin{corollary}[Upper bound on the product of first $k$ singular values]
				\begin{equation}
					\prod_{n=1}^k \sigma_n(\mathbf{H}) \le \prod_{n=1}^k \sigma_n(\mathbf{H}_\mathrm{B}) \prod_{n=1}^k \sigma_n(\mathbf{H}_\mathrm{F}).
				\end{equation}
			\end{corollary}

			\begin{proof}
				This is a direct result of \eqref{iq:sv_bound_fc} with $r = k$.
			\end{proof}

			\begin{corollary}[Lower bound on the product of last $k$ singular values\label{co:sv_product_smallest}]
				\begin{equation}
					\prod_{n=\bar{N}}^{\bar{N}-k+1} \sigma_n(\mathbf{H}) \ge \prod_{n=\bar{N}}^{\bar{N}-k+1} \sigma_n(\mathbf{H}_\mathrm{B}) \prod_{n=\bar{N}}^{\bar{N}-k+1} \sigma_n(\mathbf{H}_\mathrm{F}).
				\end{equation}
			\end{corollary}

			\begin{proof}
				This can be deducted from \eqref{iq:sv_bound_fc} with $r_1 = \bar{N}-k$ and $r_2 = \bar{N}$.
			\end{proof}

			Corollaries \ref{co:sv_smallest} and \ref{co:sv_product_smallest} are less informative when $\bar{N} \ne N$ (i.e., unequal number of transmit, scatter, and receive antennas) as the lower bounds would coincide at zero.

			% \begin{corollary}[Upper bound on the sum of first $k$ singular values to the positive $p$-th power\label{co:sum_power} {\cite[Inequality~24.4.7]{Hogben2013}}]
			% 	\begin{equation}
			% 		\sum_{n=1}^k \sigma_n^p(\mathbf{H}) \le \sum_{n=1}^k \sigma_n^p(\mathbf{H}_\mathrm{B}) \sigma_n^p(\mathbf{H}_\mathrm{F}), \quad p > 0.
			% 		\label{iq:sv_bound_fc_power}
			% 	\end{equation}
			% 	When $k = N$ and $p = 2$, \eqref{iq:sv_bound_fc_power} suggests the channel power is upper bounded by the sum of (sorted) element-wise power product of backward and forward subchannels
			% 	\begin{equation}
			% 		\lVert \mathbf{H} \rVert _\mathrm{F}^2 \le \sum_{n=1}^N \sigma_n^2(\mathbf{H}_\mathrm{B}) \sigma_n^2(\mathbf{H}_\mathrm{F}),
			% 	\end{equation}
			% \end{corollary}

			\begin{corollary}[Upper bound on the channel power gain\label{co:sum_power}]
				% The channel power gain is upper bounded by the sum of sorted element-wise power product of backward and forward subchannels
				The channel power gain is upper bounded by the sum of sorted element-wise product of squared singular values of backward and forward subchannels
				\begin{equation}
					\lVert \mathbf{H} \rVert _\mathrm{F}^2 = \sum_{n=1}^N \sigma_n^2(\mathbf{H}) \le \sum_{n=1}^N \sigma_n^2(\mathbf{H}_\mathrm{B}) \sigma_n^2(\mathbf{H}_\mathrm{F}).
					\label{iq:power_gain}
				\end{equation}
			\end{corollary}

			\begin{proof}
				Please refer to \cite[Inequality~24.4.7]{Hogben2013}.
			\end{proof}

			To achieve the equalities in Corollaries \eqref{co:sv_largest} -- \eqref{co:sum_power}, the \gls{ris} needs to completely align the subspaces of $\mathbf{H}_\mathrm{B}$ and $\mathbf{H}_\mathrm{F}$.
			The resulting scattering matrix is generally required to be unitary
			\begin{equation}
				\mathbf{\Theta}^\star = \mathbf{V}_\mathrm{B} \mathbf{U}_\mathrm{F}^\mathsf{H},
				\label{eq:scattering_fc_optimal}
			\end{equation}
			which can be concluded from \eqref{eq:scattering_fc} and \eqref{eq:channel_equivalent_fc} in Appendix~\ref{ap:fully_connected}.
			Interestingly, diagonal \gls{ris} can attain those equalities if and only if $\mathbf{H}_\mathrm{B}$ and $\mathbf{H}_\mathrm{F}$ are both rank-1.
			In such case, the equivalent channel reduces to $\mathbf{H} = \sigma_\mathrm{B} \sigma_\mathrm{F} \mathbf{u}_\mathrm{B} \mathbf{v}_\mathrm{B}^\mathsf{H} \mathbf{\Theta} \mathbf{u}_\mathrm{F} \mathbf{v}_\mathrm{F}^\mathsf{H}$ and the \gls{ris} only needs to align $\mathbf{v}_\mathrm{B}^\mathsf{H}$ and $\mathbf{u}_\mathrm{F}$ by
			\begin{equation}
				\mathbf{\Theta}^\star = \mathbf{v}_\mathrm{B} \mathbf{u}_\mathrm{F}^\mathsf{H} \odot \mathbf{I},
			\end{equation}
			which becomes a special case of \eqref{eq:scattering_siso}.
			On the other hand, when $\mathbf{H}_\mathrm{B}$ and $\mathbf{H}_\mathrm{F}$ are both in Rayleigh fading, the expected maximum channel power gain $\mathbb{E}\bigl\{ \lVert \mathbf{H} \rVert _ \mathrm{F} \bigr\}_{\max}$ can be evaluated as
			\begin{equation}
				\sum_{n=1}^N \int_0^\infty f_{\lambda_n^{\min(N_\mathrm{R},N_\mathrm{S})}}(x_n) d x_n \int_0^\infty f_{\lambda_n^{\min(N_\mathrm{S},N_\mathrm{T})}}(x_n) d x_n,
				\label{eq:power_gain_rayleigh}
			\end{equation}
			where $\lambda_n^{K}$ is the $n$-th eigenvalue of the complex $K \times K$ Wishart matrix with probability density function $f_{\lambda_n^{K}}(x_n)$ given by \cite[Equation 51]{Zanella2009}.
			We notice \eqref{eq:power_gain_rayleigh} is a generalization of \cite[Equation 58]{Shen2020a} to \gls{mimo}.

			Tighter bounds are generally inapplicable when the direct link is present or the \gls{bd}-\gls{ris} is not unitary, since the direct-indirect channels and backward-forward channels cannot be completely aligned at the same time.
			In such case, we can exploit optimization approaches from a singular value perspective (Section~\ref{sc:pareto_frontier}) or a power gain perspective (Section~\ref{sc:power}).
		\end{subsubsection}
	\end{subsection}

	\begin{subsection}{Power Gain and Achievable Rate Maximization}\label{sc:power_rate}
		\begin{subsubsection}{Channel Power Gain}\label{sc:power}
			The \gls{mimo} channel power gain maximization problem is formulated with respect to the \gls{bd}-\gls{ris} scattering matrix
			\begin{maxi!}
				{\scriptstyle{\mathbf{\Theta}}}{\lVert \mathbf{H}_\mathrm{D} + \mathbf{H}_\mathrm{B} \mathbf{\Theta} \mathbf{H}_\mathrm{F} \rVert _\mathrm{F}^2}{\label{op:power_passive}}{\label{ob:power_passive}}
				\addConstraint{\mathbf{\Theta}_g^\mathsf{H} \mathbf{\Theta}_g=\mathbf{I}, \quad \forall g,}{}{}
			\end{maxi!}
			which generalizes the case of \gls{siso} \cite{Shen2020a}, \gls{miso} \cite{Santamaria2023,Fang2023}, single-stream \gls{mimo} \cite{Nerini2023,Nerini2023b}, and direct link-blocked \gls{mimo} with unitary \gls{ris} \eqref{eq:scattering_fc_optimal}.
			The key of solving \eqref{op:power_passive} is to balance the additive and multiplicative subspace alignments.
			Interestingly, in terms of maximizing the inner product $\langle \mathbf{H}_\mathrm{D}, \mathbf{H}_\mathrm{B} \mathbf{\Theta} \mathbf{H}_\mathrm{F} \rangle$, \eqref{op:power_passive} is reminiscent of the weighted orthogonal Procrustes problem \cite{Gower2004}
			\begin{mini!}
				{\scriptstyle{\mathbf{\Theta}}}{\lVert \mathbf{H}_\mathrm{D} - \mathbf{H}_\mathrm{B} \mathbf{\Theta} \mathbf{H}_\mathrm{F} \rVert _\mathrm{F}^2}{\label{op:weighted_orthogonal_procrustes}}{}
				\addConstraint{\mathbf{\Theta}^\mathsf{H} \mathbf{\Theta}=\mathbf{I},}{\label{cs:power_block_unitary}}{}
			\end{mini!}
			which relaxes the block-unitary constraint \eqref{cs:power_block_unitary} to unitary but still has no trivial solution.
			One lossy transformation exploits the Moore-Penrose inverse and moves $\mathbf{\Theta}$ to one side of the product \cite{Bell2003}, formulating two standard orthogonal Procrustes problems
			\begin{mini!}
				{\scriptstyle{\mathbf{\Theta}}}{\lVert \mathbf{H}_\mathrm{B}^\dagger \mathbf{H}_\mathrm{D} - \mathbf{\Theta} \mathbf{H}_\mathrm{F} \rVert _\mathrm{F}^2 \text{ or } \lVert \mathbf{H}_\mathrm{D} \mathbf{H}_\mathrm{F}^\dagger - \mathbf{H}_\mathrm{B} \mathbf{\Theta} \rVert _\mathrm{F}^2}{\label{op:standard_orthogonal_procrustes}}{}
				\addConstraint{\mathbf{\Theta}^\mathsf{H} \mathbf{\Theta}=\mathbf{I},}{}{}
			\end{mini!}
			which have global optimal solutions
			\begin{equation}
				\mathbf{\Theta} = \mathbf{U} \mathbf{V}^\mathsf{H},
				\label{eq:orthogonal_procrustes_solution}
			\end{equation}
			where $\mathbf{U}$ and $\mathbf{V}$ are respectively the left and right compact singular matrices of $\mathbf{H}_\mathrm{B}^\dagger \mathbf{H}_\mathrm{D} \mathbf{H}_\mathrm{F}^\mathsf{H}$ or $\mathbf{H}_\mathrm{B}^\mathsf{H} \mathbf{H}_\mathrm{D} \mathbf{H}_\mathrm{F}^\dagger$ \cite{Golub2013}.
			We emphasize that \eqref{eq:scattering_fc_optimal} and \eqref{eq:orthogonal_procrustes_solution} are valid unitary \gls{ris} solutions to \eqref{op:power_passive} when the direct link is absent and present, but the latter is neither optimal nor a generalization of the former due to the lossy transformation.

			Inspired by \cite{Nie2017}, we propose an optimal solution to problem \eqref{op:power_passive} with arbitrary group size.
			The idea is to successively approximate the quadratic objective \eqref{ob:power_passive} by local Taylor expansions and solve each step in closed form by group-wise \gls{svd}.

			\begin{proposition}\label{pp:power}
				Starting from any feasible $\mathbf{\Theta}^{(0)}$, the sequence
				\begin{equation}
					\mathbf{\Theta}_g^{(r+1)} = \mathbf{U}_g^{(r)} \mathbf{V}_g^{(r)}, \quad \forall g.
					\label{eq:scattering_power}
				\end{equation}
				converges to a stationary point of \eqref{op:power_passive}, where $\mathbf{U}_g^{(r)}$ and $\mathbf{V}_g^{(r)}$ are the left and right compact singular matrices of
				\begin{equation}
					\mathbf{M}_g^{(r)} = \mathbf{H}_{\mathrm{B},g}^\mathsf{H} \Bigl(\mathbf{H}_\mathrm{D} + \mathbf{H}_\mathrm{B} \mathrm{diag}\bigl(\mathbf{\Theta}_{[1:g-1]}^{(r+1)},\mathbf{\Theta}_{[g:G]}^{(r)}\bigr) \mathbf{H}_\mathrm{F}\Bigr) \mathbf{H}_{\mathrm{F},g}^\mathsf{H}
					\label{eq:auxiliary_matrix_power}
				\end{equation}
			\end{proposition}

			\begin{proof}
				Please refer to Appendix~\ref{ap:power}.
			\end{proof}

			We then analyze the computational complexity of solving channel gain maximization problem \eqref{op:power_passive} by Proposition \ref{pp:power}.
			To update each \gls{bd}-\gls{ris} group, matrix multiplication \eqref{eq:auxiliary_matrix_power} requires $\mathcal{O}\bigl(N_\mathrm{T} N_\mathrm{R} + (G+1)(NL^2+N_\mathrm{T} N_\mathrm{R} L)\bigr)$ operations and its compact \gls{svd} requires $\mathcal{O}(L^3)$ operations.
			The overall complexity is thus $\mathcal{O}\bigl(I_\text{SAA} G \bigl(N_\mathrm{T} N_\mathrm{R} + (G+1)(NL^2+N_\mathrm{T} N_\mathrm{R} L) + L^3\bigr)\bigr)$, where $I_\text{SAA}$ is the number iterations for successive affine approximation.
		\end{subsubsection}

		\begin{subsubsection}{Achievable Rate Maximization}\label{sc:rate}
			We aim to maximize the achievable rate of the \gls{bd}-\gls{ris}-aided \gls{mimo} system by jointly optimizing the active and passive beamforming
			\begin{maxi!}
				{\scriptstyle{\mathbf{W},\mathbf{\Theta}}}{R = \log \det \biggl(\mathbf{I} + \frac{\mathbf{W}^\mathsf{H}\mathbf{H}^\mathsf{H}\mathbf{H}\mathbf{W}}{\eta}\biggr)}{\label{op:rate}}{\label{ob:rate}}
				\addConstraint{\lVert \mathbf{W} \rVert _\mathrm{F}^2}{\le P}
				\addConstraint{\mathbf{\Theta}_g^\mathsf{H} \mathbf{\Theta}_g}{=\mathbf{I}, \quad \forall g,\label{cs:rate_block_unitary}}
			\end{maxi!}
			where $\mathbf{W}$ is the transmit precoder, $R$ is the achievable rate, $\eta$ is the average noise power, and $P$ is the transmit power constraint.
			Problem \eqref{op:rate} is non-convex due to the block-unitary constraint \eqref{cs:rate_block_unitary} and the coupling between variables.
			We propose a local-optimal approach via \gls{ao} and a low-complexity approach based on channel shaping.

			\begin{paragraph}{Alternating Optimization}
				This approach updates $\mathbf{\Theta}$ and $\mathbf{W}$ iteratively until convergence.
				For a given $\mathbf{W}$, the passive beamforming subproblem is
				\begin{maxi!}
					{\scriptstyle{\mathbf{\Theta}}}{\log \det \biggl(\mathbf{I} + \frac{\mathbf{H} \mathbf{Q} \mathbf{H}^\mathsf{H}}{\eta}\biggr)}{\label{op:rate_passive}}{\label{ob:rate_passive}}
					\addConstraint{\mathbf{\Theta}_g^\mathsf{H} \mathbf{\Theta}_g=\mathbf{I}, \quad \forall g,}{}{}
				\end{maxi!}
				where $\mathbf{Q} \triangleq \mathbf{W} \mathbf{W}^\mathsf{H}$ is the transmit covariance matrix.
				Problem \eqref{op:rate_passive} can be solved optimally by Algorithm \ref{ag:rcg} with the Euclidean gradient given by Lemma \ref{lm:rate_gradient}.
				\begin{lemma}\label{lm:rate_gradient}
					The Euclidean gradient of \eqref{ob:rate_passive} with respect to \gls{bd}-\gls{ris} block $g$ is
					\begin{equation}
						\frac{\partial R}{\partial \mathbf{\Theta}_g^*} = \frac{1}{\eta} \mathbf{H}_{\mathrm{B},g}^\mathsf{H} \biggl(\mathbf{I} + \frac{\mathbf{H}\mathbf{Q}\mathbf{H}^\mathsf{H}}{\eta}\biggr)^{-1} \mathbf{H} \mathbf{Q} \mathbf{H}_{\mathrm{F},g}^\mathsf{H}.
						\label{eq:rate_gradient}
					\end{equation}
				\end{lemma}

				\begin{proof}
					Please refer to Appendix~\ref{ap:rate_gradient}.
				\end{proof}
				For a given $\mathbf{\Theta}$, the global optimal transmit precoder is given by eigenmode transmission \cite{Clerckx2013}
				\begin{equation}
					\mathbf{W}^\star = \mathbf{V} {\mathrm{diag}(\mathbf{s}^\star)}^{1/2},
					\label{eq:precoder_eigenmode}
				\end{equation}
				where $\mathbf{V}$ is the right singular matrix of the equivalent channel and $\mathbf{s}^\star$ is the optimal water-filling power allocation obtainable by the iterative method \cite{Tse2005}.

				The \gls{ao} algorithm is guaranteed to converge to local-optimal points of problem \eqref{op:rate} since each subproblem is solved optimally and the objective is bounded above.
				Similar to the analysis in Section \ref{sc:pareto_frontier}, the computational complexity of solving subproblem \eqref{op:rate_passive} by geodesic \gls{rcg} is $\mathcal{O}\bigl(I_\text{RCG} G (NL^2 + L N_\mathrm{T} N_\mathrm{R} + N_\mathrm{T}^2 N_\mathrm{R} + N_\mathrm{T} N_\mathrm{R}^2 + N_\mathrm{R}^3 + I_\text{BLS} L^3)\bigr)$.
				On the other hand, the complexity of solving active beamforming subproblem by \eqref{eq:precoder_eigenmode} is $\mathcal{O}\bigl(N N_\mathrm{T} N_\mathrm{R}\bigr)$.
				The overall complexity is thus $\mathcal{O}\bigl(I_\text{AO}\bigl(I_\text{RCG} G (NL^2 + L N_\mathrm{T} N_\mathrm{R} + N_\mathrm{T}^2 N_\mathrm{R} + N_\mathrm{T} N_\mathrm{R}^2 + N_\mathrm{R}^3 + I_\text{BLS} L^3) + N N_\mathrm{T} N_\mathrm{R}\bigr)\bigr)$, where $I_\text{AO}$ is the number of iterations for \gls{ao}.
			\end{paragraph}

			\begin{paragraph}{Low-Complexity Solution}\label{sc:low_complexity}
				We then propose a suboptimal two-stage solution to problem \eqref{op:rate} that decouples the joint \gls{ris}-transceiver design.
				The idea is to first consider channel shaping and replace the rate maximization subproblem \eqref{op:rate_passive} by channel power gain maximization problem \eqref{op:power_passive}, then proceed to conventional eigenmode transmission \eqref{eq:precoder_eigenmode}.
				Both steps are solved in closed form and the computational complexity is $\mathcal{O}\bigl(I_\text{SAA} G \bigl(N_\mathrm{T} N_\mathrm{R} + (G+1)(NL^2+N_\mathrm{T} N_\mathrm{R} L) + L^3\bigr) + N N_\mathrm{T} N_\mathrm{R}\bigr)$.
				While suboptimal, this shaping-inspired solution avoids outer iterations and implements inner iterations more efficiently.
			\end{paragraph}
		\end{subsubsection}
	\end{subsection}

\end{section}

\begin{section}{\glsfmtshort{mimo}-\glsfmtshort{ic}}
	\begin{subsection}{System Model}
		We then shift the focus to an $N_\mathrm{S}$-element \gls{bd}-\gls{ris}-aided interference \gls{mimo} system of $K$ transmitter-receiver pairs, where each transmitter is equipped with $N_\mathrm{T}$ antennas and each receiver is equipped with $N_\mathrm{R}$ antennas.
		Let $\mathbf{H}_\mathrm{D}^{[kj]} \in \mathbb{C}^{N_\mathrm{R} \times N_\mathrm{T}}$, $\mathbf{H}_\mathrm{B}^{[k]} \in \mathbb{C}^{N_\mathrm{R} \times N_\mathrm{S}}$, $\mathbf{H}_\mathrm{F}^{[j]} \in \mathbb{C}^{N_\mathrm{S} \times N_\mathrm{T}}$ denote respectively the direct channel from transmitter $j$ to receiver $k$, backward channel of receiver $k$, and forward channel of transmitter $j$, where $j, k \in \{1,\ldots,K\} \triangleq \mathcal{K}$.
		It is assumed that all devices are at the same side of the \gls{ris} and share a common scattering matrix $\mathbf{\Theta}$.
		The equivalent channel from transmitter $j$ to receiver $k$ is
		\begin{equation}
			\mathbf{H}^{[kj]} = \mathbf{H}_\mathrm{D}^{[kj]} + \mathbf{H}_\mathrm{B}^{[k]} \mathbf{\Theta} \mathbf{H}_\mathrm{F}^{[j]} = \mathbf{H}_\mathrm{D}^{[kj]} + \sum_g \mathbf{H}_{\mathrm{B},g}^{[k]} \mathbf{\Theta}_g \mathbf{H}_{\mathrm{F},g}^{[j]},
			\label{eq:channel_equivalent_ic}
		\end{equation}
		where $\mathbf{H}_{\mathrm{B},g}^{[k]} \in \mathbb{C}^{N_\mathrm{R} \times L}$ and $\mathbf{H}_{\mathrm{F},g}^{[j]} \in \mathbb{C}^{L \times N_\mathrm{T}}$ are associated with \gls{ris} group $g$.
	\end{subsection}
	\begin{subsection}{Leakage Interference Minimization}
		Leakage interference refers to the sum of all interference terms in \gls{ic}.
		Assume all transmitter sends $N_\mathrm{E} \le N \triangleq \min(N_\mathrm{T}, N_\mathrm{R})$ data streams to its intended receiver.
		The leakage interference minimization problem is formulated with respect to all transmit precoders, all receiver combiners, and the scattering matrix as
		\begin{mini!}
			{\scriptstyle{\mathbf{\Theta}, \{\mathbf{G}_k\}, \{\mathbf{W}_k\}}}{\mathop{\sum\sum}_{j \neq k} \left\lVert \mathbf{G}_k (\mathbf{H}^{[kj]}_\mathrm{D} + \mathbf{H}^{[k]}_\mathrm{B} \mathbf{\Theta} \mathbf{H}^{[j]}_{\mathrm{F}}) \mathbf{W}_j \right\rVert _{\mathrm{F}}^2}{\label{op:leakage}}{}
			\addConstraint{\mathbf{\Theta}_g^\mathsf{H} \mathbf{\Theta}_g=\mathbf{I}, \quad}{\forall g}{\label{co:scatter}}
			\addConstraint{\mathbf{G}_k \mathbf{G}_k^\mathsf{H}=\mathbf{I}, \quad}{\forall k}{\label{co:combiner}}
			\addConstraint{\mathbf{W}_j^\mathsf{H} \mathbf{W}_j=\mathbf{I}, \quad}{\forall j,}{\label{co:precoder}}
		\end{mini!}
		where $\mathbf{G}_k \in \mathbb{C}^{N_\mathrm{D} \times N_\mathrm{R}}$ is the combiner at receiver $k$ and $\mathbf{W}_j \in \mathbb{C}^{N_\mathrm{T} \times N_\mathrm{D}}$ is the precoder at transmitter $j$.
		It is worth noting that $\mathbf{\Theta}_g$ is a square unitary matrix and $\mathbf{\Theta}_g^\mathsf{H} \mathbf{\Theta}_g=\mathbf{I}$ is equivalent to $\mathbf{\Theta}_g \mathbf{\Theta}_g^\mathsf{H}=\mathbf{I}$, while $\mathbf{G}_k$ and $\mathbf{W}_j$ may be respectively ``fat'' and ``tall'' rectangular semi-unitary matrices such that \eqref{co:combiner} and \eqref{co:precoder} are required.
		The non-convex problem can be solved iteratively by the \gls{bcd} method detailed below.

		\begin{subsubsection}{Combiner and Precoder Design}
			For a given $\mathbf{\Theta}$, problem \eqref{op:leakage} reduces to conventional linear beamforming problem, for which an iterative algorithm alternating between the original and reciprocal networks is proposed in \cite{Gomadam2011,Clerckx2013}.
			At iteration $r$, the combiner at receiver $k$ is updated as
			\begin{equation}
				\mathbf{G}_k^{(r)} = {\mathbf{U}_{k,N}^{(r-1)}}^\mathsf{H},
				\label{eq:combiner_leakage}
			\end{equation}
			where $\mathbf{U}_{k,N}^{(r-1)}$ is the eigenvectors corresponding to $N$ smallest eigenvalues of the interference covariance matrix $\mathbf{Q}_k^{(r-1)} = \sum_{j \ne k} {\mathbf{H}^{[kj]}} \mathbf{W}_j^{(r-1)} {\mathbf{W}_j^{(r-1)}}^\mathsf{H} {\mathbf{H}^{[kj]}}^\mathsf{H}$.
			The precoder at transmitter $j$ is updated as
			\begin{equation}
				\mathbf{W}_j^{(r)} = \bar{\mathbf{U}}_{j,N}^{(r)},
				\label{eq:precoder_leakage}
			\end{equation}
			where $\bar{\mathbf{U}}_{j,N}^{(r)}$ corresponds $\bar{\mathbf{Q}}_j^{(r)} = \sum_{k \ne j} {\mathbf{H}^{[kj]}}^\mathsf{H} {\mathbf{G}_k^{(r)}}^\mathsf{H} \mathbf{G}_k^{(r)} {\mathbf{H}^{[kj]}}$ in the reciprocal network.
		\end{subsubsection}

		\begin{subsubsection}{Scattering Matrix Design}
			Once $\{\mathbf{G}_k\}$ and $\{\mathbf{W}_j\}$ are determined, we define $\bar{\mathbf{H}}^{[kj]}_\mathrm{D} \triangleq \mathbf{G}_k \mathbf{H}^{[kj]}_\mathrm{D} \mathbf{W}_j$, $\bar{\mathbf{H}}^{[k]}_\mathrm{B} \triangleq \mathbf{G}_k \mathbf{H}^{[k]}_\mathrm{B}$, and $\bar{\mathbf{H}}^{[j]}_\mathrm{F} \triangleq \mathbf{H}^{[j]}_\mathrm{F} \mathbf{W}_j$.
			The \gls{bd}-\gls{ris} subproblem boils down to
			\begin{mini!}
				{\scriptstyle{\mathbf{\Theta}}}{\mathop{\sum\sum}_{j \neq k} \left\lVert (\bar{\mathbf{H}}^{[kj]}_\mathrm{D} + \bar{\mathbf{H}}^{[k]}_\mathrm{B} \mathbf{\Theta} \bar{\mathbf{H}}^{[j]}_\mathrm{F}) \right\rVert _{\mathrm{F}}^2}{\label{op:ic_interference_ris}}{}
				\addConstraint{\mathbf{\Theta}_g^\mathsf{H} \mathbf{\Theta}_g=\mathbf{I}, \quad \forall g,}{}{}
			\end{mini!}
			which can be viewed as a channel shaping problem on $\bar{\mathbf{H}}^{[kj]}_\mathrm{D}$, $\bar{\mathbf{H}}^{[k]}_\mathrm{B}$, and $\bar{\mathbf{H}}^{[j]}_\mathrm{F}$.

			\begin{proposition}
				Start from any $\mathbf{\Theta}^{(0)}$, the sequence
				\begin{equation}
					\mathbf{\Theta}_g^{(r+1)} = \mathbf{U}_g^{(r)} \mathbf{V}_g^{(r)}, \quad \forall g
					\label{eq:scatter_leakage}
				\end{equation}
				converges to a stationary point of \eqref{op:ic_interference_ris}, where $\mathbf{U}_g^{(r)}$ and $\mathbf{V}_g^{(r)}$ are left and right singular matrix of
				\begin{equation}
					\mathbf{M}_g^{(r)} = \mathop{\sum\sum}_{j \neq k} \bigl(\mathbf{B}_{k,g} \mathbf{\Theta}_g^{(r)} \mathbf{H}^{[j]}_{\mathrm{F},g} - {\mathbf{H}^{[k]}_{\mathrm{B},g}}^\mathsf{H} {\mathbf{D}^{[kj]}_{g}}^{(r)}\bigr) {\mathbf{H}^{[j]}_{\mathrm{F},g}}^\mathsf{H},
				\end{equation}
				where $\mathbf{B}_{k,g} = \lambda_1\bigl({\mathbf{H}^{[k]}_{\mathrm{B},g}}^\mathsf{H} \mathbf{H}^{[k]}_{\mathrm{B},g}\bigr) \mathbf{I} - {\mathbf{H}^{[k]}_{\mathrm{B},g}}^\mathsf{H} \mathbf{H}^{[k]}_{\mathrm{B},g}$ and
				\begin{equation}
					{\mathbf{D}^{[kj]}_{g}}^{(r)} = \mathbf{H}^{[jk]}_\mathrm{D} + \sum_{g'<g} {\mathbf{H}_{k,g'}^\mathrm{B}}^\mathsf{H} \mathbf{\Theta}_{g'}^{(r+1)} \mathbf{H}_{k,g'}^\mathrm{F} + \sum_{g'>g} {\mathbf{H}_{k,g'}^\mathrm{B}}^\mathsf{H} \mathbf{\Theta}_{g'}^{(r)} \mathbf{H}_{k,g'}^\mathrm{F}.
				\end{equation}
			\end{proposition}
			\begin{proof}
				The proof is similar to Appendix~\ref{ap:power} and omitted here.
			\end{proof}
		\end{subsubsection}

		Problem \eqref{op:leakage} can be solved iteratively by \eqref{eq:combiner_leakage}, \eqref{eq:precoder_leakage}, and \eqref{eq:scatter_leakage}.
		Convergence to local-optimal points is guaranteed since each subproblem is solved optimally and the objective is bounded above.

	\end{subsection}

	\begin{subsection}{Weighted Sum-Rate Maximization}
		Finally, we consider weighted sum-rate maximization in \gls{mimo}-\gls{ic} by joint active beamforming design at the transmitters and passive beamforming design at the \gls{ris}.
		The achievable rate of transmission $k$ is given by
		\begin{equation}
			R_k = \log \det \biggl(\mathbf{I} + \mathbf{W}_k {\mathbf{H}^{[kj]}}^\mathsf{H} \mathbf{Q}_k^{-1} {\mathbf{H}^{[kj]}} \mathbf{W}_k\biggr),
		\end{equation}
		where $\mathbf{Q}_k$ is the interference-plus-noise covariance matrix at receiver $k$
		\begin{equation}
			\mathbf{Q}_k = \sum_{j \ne k} {\mathbf{H}^{[kj]}} \mathbf{W}_j {\mathbf{W}_j}^\mathsf{H} {\mathbf{H}^{[kj]}}^\mathsf{H} + \eta \mathbf{I},
		\end{equation}
		and $\eta$ is the average noise power.
		The optimization problem is formulated as
		\begin{maxi!}
			{\scriptstyle{\mathbf{\Theta}, \{\mathbf{W}_k\}}}{\sum_k \rho_k R_k}{\label{op:ic_rate}}{\label{ob:ic_rate}}
			\addConstraint{\mathbf{\Theta}_g^\mathsf{H} \mathbf{\Theta}_g=\mathbf{I}, \quad \forall g}{}{}
			\addConstraint{\lVert \mathbf{W}_k \rVert _\mathrm{F}^2 \le P_k. \quad \forall k}{}{}
		\end{maxi!}
		where $\rho_k$ is the weight that denotes the priority of transmission $k$.
		The problem is non-convex and can be solved by the \gls{ao} method detailed below.

		\begin{subsubsection}{Precoder Design}
			For a given $\mathbf{\Theta}$, \eqref{op:ic_rate} reduces to conventional linear beamforming problem, for which a closed-form iterative solution has been proposed in \cite{Negro2010} based on mutual information-\gls{mmse} relationship \cite{Guo2005}.
			We restate the steps as follows.

			At iteration $r$, the \gls{mmse} combiner at receiver $k$ is
			\begin{equation}
				\mathbf{G}_k^{(r)} = {\mathbf{W}_k^{(r-1)}}^\mathsf{H} {\mathbf{H}^{[kk]}}^\mathsf{H} \bigl(\mathbf{Q}_k^{(r-1)} + {\mathbf{H}^{[kk]}} \mathbf{W}_k^{(r-1)} {\mathbf{W}_k^{(r-1)}}^\mathsf{H} {\mathbf{H}^{[kk]}}^\mathsf{H}\bigr)^{-1},
			\end{equation}
			the corresponding error matrix is
			\begin{equation}
				\mathbf{E}_k^{(r)} = \bigl(\mathbf{I} + {\mathbf{W}_k^{(r-1)}}^\mathsf{H} {\mathbf{H}^{[kk]}}^\mathsf{H} \mathbf{Q}_k^{(r-1)} {\mathbf{H}^{[kk]}} \mathbf{W}_k^{(r-1)}\bigr)^{-1},
			\end{equation}
			the \gls{mse} weight is
			\begin{equation}
				\mathbf{\Omega}_k^{(r)} = \rho_k {\mathbf{E}_k^{(r)}}^{-1},
			\end{equation}
			the Lagrange multiplier is
			\begin{equation}
				\lambda_k^{(r)} = \frac{\mathrm{tr}\bigl(\eta \mathbf{\Omega}_k^{(r)} \mathbf{G}_k^{(r)}{\mathbf{G}_k^{(r)}}^\mathsf{H} + \sum_j \mathbf{\Omega}_k^{(r)}\mathbf{T}_{kj}^{(r)} {\mathbf{T}_{kj}^{(r)}}^\mathsf{H} - \mathbf{\Omega}_j^{(r)}\mathbf{T}_{jk}^{(r)} {\mathbf{T}_{jk}^{(r)}}^\mathsf{H} \bigr)}{P_k},
			\end{equation}
			where $\mathbf{T}_{kj}^{(r)} = \mathbf{G}_k^{(r)} {\mathbf{H}^{[kj]}} \mathbf{W}_j^{(r)}$.
			The optimal precoder at transmitter $k$ is given by
			\begin{equation}
				\mathbf{W}_k^{(r)} = \Bigl(\sum_j {\mathbf{H}^{[jk]}}^\mathsf{H} {\mathbf{G}_j^{(r)}}^\mathsf{H} \mathbf{\Omega}_k^{(r)} \mathbf{G}_j^{(r)} \mathbf{H}^{[jk]} + \lambda_k^{(r)} \mathbf{I} \Bigr)^{-1} {\mathbf{H}^{[kk]}}^\mathsf{H} {\mathbf{G}_j^{(r)}}^\mathsf{H} \mathbf{\Omega}_k^{(r)}.
			\end{equation}
		\end{subsubsection}

		\begin{subsubsection}{Scattering Matrix Design}
			Once $\{\mathbf{W}_k\}$ is determined, the complex derivative of \eqref{ob:ic_rate} with respect to \gls{ris} block $g$ is
			\begin{equation}
				\frac{\partial \rho_k R_k}{\partial \mathbf{\Theta}_g^*} = \sum_k \rho_k {\mathbf{H}^{[k]}_{\mathrm{B},g}}^\mathsf{H} \mathbf{Q}_k^{-1} {\mathbf{H}^{[kk]}} \mathbf{W}_k \mathbf{E}_k \mathbf{W}_k^\mathsf{H} \bigl({\mathbf{H}_{k,g}^\mathrm{F}}^\mathsf{H} - {\mathbf{H}^{[kk]}}^\mathsf{H} \mathbf{Q}_k^{-1} \sum_{j \ne k} {\mathbf{H}^{[kj]}} \mathbf{W}_j \mathbf{W}_j^\mathsf{H} {\mathbf{H}^{[j]}_{\mathrm{F},g}}^\mathsf{H}\bigr).
				\label{eq:ic_rate_gradient_ris}
			\end{equation}
		\end{subsubsection}
		The \gls{ris} subproblem can be solved by \gls{rcg} Algorithm~\ref{ag:rcg} where line \ref{ln:gradient_euclidean} uses \eqref{eq:ic_rate_gradient_ris} explicitly.
	\end{subsection}
\end{section}


\begin{section}{Simulation Results}\label{sc:simulation}
	In this section, we provide numerical results to evaluate the proposed \gls{bd}-\gls{ris} designs.
	Consider a distance-dependent path loss model $\Lambda(d) = \Lambda_0 d^{-\gamma}$ where $\Lambda_0$ is the reference path loss at distance \qty{1}{m}, $d$ is the propagation distance, and $\gamma$ is the path loss exponent.
	The small-scale fading model is $\mathbf{H} = \sqrt{\kappa/(1+\kappa)} \mathbf{H}_\text{LoS} + \sqrt{1/(1+\kappa)} \mathbf{H}_\text{NLoS}$, where $\kappa$ is the Rician $K$-factor, $\mathbf{H}_\text{LoS}$ is the deterministic \gls{los} component, and $\mathbf{H}_\text{NLoS} \sim \mathcal{CN}(\mathbf{0}, \mathbf{I})$ is the Rayleigh component.
	We set $\Lambda_0=\qty{-30}{dB}$, $d_\mathrm{D}=\qty{14.7}{m}$, $d_\mathrm{F}=\qty{10}{m}$, $d_\mathrm{B}=\qty{6.3}{m}$, $\gamma_\mathrm{D}=3$, $\gamma_\mathrm{F}=2.4$ and $\gamma_\mathrm{B}=2$ for reference, which corresponds to a typical indoor environment with $\Lambda_\mathrm{D}=\qty{-65}{dB}$, $\Lambda_\mathrm{F}=\qty{-54}{dB}$, $\Lambda_\mathrm{B}=\qty{-46}{dB}$.
	The indirect path via \gls{ris} is thus \qty{35}{\dB} weaker than the direct path.
	Rayleigh fading (i.e., $\kappa = 0$) is assumed for all channels unless otherwise specified.

	\begin{subsection}{Algorithm Evaluation}
		\begin{table}[H]
			\footnotesize
			\caption{Average Performance of Geodesic and Non-Geodesic \gls{rcg} Algorithms on Problem \eqref{op:pareto}}
			\label{tb:complexity_test}
			\centering
			\begin{tabular}{ccccccc}
				\toprule
				\multirow{2}{*}{\gls{rcg} path} & \multicolumn{3}{c}{$N_\mathrm{S}=16$} & \multicolumn{3}{c}{$N_\mathrm{S}=256$}                                                               \\ \cmidrule(lr){2-4} \cmidrule(lr){5-7}
												& Objective                             & Iterations                             & Time [s]         & Objective        & Iterations & Time [s] \\ \midrule
				Geodesic                        & $\num{4.359e-3}$                      & 11.59                                  & $\num{1.839e-2}$ & $\num{1.163e-2}$ & 25.58      & 3.461    \\
				Non-geodesic                    & $\num{4.329e-3}$                      & 30.92                                  & $\num{5.743e-2}$ & $\num{1.116e-2}$ & 61.40      & 13.50    \\ \bottomrule
			\end{tabular}
		\end{table}

		We first compare the geodesic and non-geodesic \gls{rcg} algorithm on problem \eqref{op:pareto} in a 4T4R system with \gls{bd}-\gls{ris} group size $L=4$.
		The statistics are averaged over \num{100} independent runs.
		It is observed that the geodesic \gls{rcg} method achieves a slightly higher objective value with significantly (up to 3$\times$) lower number of iterations and shorter (up to 4$\times$) computational time than the non-geodesic method.
		The results demonstrate the efficiency of the proposed geodesic \gls{rcg} algorithm especially in large-scale \gls{bd}-\gls{ris} design problems.
		If the scattering matrix is constrained to be symmetric, one can project the solution to the feasible domain by $\mathbf{\Theta} \gets (\mathbf{\Theta} + \mathbf{\Theta}^\mathsf{T})/2$.
	\end{subsection}


	\begin{subsection}{\glsfmtshort{mimo}-\glsfmtshort{pc}}
		\begin{subsubsection}{Pareto Frontier}
			\begin{figure}[H]
				\centering
				\subfloat[$2 \times 32 \times 2$ (no direct)\label{fg:singular_pareto_sx32_nd}]{
					\resizebox{0.45\columnwidth}{!}{
						\input{assets/chapter_5/singular_pareto_sx32_nd.tex}
					}
				}
				\subfloat[$2 \times 32 \times 2$\label{fg:singular_pareto_sx32}]{
					\resizebox{0.45\columnwidth}{!}{
						\input{assets/chapter_5/singular_pareto_sx32.tex}
					}
				}
				\\
				\subfloat[$2 \times 64 \times 2$\label{fg:singular_pareto_sx64}]{
					\resizebox{0.45\columnwidth}{!}{
						\input{assets/chapter_5/singular_pareto_sx64.tex}
					}
				}
				\subfloat[$2 \times 128 \times 2$\label{fg:singular_pareto_sx128}]{
					\resizebox{0.45\columnwidth}{!}{
						\input{assets/chapter_5/singular_pareto_sx128.tex}
					}
				}
				\caption{Pareto frontiers of singular values of a 2T2R channel reshaped by a \gls{ris}.}
				\label{fg:singular_pareto}
			\end{figure}
			% Fig. \ref{fg:singular_pareto} shows the Pareto frontiers of channel singular values reshaped by \gls{bd}-\gls{ris}.
			Fig. \ref{fg:singular_pareto} shows the Pareto singular values of a 2T2R \gls{mimo} reshaped by a \gls{ris}.
			When the direct link is absent, the achievable regions in Fig. \subref*{fg:singular_pareto_sx32_nd} are shaped like pizza slices.
			This is because $\sigma_1(\mathbf{H}) \ge \sigma_2(\mathbf{H}) \ge 0$ and there exists a trade-off between the alignment of two subspaces.
			%  since $\sigma_1(\mathbf{H}) \ge \sigma_2(\mathbf{H}) \ge 0$.
			We observe that the smallest singular value can be enhanced up to \num{2e-4} by diagonal \gls{ris} and \num{3e-4} by unitary \gls{bd}-\gls{ris}, corresponding to a \qty{50}{\percent} gain.
			When the direct link is present, the shape of the singular value region depends heavily on the relative strength of the indirect link.
			In Fig. \subref*{fg:singular_pareto_sx32}, a 32-element \gls{ris} is insufficient to compensate the \qty{35}{dB} path loss imbalance and results in a limited singular value region that is symmetric around the direct point.
			As the group size $L$ increases, the shape of the region evolves from elliptical to square.
			This transformation not only improves the dynamic range of $\sigma_1(\mathbf{H})$ and $\sigma_2(\mathbf{H})$ by \qty{22}{\percent} and \qty{38}{\percent}, but also provides a better trade-off in manipulating both singular values.
			It suggests the design freedom from larger group size allows better alignment of multiple subspaces simultaneously.
			% not only provides a better trade-off in subchannel manipulation but also improves the dynamic range of $\sigma_1(\mathbf{H})$ and $\sigma_2(\mathbf{H})$ by \qty{22}{\percent} and \qty{38}{\percent}, respectively.
			The singular value region also enlarges as the number of scattering elements $N_\mathrm{S}$ increases.
			In particular, Fig. \subref*{fg:singular_pareto_sx128} shows that the equivalent channel can be completely nulled (corresponding to the origin) by a 128-element \gls{bd}-\gls{ris} but not by a diagonal one.
			Those results demonstrate the superior channel shaping capability of \gls{bd}-\gls{ris} and emphasizes the importance of cooperative wave scattering.
		\end{subsubsection}

		\begin{subsubsection}{Analytical Bounds and Numerical Results}
			\begin{figure}[H]
				\centering
				\subfloat[$4 \times 32 \times 4$ (rank-1)\label{fg:singular_bound_rank1_sx32}]{
					\resizebox{0.45\columnwidth}{!}{
						\input{assets/chapter_5/singular_bound_rank1_sx32.tex}
					}
				}
				\subfloat[$4 \times 64 \times 4$ (rank-1)\label{fg:singular_bound_rank1_sx64}]{
					\resizebox{0.45\columnwidth}{!}{
						\input{assets/chapter_5/singular_bound_rank1_sx64.tex}
					}
				}
				\\
				\subfloat[$4 \times 128 \times 4$ (rank-2)\label{fg:singular_bound_rank2_sx128}]{
					\resizebox{0.45\columnwidth}{!}{
						\input{assets/chapter_5/singular_bound_rank2_sx128.tex}
					}
				}
				\subfloat[$4 \times 256 \times 4$ (rank-4)\label{fg:singular_bound_rank4_sx256}]{
					\resizebox{0.45\columnwidth}{!}{
						\input{assets/chapter_5/singular_bound_rank4_sx256.tex}
					}
				}
				\caption{
					Achievable channel singular values: analytical bounds (green lines) and numerical optimization results (blue and red bars).
					The intersections of the blue and red bars denote the singular values of the direct channel.
					The blue (resp. red) bars are obtained by solving problem \eqref{op:pareto} with $\rho_n = -1$ (resp. $+1$) and $\rho_{n'} = 0$, $\forall n' \ne n$.
					`D' means diagonal \gls{ris} and `BD' means fully-connected \gls{bd}-\gls{ris}.
					`rank-$k$' refers to the rank of the forward channel.
				}
				\label{fg:singular_bound}
			\end{figure}
			Fig. \ref{fg:singular_bound} illustrates the analytical singular value bounds in Proposition \ref{pp:rank_deficient} and the numerical results obtained by solving problem \eqref{op:pareto} with $\rho_n = \pm 1$ and $\rho_{n'} = 0$, $\forall n' \ne n$.
			Here we assme a rank-$k$ forward channel without loss of generality.
			When the \gls{ris} is in the vicinity of the transmitter, Figs. \subref*{fg:singular_bound_rank1_sx32} and \subref*{fg:singular_bound_rank1_sx64} show that the achievable channel singular values indeed satisfy Corollary \ref{co:los}, namely $\sigma_1(\mathbf{H}) \ge \sigma_1(\mathbf{T})$, $\sigma_2(\mathbf{T}) \le \sigma_2(\mathbf{H}) \le \sigma_1(\mathbf{T})$, etc.
			It is obvious that \gls{bd}-\gls{ris} can approach those bounds better than diagonal \gls{ris} especially for a small $N_\mathrm{S}$.
			% Another example is given in Fig. \subref*{fg:singular_bound_rank2_sx128} where the forward channel is rank-2.
			Another example is given in Fig. \subref*{fg:singular_bound_rank2_sx128} with rank-2 forward channel.
			% The observations align with Proposition \ref{pp:rank_deficient} that t
			The first two channel singular values are unbounded above and bounded below by the first two singular values of $\mathbf{T}$, while the last two singular values can be suppressed to zero and bounded above by the first two singular values of $\mathbf{T}$.
			Those observations align with Proposition \ref{pp:rank_deficient} and Corollary \ref{co:extreme}.
			Finally, Fig. \subref*{fg:singular_bound_rank4_sx256} confirms there are no extra singular value bounds when both forward and backward channels are full-rank.
			This can be predicted from \eqref{eq:auxiliary_matrix} where the compact singular matrix $\mathbf{V}_\mathrm{F}$ becomes unitary and $\mathbf{T}=\mathbf{0}$.
			The numerical results are consistent with the analytical bounds, and we conclude that the channel shaping advantage of \gls{bd}-\gls{ris} over diagonal \gls{ris} scales with forward and backward channel ranks.

			\begin{figure}[H]
				\centering
				\subfloat[$1 \times 256 \times 1$ (no direct)\label{fg:power_bond_txrx1_nd}]{
					\resizebox{0.45\columnwidth}{!}{
						\input{assets/chapter_5/power_bond_txrx1_nd.tex}
					}
				}
				\subfloat[$4 \times 256 \times 4$ (no direct)\label{fg:power_bond_txrx4_nd}]{
					\resizebox{0.45\columnwidth}{!}{
						\input{assets/chapter_5/power_bond_txrx4_nd.tex}
					}
				}
				\caption{
					Average maximum channel power versus \gls{bd}-\gls{ris} group size and \gls{mimo} dimensions.
					`Cascaded' refers to the available power of the cascaded channel, i.e., the sum of (sorted) element-wise power product of backward and forward subchannels.
				}
				\label{fg:power_bond}
			\end{figure}

			Fig. \ref{fg:power_bond} compares the analytical channel power bound in Corollary \ref{co:sum_power} and the numerical results obtained by solving problem \eqref{op:power_passive} when the direct link is absent.
			Here, a fully-connected \gls{bd}-\gls{ris} can attain the upper bound either in closed form \eqref{eq:scattering_fc_optimal} or via optimization approach \eqref{eq:scattering_power}.
			% Recall that a fully-connected \gls{bd}-\gls{ris} in closed form \eqref{eq:scattering_fc_tight} can attain the bound.
			% We observe that the bound is also tight for fully-connected \gls{bd}-\gls{ris}.
			% Increasing the group size of \gls{bd}-\gls{ris} significantly improves the channel power especially for a large \gls{mimo}.
			% Apparently, \gls{bd}-\gls{ris} with a larger group size can approach the bound better, and a fully-connected one can achieve the bound.
			% The bound here indicates how much power can potentially be extracted from the forward and backward channels.
			% We observe that using a larger group size can approach the bound better, and full
			% We observe that the bound is tight for fully-connected \gls{bd}-\gls{ris} in \gls{siso} and \gls{mimo}.
			For the \gls{siso} case in Fig. \subref*{fg:power_bond_txrx1_nd}, the maximum channel power is approximately \num{4e-6} by diagonal \gls{ris} and \num{6.5e-6} by fully-connected \gls{bd}-\gls{ris}, corresponding to a \qty{62.5}{\percent} gain.
			This aligns with the asymptotic \gls{bd}-\gls{ris} scaling law derived for \gls{siso} in \cite{Shen2020a}.
			Interestingly, the gain surges to \qty{270}{\percent} in 4T4R \gls{mimo} as shown in Fig. \subref*{fg:power_bond_txrx4_nd}.
			This is because subspace alignment boils down to phase matching in \gls{siso} such that both triangular and Cauchy-Schwarz inequalities in \cite[(50)]{Shen2020a} can be simultaneously tight regardless of the group size.
			That is, diagonal \gls{ris} is sufficient for subspace alignment in \gls{siso} while the \qty{62.5}{\percent} gain from \gls{bd}-\gls{ris} comes purely from subchannel rearrangement.
			Now consider a diagonal \gls{ris} in \gls{mimo}.
			Each element can only apply a common phase shift to the associated rank-1 $N_\mathrm{R} \times N_\mathrm{T}$ indirect channel.
			Therefore, perfect subspace alignment of indirect channels through different elements is generally impossible.
			It means the disadvantage of diagonal \gls{ris} in subspace alignment and subchannel rearrangement scales with \gls{mimo} dimensions.
			We thus conclude that the power gain of \gls{bd}-\gls{ris} scales with group size and \gls{mimo} dimensions.
		\end{subsubsection}


		% \begin{subsection}{Channel Singular Values Redistribution}
		% 	% The Pareto singular value frontiers of a $2 \times 2$ \gls{mimo} with a 32-element \gls{bd}-\gls{ris} is shown in Fig.~\ref{fg:singular_pareto}.
		% 	% The Pareto frontier of channel singular values reshaped by \gls{bd}-\gls{ris} is shown in Fig. \ref{fg:singular_pareto}.
		% 	% The Pareto frontier of channel singular values reshaped by \gls{bd}-\gls{ris} is shown in Fig. \ref{fg:singular_pareto}.
		% 	% Clearly, a larger group size provides
		% 	% and evolving trend of channel singular values are shown in Fig.~\ref{fg:singular_pareto} and \ref{fg:pc_singular_bound}.
		% 	% Clearly, \gls{bd}-\gls{ris} with a larger group size can redistribute the channel singular values to a wider range.
		% \end{subsection}

		\begin{subsubsection}{Channel Power and Achievable Rate Maximization}
			\begin{figure}[H]
				\centering
				\subfloat[$16 \times N_\mathrm{S} \times 16$ (no direct)\label{fg:power_sx_txrx16_nd}]{
					\resizebox{0.45\columnwidth}{!}{
						\input{assets/chapter_5/power_sx_txrx16_nd.tex}
					}
				}
				\subfloat[$16 \times N_\mathrm{S} \times 16$\label{fg:power_sx_txrx16}]{
					\resizebox{0.45\columnwidth}{!}{
						\input{assets/chapter_5/power_sx_txrx16.tex}
					}
				}
				\caption{
					Average maximum channel power versus \gls{ris} configuration.
					`OP-left' and `OP-right' refer to the suboptimal solutions to problem \eqref{op:power_passive} by lossy transformation \eqref{op:standard_orthogonal_procrustes} where $\mathbf{\Theta}$ is to the left and right of the product, respectively.
				}
				\label{fg:power_sx}
			\end{figure}

			We first focus on channel power gain maximization problem \eqref{op:power_passive}.
			Fig. \ref{fg:power_sx} shows the achievable channel power under different \gls{ris} configurations.
			An interesting observation is that the relative power gain of \gls{bd}-\gls{ris} over diagonal \gls{ris} is even larger with direct link.
			For example, a 64-element fully \gls{bd}-\gls{ris} can almost provide the same channel power as a 256-element diagonal \gls{ris} in Fig. \ref{fg:power_sx_txrx16}, but not in Fig. \ref{fg:power_sx_txrx16_nd}.
			% This again credits to the superior subspace alignment capability of \gls{bd}-\gls{ris}.
			This is because the \gls{ris} needs to balance the multiplicative forward-backward combining and the additive direct-indirect combining, such that the subspace alignment advantage of \gls{bd}-\gls{ris} becomes more pronounced.
			We also notice that the suboptimal solutions \eqref{eq:orthogonal_procrustes_solution} for fully-connected \gls{bd}-\gls{ris} by lossy transformation \eqref{op:standard_orthogonal_procrustes} are very close to optimal especially for a large $N_\mathrm{S}$.

			\begin{figure}[H]
				\centering
				\subfloat[$4 \times 128 \times 4$\label{fg:rate_beamforming}]{
					\resizebox{0.45\columnwidth}{!}{
						\input{assets/chapter_5/rate_beamforming.tex}
					}
				}
				\subfloat[$N_\mathrm{T} \times 128 \times N_\mathrm{R}$\label{fg:rate_txrx}]{
					\resizebox{0.45\columnwidth}{!}{
						\input{assets/chapter_5/rate_txrx.tex}
					}
				}
				\\
				\subfloat[$4 \times N_\mathrm{S} \times 4$\label{fg:rate_sx}]{
					\resizebox{0.45\columnwidth}{!}{
						\input{assets/chapter_5/rate_sx.tex}
					}
				}
				\subfloat[$4 \times 128 \times 4$\label{fg:rate_kfactor}]{
					\resizebox{0.45\columnwidth}{!}{
						\input{assets/chapter_5/rate_kfactor.tex}
					}
				}
				\caption{
					Average achievable rate versus \gls{mimo} and \gls{ris} configurations.
					The noise power is $\eta = \qty{-75}{dB}$, corresponding to a direct \gls{snr} of \num{-10} to \qty{30}{dB}.
					`Alternate' refers to the alternating optimization and `Decouple' refers to the low-complexity design.
					`D' means diagonal \gls{ris} and `BD' means fully-connected \gls{bd}-\gls{ris}.
				}
				\label{fg:rate}
			\end{figure}

			Fig. \ref{fg:rate} presents the achievable rate under different \gls{mimo} and \gls{ris} configurations.
			At a transmit power of \qty{10}{dB}, Fig. \subref*{fg:rate_beamforming} shows that introducing a 128-element diagonal \gls{ris} to 4T4R \gls{mimo} can improve the achievable rate from \qty{22.2}{bps/Hz} to \qty{29.2}{bps/Hz} ($+\qty{31.5}{\percent}$).
			In contrast, a \gls{bd}-\gls{ris} of group size 4 and 128 can further improve the rate to \qty{32.1}{bps/Hz} ($+\qty{44.6}{\percent}$) and \qty{34}{bps/Hz}  ($+\qty{53.2}{\percent}$), respectively.
			Interestingly, the gap between the optimal \gls{ao} approach \eqref{op:rate_passive}--\eqref{eq:precoder_eigenmode} and the low-complexity solution \eqref{eq:scattering_power} and \eqref{eq:precoder_eigenmode} narrows as the group size increases, and completely vanishes for a fully-connected \gls{bd}-\gls{ris}.
			This implies that the \gls{ris}-transceiver design can often be decoupled via channel shaping with marginal performance loss.
			Figs. \subref*{fg:rate_txrx} and \subref*{fg:rate_sx} also confirm the advantage of \gls{bd}-\gls{ris} grows with the number of transmit, scatter, and receive antennas.
			In the low power regime (\num{-20} to \qty{-10}{dB}), the slope of the achievable rate is significantly larger with \gls{bd}-\gls{ris}, suggesting that multiple streams can be activated at a much lower \gls{snr}.
			This is because \gls{bd}-\gls{ris} not only spreads the channel singular values to a wider range, but also provides a better trade-off between subchannels (c.f. Fig. \ref{fg:singular_pareto}).
			Finally, Fig. \subref*{fg:rate_kfactor} shows that the gap between diagonal and \gls{bd}-\gls{ris} narrows as the Rician $K$-factor increases and becomes indistinguishable in \gls{los} environment.
			The observation is expected from previous studies \cite{Shen2020a,Li2023b,Nerini2023} and aligns with Corollary \ref{co:los}, which suggests that the \gls{bd}-\gls{ris} should be deployed in rich-scattering environments to exploit its channel shaping potential.
		\end{subsubsection}
	\end{subsection}

	\begin{subsection}{\glsfmtshort{mimo}-\glsfmtshort{ic}}
		\begin{subsubsection}{Leakage Interference Minimization}
			\begin{figure}[H]
				\centering
				\resizebox{0.65\columnwidth}{!}{
					\input{assets/chapter_5/ic_interference_sx.tex}
				}
				\caption{Average leakage interference versus \gls{ris} elements $N_\mathrm{S}$ and group size $L$. Transmitters and receivers are randomly generated in a disk of radius 50 m centered at the \gls{ris}. $(N_\mathrm{T}, N_\mathrm{R}, N_\mathrm{E}, K) = (8, 4, 3, 5)$.}
				\label{sm:ic_interference_sx}
			\end{figure}
			Fig.~\ref{sm:ic_interference_sx} illustrates how \gls{bd}-\gls{ris} helps to reduce the leakage interference.
			In this case, a fully-connected $2^n$-element \gls{bd}-\gls{ris} is almost as good as a diagonal $2^{n+2}$-element \gls{ris} in terms of leakage interference.
			Interestingly, the result suggests that \gls{bd}-\gls{ris} can achieve a higher \gls{dof} than diagonal \gls{ris} in \gls{mimo}-\gls{ic}, which is not the case in \gls{mimo}-\gls{pc} as discussed in \ref{pp:dof}.
			This demonstrates the potential of \gls{bd}-\gls{ris} in interference alignment or cancellation.
		\end{subsubsection}

		\begin{subsubsection}{Weighted Sum-Rate Maximization}
			\begin{figure}[H]
				\centering
				\resizebox{0.65\columnwidth}{!}{
					% This file was created by matlab2tikz.
%
%The latest updates can be retrieved from
%  http://www.mathworks.com/matlabcentral/fileexchange/22022-matlab2tikz-matlab2tikz
%where you can also make suggestions and rate matlab2tikz.
%
\definecolor{mycolor1}{rgb}{0.00000,0.44706,0.74118}%
\definecolor{mycolor2}{rgb}{0.85098,0.32549,0.09804}%
%
\begin{tikzpicture}

\begin{axis}[%
width=9.509cm,
height=7.5cm,
at={(0cm,0cm)},
scale only axis,
xmin=-20,
xmax=20,
xlabel style={font=\color{white!15!black}},
xlabel={Transmit Power [dB]},
ymin=6.48635941156549,
ymax=201.987610000957,
ylabel style={font=\color{white!15!black}},
ylabel={Weighted Sum-Rate [bit/s/Hz]},
axis background/.style={fill=white},
xmajorgrids,
ymajorgrids,
legend style={at={(0.03,0.97)}, anchor=north west, legend cell align=left, align=left, draw=white!15!black}
]
\addplot [color=black, line width=2.0pt]
  table[row sep=crcr]{%
-20	6.48635941156549\\
-10	21.5214871275404\\
0	45.0265046723773\\
10	72.422439336639\\
20	100.563098683264\\
};
\addlegendentry{$N_\mathrm{S} = 0$}

\addplot [color=mycolor1, line width=2.0pt, mark=o, mark options={solid, mycolor1}]
  table[row sep=crcr]{%
-20	7.02117917137241\\
-10	22.8962031384014\\
0	47.4237237465048\\
10	75.0272993980888\\
20	103.194018898239\\
};
\addlegendentry{$(N_\mathrm{S}, L) = (32, 1)$}

\addplot [color=mycolor1, dashed, line width=2.0pt, mark=o, mark options={solid, mycolor1}]
  table[row sep=crcr]{%
-20	7.37089652749512\\
-10	23.9754474544347\\
0	48.9396257237975\\
10	76.6135852502643\\
20	104.777858029664\\
};
\addlegendentry{$(N_\mathrm{S}, L) = (32, 4)$}

\addplot [color=mycolor1, dotted, line width=2.0pt, mark=o, mark options={solid, mycolor1}]
  table[row sep=crcr]{%
-20	7.92080329467276\\
-10	25.4942113008919\\
0	50.8392698133478\\
10	78.6205392734233\\
20	106.807080544116\\
};
\addlegendentry{$(N_\mathrm{S}, L) = (32, 32)$}

\addplot [color=mycolor2, line width=2.0pt, mark=+, mark options={solid, mycolor2}]
  table[row sep=crcr]{%
-20	10.4149588179023\\
-10	31.4446145028552\\
0	59.6517377835957\\
10	89.3374682956471\\
20	119.211944846161\\
};
\addlegendentry{$(N_\mathrm{S}, L) = (256, 1)$}

\addplot [color=mycolor2, dashed, line width=2.0pt, mark=+, mark options={solid, mycolor2}]
  table[row sep=crcr]{%
-20	13.5990610062377\\
-10	38.8358018247535\\
0	70.6878846178664\\
10	103.750160929285\\
20	136.952951531633\\
};
\addlegendentry{$(N_\mathrm{S}, L) = (256, 4)$}

\addplot [color=mycolor2, dotted, line width=2.0pt, mark=+, mark options={solid, mycolor2}]
  table[row sep=crcr]{%
-20	21.0669283703778\\
-10	56.5659878613809\\
0	102.883995624597\\
10	152.237987738074\\
20	201.987610000957\\
};
\addlegendentry{$(N_\mathrm{S}, L) = (256, 256)$}

\end{axis}

\begin{axis}[%
width=12.27cm,
height=9.202cm,
at={(-1.595cm,-1.012cm)},
scale only axis,
xmin=0,
xmax=1,
ymin=0,
ymax=1,
axis line style={draw=none},
ticks=none,
axis x line*=bottom,
axis y line*=left
]
\end{axis}
\end{tikzpicture}%

				}
				\caption{Average weighted sum-rate versus average transmit power, \gls{ris} elements $N_\mathrm{S}$, and group size $L$. $(N_\mathrm{T}, N_\mathrm{R}, N_\mathrm{E}, K) = (8, 4, 3, 5)$.}
				\label{sm:ic_rate_sx}
			\end{figure}

			Fig.~\ref{sm:ic_rate_sx} shows the average weighted sum-rate versus average transmit power and \gls{ris} configuration.
			We observe that the advantage of \gls{bd}-\gls{ris} over diagonal \gls{ris} becomes more pronounced when the number of reflecting elements increases.
			This is because the total number of subchannels in the system grows linearly with $N_\mathrm{S}$ such that the gain from subchannel rearrangement capability of \gls{bd}-\gls{ris} is amplified.
			It suggests cooperative scattering among elements is particularly important for large-scale \gls{ris} deployments in \gls{mimo}-\gls{ic} systems, where the performance-complexity trade-off is more favorable than \gls{siso} and single-user scenarios.

			\begin{figure}[H]
				\centering
				\resizebox{0.65\columnwidth}{!}{
					\input{assets/chapter_5/ic_rate_user.tex}
				}
				\caption{Average weighted sum-rate versus transmitter-receiver pairs $K$, \gls{ris} elements $N_\mathrm{S}$, and group size $L$. $(N_\mathrm{T}, N_\mathrm{R}, N_\mathrm{E}) = (4, 4, 3)$, $P = \qty{25}{dB}$, and $\rho_k = 1, \ \forall k$.}
				\label{sm:ic_rate_user}
			\end{figure}

			Fig.~\ref{sm:ic_rate_user} shows the average weighted sum-rate versus the number of transmitter-receiver pairs and \gls{ris} configuration.
			We observe that introducing a \gls{ris} to \gls{ic} systems is helpful to mitigate the rate saturation effect as the number of users increases.
			In the overloaded regime ($K \ge 8$), \gls{bd}-\gls{ris} provides a larger slope of the achievable rate than diagonal \gls{ris}.
			This is because the \gls{bd}-\gls{ris} can better align the interference subspaces of different users.
			At $K=10$, adding a 64-element diagonal and \gls{bd}-\gls{ris} can improve the weighted sum-rate by \qty{23}{\percent} and \qty{43}{\percent}, respectively.
			% These results highlight the importance of \gls{ris}, and particularly t configurations, for maximizing the capacity of overloaded multi-user interference channels
			These results provide valuable insights for practical \gls{ris} design in dense connection scenarios, where \gls{bd} configurations can significantly enhance the network capacity.
		\end{subsubsection}
	\end{subsection}
\end{section}

\begin{section}{Conclusion}
	This chapter studied the channel shaping capability of \gls{ris} in \gls{mimo}-\gls{pc} and \gls{ic}.
	We considered a general \gls{bd} architecture that allows elements within the same group to cooperate, enabling advanced wave manipulation than conventional \gls{ris}.
	% This translates to a wider dynamic range (with better trade-off) of channel singular values with significant power and rate gains in \gls{pc}, and lower leakage interference and higher \gls{dof} in \gls{ic}.
	% Those advantages are more pronounced in rich-scattering environments and large-scale systems.
	An efficient \gls{rcg} algorithm was proposed for smooth \gls{bd}-\gls{ris} optimization problems, which offers lower computation complexity and faster convergence than existing methods.
	For \gls{pc}, we characterized the Pareto frontiers of channel singular values via optimization approach and provided analytical bounds in rank-deficient and fully-connected scenarios.
	Two joint beamforming designs were proposed for rate maximization problem, one based on \gls{ao} for optimal performance and one based on channel shaping for lower complexity.
	For \gls{ic}, we highlighted that \gls{bd}-\gls{ris} can reduce the leakage interference and improve the weighted sum-rate.
	Extensive simulation results and theoretical analysis showed that the advantage of \gls{bd}-\gls{ris} scales with the number of elements, group size, \gls{mimo} dimensions, and channel diversity, thanks to its superior subspace alignment and subchannel rearrangement capability.

	One future direction is incorporating one \gls{ris} at transmitter and one at receiver for stronger channel manipulation, which may fully align both direct-indirect and forward-backward subspaces simultaneously.
	% Finally, integrating the \gls{ris} into the \gls{mimo} precoding and combining design is also an interesting topic for future research.
	% into the \gls{mimo} precoding and combining design is also an interesting topic for future research.
	% even better channel shaping can be achieved by incorporating the \gls{ris} into the \gls{mimo} precoding and combining design.
\end{section}
