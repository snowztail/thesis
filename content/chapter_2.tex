%!TEX root = ../thesis.tex

\graphicspath{{assets/chapter_2/}}

\chapter{Background and Literature Review}\label{ch:background}

\begin{section}{\glsfmtfull{ris}}
	% \begin{subsection}{Programmable Metamaterials}

	% \end{subsection}

	\begin{subsection}{Hardware Implementation}
		\begin{subsubsection}{Programmable Metamaterials}
			Metamaterials refer to artificial structures engineered for unusual properties that may not be found in nature.
			The concept was initially proposed by Victor Veselago in 1967, who conjectured the existence of mediums with negative dielectric constant $\epsilon < 0$ and negative permeability $\mu < 0$ \cite{Veselago1968}.
			Such metamaterials are known as ``negative-index'' because the refraction index is defined as the \emph{negative} square root $n = - \sqrt{\epsilon \mu} < 0$, in order to be consistent with Maxwell's equations.
			It was not until 1999 that their feasibility was experimentally demonstrated by John Pendry at Imperial College using split-ring resonators \cite{Pendry1999}.
			Since then, metamaterials have attracted significant interests due to their counterintuitive properties, to name a few:
			\begin{itemize}
				\item \emph{Negative refraction:} As shown in Fig. \ref{fg:nim_refraction}, the incident and refracted rays stay at the same side of the normal axis. This phenomenon is in contrast to the usual refraction but can still be predicted from Snell's law
				\begin{equation}
					\frac{\sin \theta_1}{\sin \theta_2} = n.
				\end{equation}
				\begin{figure}[H]
					\centering
					\resizebox{0.5\columnwidth}{!}{
						\tikzstyle{glass}=[color=red!10]

\tikzset{
	photon/.style={
			draw=black,decorate,
			decoration={snake, segment length=3mm, amplitude=1mm,post length=2mm}
		}
}

% REFLECTION & REFRACTION
\begin{tikzpicture}
	\def\L{7}   % width interface
	\def\l{2}   % length ray
	\def\t{2}   % depth glass gradient
	\def\x{3}   % spacing between plots
	\def\h{1.5}   % bisector height
	\def\f{0.3}   % fraction of interface to the left
	%   \def\na{1.0}  % air
	%   \def\ng{-2}  % glass
	\def\n{-0.5}
	\def\anga{45} % angle of incident ray
	%   \def\angg{asin(\na/\ng*sin(\anga))}
	\def\angg{asin(\n*sin(\anga))}
	\coordinate (O) at (0,0);            % point of contact
	\coordinate (I) at (90+\anga:\l);    % point incident (top left)
	\coordinate (M) at (90-\anga:\l);    % point reflected (top right)
	\coordinate (F) at ({-90+\angg}:\l); % point refracted (bottom)
	\coordinate (L) at (-\f*\L,0);       % left point interface
	\coordinate (R) at ({0.5*\L},0);  % right point interface
	\coordinate (T) at (0,\h);           % top middle point (bisector)
	\coordinate (B) at (0,-1.0*\h);      % bottom middle point (bisector)

	\coordinate (T1) at ([shift={(\x,0)}] T);
	\coordinate (B1) at ([shift={(\x,0)}] B);
	\coordinate (I1) at ([shift={(\x,0)}] I);
	\coordinate (O1) at ([shift={(\x,0)}] O);
	\coordinate (F1) at ([shift={(\x,0)}] F);

	% MEDIUM
	\fill[glass] (L) rectangle++ (\L,-\t); % glass gradient
	%   \node[below left=2] at (R) {$n=-2$};
	\node at (1.2,-1.7) {$n=-2$};

	% LINES
	\draw[dashed] (T) -- (B); % bisector
	\draw[-latex,black,semithick] (I) -- (O); % incoming ray
	\draw[-latex,black,semithick] (O) -- (F); % refracted ray

	\draw[dashed] (T1) -- (B1); % refracted ray
	\draw[-latex,photon,blue,semithick] (I1) -- (O1); % incoming ray
	\draw[-latex,photon,blue,semithick] (F1) -- (O1); % refracted ray

	% ANGLES
	\draw pic["$\theta_1$",draw=black,angle radius=28,angle eccentricity=1.3] {angle = T--O--I};
	\draw pic["$\theta_2$",draw=black,angle radius=35,angle eccentricity=1.25] {angle = F--O--B};

	% TEXTS
	\node[align=center,below right] at (O) {Rays\\(energy flow)};
	\node[align=center,below right,blue] at (O1) {Wave\\vectors};
	%   \draw pic["$\theta_1$",draw=black,angle radius=28,angle eccentricity=1.3] {angle = T1--O1--I1};
	%   \draw pic["$\theta_2$",draw=black,angle radius=35,angle eccentricity=1.25] {angle = F1--O1--B1};

\end{tikzpicture}

					}
					\caption{Refraction of electromagnetic waves at the interface between a positive-index and a negative-index material.}
					\label{fg:nim_refraction}
				\end{figure}
				\item \emph{Opposite wave direction:} As shown in Fig. \ref{fg:nim_flows}, the wave vector and energy flow (indicated by the Poynting vector) are opposite to each other in a negative-index material. This can be inferred from the electric field equation
				\begin{equation}
					\vec{E} = \vec{E}_0 \exp(\jmath k z - \jmath \omega t)
				\end{equation}
				where $k = k_0 n < 0$ is the wavenumber, $\vec{E}_0$ and $k_0$ are the free-space reference, $z$ is the propagation distance, $\omega$ is the angular frequency, and $t$ is the time.
				Negative-index materials are thus also called ``left-hand'' because the propagation direction of the electric and magnetic fields can be determined by a left-hand rule.
				\begin{figure}[H]
					\centering
					\resizebox{0.5\columnwidth}{!}{
						\begin{tikzpicture}[background rectangle/.style={fill=blue!10}, show background rectangle]

	\def\f{1/exp(((\x)^2)/2)}

	\begin{axis}[thick,ticks=none,domain=-pi:pi,samples=100,axis x line*=middle,axis y line=none,xlabel=\empty,xmin=-4,xmax=4,ymax=4,ymin=-2]

		\addplot[smooth, color=blue] (\x,{sin((9*(deg(x))) )*\f}) coordinate[pos=0.42] (W);
		\addplot[smooth, color=black] (\x,\f) coordinate[pos=0.6] (E);
		\addplot[smooth, color=black] (\x,-\f);

	\end{axis}
	\draw[-latex,thick] (E) to ++(1,0) node[anchor=west] {energy flow};
	\draw[-latex,thick,blue] (W) to ++(-1,0) node[anchor=east,blue] {wave direction};
\end{tikzpicture}

					}
					\caption{Wave and energy flows in a negative-index material.}
					\label{fg:nim_flows}
				\end{figure}
			\end{itemize}
		\end{subsubsection}

		\begin{subsubsection}{Physical Architecture}

		\end{subsubsection}
	\end{subsection}

	\begin{subsection}{Wave Scattering Model}
		\begin{subsubsection}{Active and Passive Elements}

		\end{subsubsection}

		\begin{subsubsection}{Independent Scattering: Diagonal Model}

		\end{subsubsection}

		\begin{subsubsection}{Cooperative Scattering: \glsfmtfull{bd} Model}

		\end{subsubsection}

		\begin{subsubsection}{Impact of Radiation Pattern and Circuit Topology}

		\end{subsubsection}
	\end{subsection}

	\begin{subsection}{Comparison with Reflectarrays and Relays}
		\begin{subsubsection}{Functionality}

		\end{subsubsection}

		\begin{subsubsection}{Reconfigurability}

		\end{subsubsection}

		\begin{subsubsection}{Complexity}

		\end{subsubsection}

		\begin{subsubsection}{Power Consumption}

		\end{subsubsection}

		\begin{subsubsection}{Dimension and Cost}

		\end{subsubsection}

		\begin{subsubsection}{Applications}

		\end{subsubsection}
	\end{subsection}
\end{section}

\begin{section}{\glsfmtfull{wpt}}
	\begin{subsection}{Near- and Far-Field Techniques}

	\end{subsection}

	\begin{subsection}{Modules and Coupling Effect}
		\begin{subsubsection}{Block Diagram}

		\end{subsubsection}

		\begin{subsubsection}{Blockwise Coupling}

		\end{subsubsection}
	\end{subsection}

	\begin{subsection}{Non-Linear Harvester Behavior}
		\begin{subsubsection}{Rectifier Circuits}

		\end{subsubsection}

		\begin{subsubsection}{Operation Regions and Signal Models}

		\end{subsubsection}

		\begin{subsubsection}{Impact on Waveform Selection}

		\end{subsubsection}
	\end{subsection}
\end{section}

\begin{section}{\glsfmtfull{swipt}}
	\begin{subsection}{\glsfmtfull{r-e} Tradeoff}

	\end{subsection}

	\begin{subsection}{Receiver Architectures}

	\end{subsection}
\end{section}

\begin{section}{\glsfmtfull{bc}}
	\begin{subsection}{\glsfmtfull{mbc}}

	\end{subsection}

	\begin{subsection}{\glsfmtfull{bbc}}

	\end{subsection}

	\begin{subsection}{\glsfmtfull{ambc}}

	\end{subsection}

	\begin{subsection}{\glsfmtfull{sr}}

	\end{subsection}
\end{section}

% \begin{section}{\glsfmtfull{im}}

% \end{section}

\begin{section}{\glsfmtfull{mimo}}
	\begin{subsection}{\glsfmtfull{pc}: Channel Shaping}

	\end{subsection}

	\begin{subsection}{\glsfmtfull{ic}: Interference Alignment}

	\end{subsection}
\end{section}
